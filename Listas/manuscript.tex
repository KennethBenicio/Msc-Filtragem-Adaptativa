\documentclass[a4paper,10pt]{article}
\usepackage[portuguese]{babel}

\usepackage{cite}
\usepackage{xcolor}
\usepackage{graphicx}
\usepackage{fancyhdr}
\usepackage[shortlabels]{enumitem}
\usepackage{amstext,amsmath,amssymb}

\usepackage{sectsty}
\sectionfont{\LARGE}

\setlength{\oddsidemargin}{0cm} %
\setlength{\evensidemargin}{0cm} %
\setlength{\topmargin}{0cm} %
\setlength{\textwidth}{16cm} %
\setlength{\textheight}{22.5cm} %

\pagestyle{empty}
\newcommand{\assunto}{Filtragem Adaptativa}

\sloppy

\begin{document}
	
	\thispagestyle{empty}

\begin{center}
  
    \includegraphics[scale=0.10]{figs/icon.png}
    
    \LARGE{Universidade Federal do Ceará}
    
    \LARGE{Centro de Tecnologia}
    
    \LARGE{Departamento de Engenharia de Teleinformática}
    
    \LARGE{Engenharia de Teleinformática}
    
    \vspace{180pt}
      
    \LARGE{Filtragem Adaptativa}
      
    \LARGE{Listas de Exercícios Propostos}
      
    \vspace{100pt}
    
\end{center}

\vspace{25pt}

\begin{flushleft}
	\begin{tabbing}
		Student \qquad Kenneth Brenner dos Anjos Benício – 519189\\
	   \qquad\qquad\qquad\= \\
		Professor\> Charles Casimiro Cavalcante e Guilherme Barreto\\
		Course \> Filtragem Adaptativa - TIP7188\\
	\end{tabbing}
\end{flushleft}

\vspace{25pt}

\begin{center}
    Fortaleza, 2021
\end{center}
	
	\section*{Estatísticas de Segunda Ordem}
	
		\begin{enumerate}
			
			\item Determine a média e a função de autocorrelação para o processo aleatório em que $v(n)$ é uma sequência de variáveis aleatórias independentes com média $\mu$ e variância $\sigma^2$. $x(n)$ é estacionário? Justifique.
			
				\begin{align}
					&x(n) = v(n) + 3 v(n-1),
				\end{align}
				
				\textcolor{red}{Solução:}
				
				Para o cálculo da média é possível escrever
				
				\begin{align}
					&E[x(n)] = E[v(n) + 3v(n-1)], 
				\end{align}
				
				mas uma vez que todas as variáveis aleatórias possuem mesma média o resultado será
				
				\begin{align}
					&E[x(n)] = \mu + 3\mu = 4\mu.  
				\end{align}
				
				De forma semelhante a variância pode ser obtida por 
				
				\begin{align}
					&\mathbb{E}\{[(x(n) - \mu_{X})]^2\}, \\
					&\mathbb{E}\{[x(n) - 4\mu]^2\} = \mathbb{E}\{[v(n) - 3v(n-1) -4\mu]^{2}\}, \\
					&\mathbb{E}\{[x(n) - 4\mu]^2\} = \mathbb{E}\{[v(n) - \mu + 3v(n-1) -3\mu]^{2}\},
				\end{align}
				
				e ao desenvolver o valor quadrado da expressão chega-se em
				
				\begin{align}
					&\mathbb{E}\{[x(n) - 4\mu]^2\} = \mathbb{E}\{ [v(n) - \mu]^{2} + 9[v(n-1) - \mu]^{2} + 6[v(n) - \mu][v(n-1) - \mu] \}, \\
					&\mathbb{E}\{[x(n) - 4\mu]^2\} = \sigma^{2} + 9\sigma^{2} + \mathbb{E}\{6[v(n) - \mu][v(n-1) - \mu]\}.
				\end{align}
				
				Sabendo que os sinais são descorrelacionados uma vez que temos duas variáveis aleatórias independentes temos
				
				\begin{align}
					&6\mathbb{E}\{[v(n) - \mu][v(n-1) - \mu]\} = \mathbb{E}\{v(n)v(n-1) - \mu v(n) -\mu v(n-1) + \mu^{2}\}, \\
					&6\mathbb{E}\{[v(n) - \mu][v(n-1) - \mu]\} = \mathbb{E}\{v(n)v(n-1)\} - \mu^{2} -\mu^{2} + \mu^{2}, \\
					&6\mathbb{E}\{[v(n) - \mu][v(n-1) - \mu]\} = \mathbb{E}\{v(n)\} \mathbb{E}\{v(n-1)\} - \mu^{2}, \\
					&6\mathbb{E}\{[v(n) - \mu][v(n-1) - \mu]\} = \mu^{2} - \mu^{2} = 0,
				\end{align}
				
				Portanto, a média pode ser dada por
				
				\begin{align}
					&\mathbb{E}\{[x(n) - 4\mu]^2\} = 10\sigma^{2}. 
				\end{align}
				
				Além disso, a função de correlação pode ser adquirida por meio da seguinte operação 
				
				\begin{align}
					&\mathbb{E}\{x(n)x^{*}(n)\} = \mathbb{E}\{[v(n) + 3v(n-1)][v(n) + 3v(n-1)]^{*}\}, \\
					&\mathbb{E}\{x(n)x^{*}(n)\} = \mathbb{E}\{[v(n)v^{*}(n)] + 3[v(n)v^{*}(n-1)] + 3[v(n-1)v^{*}(n)] + [9v(n-1)v^{*}(n-1)]\},
				\end{align}
				
				onde é possível reescrever a equação acima ao considerarmos novamente a descorrelação entre as variáveis aleatórias 
				
				\begin{align}
					&\mathbb{E}\{x(n)x^{*}(n)\} = \mathbb{E}\{\mu^{2} + 3\mu^{2} + 3\mu^{2} + 9\mu^{2}\}, \\
					&\mathbb{E}\{x(n)x^{*}(n)\} = 16\mu^{2}.
				\end{align}
				
				Uma vez que as estatísticas de primeira e de segunda ordem são independentes quanto ao deslocamento temporal é possível afirmar que o processo descrito é Estacionário no Sentido Amplo(WSS). Entretanto, não é possível afirmar que o processo é Estacionário no Sentido Estrito(SSS) uma vez que não se tem conhecimento da função que descreve a densidade de probabilidade desse sistema.
			
			\item Sejam os processos aleatórios $x(n)$ e $y(n)$ definidos por
			
				\begin{align} 
					&x(n) = v_1(n) + 3v_2(n-1), \\
					&y(n) = v_2(n + 1) + 3v_1(n-1),
				\end{align}
				
				em que $v_1(n)$ e $v_2(n)$ são processos de ruído branco independentes cada um com variância igual a $0,5$.
				
				\begin{enumerate}
					
					\item Quais são as funções de autocorrelação de $x$ e de $y$? Os processos são WSS?
					
						\textcolor{red}{Solução:}
						
						Antes de tudo é de interesse definir o processo estocástico de ruído branco. Um processo desse tipo é formado por um número qualquer de amostras
						independentes entre si e com média nula. Desse modo, uma vez que a média de processos de ruído branco são nulas é possível obter facilmente a média dos novos processos formados pela combinação dos ruídos
						
						\begin{align}
							&\mathbb{E}\{x(n)\} = \mathbb{E}\{v_{1}(n) + 3v_{2}(n-1)\} = \mu_{1} + 3\mu_{2} = 0, \\
							&\mathbb{E}\{y(n)\} = \mathbb{E}\{v_{2}(n+1) + 3v_{1}(n-1)\} = \mu_{2} + 3\mu_{1} = 0.
						\end{align}
						
						Para a variância teremos uma resposta similar
						
						\begin{align}
							&\mathbb{E}\{[x(n) - \mu]^{2}\} = \mathbb{E}\{[x(n) - 0]^{2}\} = \mathbb{E}\{[x(n)]^{2}\}, \\
							&\mathbb{E}\{[x(n) - \mu]^{2}\} = \mathbb{E}\{[v_{1}(n) + 3v_{2}(n-1)]^{2}\} = \mathbb{E}\{ [v^{2}_{1}(n)] + 6[v_{1}(n)v_{2}(n-1)] + 9[v^{2}_{2}(n-1)]\}, \\
							&\mathbb{E}\{[y(n) - \mu]^{2}\} = \mathbb{E}\{[y(n) - 0]^{2}\} = \mathbb{E}\{[y(n)]^{2}\}, \\
							&\mathbb{E}\{[y(n) - \mu]^{2}\} = \mathbb{E}\{[v_{2}(n+1) + 3v_{1}(n-1)]^{2}\} = \mathbb{E}\{ [v^{2}_{2}(n+1)] + 6[v_{2}(n+1)v_{1}(n-1)] + 9[v^{2}_{1}(n-1)]\}.
						\end{align}
						
						Rescrevendo a expressão para obter a formula da variância e utilizando o fato dos processos de ruído serem descorrelacionados uma nova expressão é obtida
						
						\begin{align}
							&\mathbb{E}\{[x(n) - \mu]^{2}\} = \mathbb{E}\{ [v^{2}_{1}(n) - 0] + 0 + 9[v^{2}_{2}(n-1) - 0]\}, \\
							&\mathbb{E}\{[x(n) - \mu]^{2}\} = \sigma^{2}_{1} + 9\sigma^{2}_{2} = 10*0.5 = 5, \\
							&\mathbb{E}\{[y(n) - \mu]^{2}\} = \mathbb{E}\{ [v^{2}_{2}(n+1) - 0] + 0 + 9[v^{2}_{1}(n-1) - 0]\}, \\
							&\mathbb{E}\{[y(n) - \mu]^{2}\} = \sigma^{2}_{2} + 9\sigma^{2}_{1} = 10*0.5 = 5.
						\end{align}
						
						A função de autocorrelação para $X$ é calculada como se segue 
						\begin{align}
							&r_{x}(\tau) = \mathbb{E}\{x(n)x(n + \tau)\} = \mathbb{E}\{[v_{1}(n) + 3v_{2}(n-1)][v_{1}(n + \tau) + 3v_{2}(n-1 + \tau)]\}, \\
							&r_{x}(\tau) = \mathbb{E}\{[v_{1}(n)v_{1}(n + \tau)] + 3[v_{1}(n)v_{2}(n-1 + \tau)] + 3[v_{2}(n-1)v_{1}(n + \tau)] + 9[v_{2}(n-1)v_{2}(n-1 + \tau)] \}.    
						\end{align}
						
						Sabendo que amostras de um mesmo processo são descorrelacionadas e que ambos os processos também são descorrelacionados, além de termos conhecimento 
						de que ambos possuem média nula, então 
						
						\begin{align}
							&\mathbb{E}\{[v_{1}(n)v_{1}(n + \tau)]\} = \mathbb{E}\{[v_{1}(n)]\} \mathbb{E}\{[v_{1}(n + \tau)]\} = 0, \\ 
							&3\mathbb{E}\{[v_{1}(n)v_{2}(n-1 + \tau)]\} = 3\mathbb{E}\{[v_{1}(n)]\} \mathbb{E}\{[v_{2}(n-1 + \tau)]\} = 0, \\
							&3\mathbb{E}\{[v_{2}(n-1)v_{1}(n + \tau)]\} = 3\mathbb{E}\{[v_{2}(n-1)]\} \mathbb{E}\{[v_{1}(n + \tau)]\} = 0, \\
							&9\mathbb{E}\{[v_{2}(n-1)v_{2}(n-1 + \tau)]\} = 9\mathbb{E}\{[v_{2}(n-1)]\} \mathbb{E}\{[v_{2}(n-1 + \tau)]\} = 0, \\ 
							&r_{x}(\tau) = 0.
						\end{align}
						
						Repetindo o procedimento para $Y$ e omitindo informações redundantes chega-se ao seguinte resultado
						
						\begin{align}
							&r_{y}(\tau) = 0.
						\end{align}
						
						Sendo as estatísticas de primeira e de segunda ordem independentes quanto ao tempo para os dois processos então é possível afirmar que eles são WSS.
					
					\item Qual é a função de correlação cruzada $r_{xy}(n_1,n_0)$? Estes processos são conjuntamente estacionários (no sentido amplo)? Justifique.
					
						\textcolor{red}{Solução:}
						
						\begin{align}
							&r_{x,y}(n_{1},n_{0}) = \mathbb{E}\{[x(n_{1})y^{*}(n_{0})]\} = \mathbb{E}\{[v_{1}(n_{1}) + 3v_{2}(n_{1}-1)][v_{2}(n_{0}+1) + 3v_{1}(n_{0}-1)]^{*}\}, \\
							&r_{x,y}(n_{1},n_{0}) = \mathbb{E}\{[v_{1}(n_{1})v^{*}_{2}(n_{0}+1)] + 3[v_{1}(n_{1})v^{*}_{1}(n_{0}-1)] + 3[v_{2}(n_{1}-1)v^{*}_{2}(n_{0}+1)] + 9[v_{2}(n_{1}-1)v^{*}_{1}(n_{0}-1)]\}.  
						\end{align}
						
						Novamente considerando que os processos de ruído branco são descorrelacionadas entre si e que as amostras individuais são também independentes então podemos simplificar a expressão acima da seguinte forma
						
						\begin{align}
							&r_{x,y}(n_{1},n_{0}) = 0 
						\end{align}
						
						Sendo assim, os processos podem ser considerados conjuntamente estacionários uma vez que a correlação cruzada é independente do instante temporal e que os processos que compoem o processo conjunto podem 
						ser considerados estacionários em sentido amplo isoladamente. Entretanto, vale notar que foi possível realizar tal afirmação apenas por termos conhecimento da relação de independência entre os processos 
						de ruído branco, pois de outro modo não seria possível realizar afirmações quanto a correlação cruzada do processo conjunto.
					
				\end{enumerate}
			
			\item Quais as condições que os elementos de uma matriz
			
				\begin{align}
					\mathbf{R} = \left[
					\begin{matrix}
						a & b  \\
						c & d  \\
					\end{matrix} \right]
				\end{align}
				
				devem satisfazer tal que $\mathbf{R}$ seja uma matriz de autocorrelação válida de
			
				\begin{enumerate}
					
					\item Um vetor aleatório bidimensional?
					
						\textcolor{red}{Solução:}
						
						Em suma, precisamos estar atentos às seguintes propriedades para garantir que temos em mãos uma matriz de autocorrelação válida
						
						
						- $\mathbf{R_{x}} = \mathbf{R^{H}_{x}}$
						
						- $\mathbf{a^{H}} \mathbf{R_{xa}} \geq 0$
						
						- $\mathbf{Ax} = \lambda \mathbf{x}, \forall \lambda \geq 0 \text{ and } \mathbf{x} \in \mathbb{R}$
						
						Desse modo, considerando um vetor aleatório bidimensional descrito por $\mathbf{X} = (x_{1},x_{2})$ podemos então
						
						\begin{align}
							\mathbf{R} &= \left[ 
							\begin{matrix}
								\mathbb{E}\{[x^{2}_{1}]\} & \mathbb{E}\{[x_{1}x^{*}_{2}]\} \\
								\mathbb{E}\{[x_{2}x^{*}_{1}]\} & \mathbb{E}\{[x^{2}_{2}]\} \\
							\end{matrix} \right], \\
							\mathbf{R}^{\text{H}} &= \left[ 
							\begin{matrix}
								\mathbb{E}\{[x^{2}_{1}]\} & \mathbb{E}\{[x_{2}x^{*}_{1}]\} \\
								\mathbb{E}\{[x_{1}x^{*}_{2}]\} & \mathbb{E}\{[x^{2}_{2}]\} \\
							\end{matrix} \right].
						\end{align}
						
						Inicialmente, podemos garantir a simetria quanto ao hermitiano se fizermos 
						
						\begin{align}
							&\mathbb{E}\{[x_{1}x^{*}_{2}]\} \overset{\Delta}{=} \mathbb{E}\{[x_{2}x^{*}_{1}]\},
							&\mathbb{E}\{[x_{2}x^{*}_{1}]\} \overset{\Delta}{=} \mathbb{E}\{[x_{1}x^{*}_{2}]\}.
						\end{align}
						
						Entretanto, sabemos que o operador esperança é linear tornando assim as expressões acima equivalentes. Em continuidade do problema, podemos 
						considerar que a restrição as quais os autovalores estão submetidos pode ser facilmente atingida ao garantirmos que o determinante da matriz 
						de correlação seja maior que a nulidade
						
						\begin{align}
							&\mathbb{E}\{[x^{2}_{1}]\} \mathbb{E}\{[x^{2}_{1}]\} - \mathbb{E}\{[x_{1}x_{2}]\} \mathbb{E}\{[x_{2}x_{1}]\} > 0, \\
							&\mathbb{E}\{[x_{1}^{2}]\} \mathbb{E}\{[x^{2}_{2}]\} > \mathbb{E}\{[x_{1}x_{2}]\} \mathbb{E}\{[x_{2}x_{1}]\}.
						\end{align}
				
					\item Um processo estocástico estacionário escalar?
					
						\textcolor{red}{Solução:}
						
						Considerando um processo estocástico estacionário escalar do tipo $\mathbf{X}_{t} = x(t)$ e uma versão atrasada desse
						processo definida por $\mathbf{X}_{t + \tau} = x(t + \tau)$ temos que a matriz de correlação pode ser escrita da seguinte forma
						
						\begin{align}
							\mathbf{R} &= \left[ 
							\begin{matrix}
								\mathbb{E}\{[x^{2}(t)_{1}]\} & \mathbb{E}\{[x(t)x^{*}(t + \tau)]\} \\
								\mathbb{E}\{[x(t + \tau)x^{*}(t)]\} & \mathbb{E}\{[x^{2}(t + \tau)]\} \\
							\end{matrix} \right], \\
							\mathbf{R}^{\text{H}} &= \left[ 
							\begin{matrix}
								\mathbb{E}\{[x^{2}(t)_{1}]\} & \mathbb{E}\{[x(t + \tau)x^{*}(t)]\} \\
								\mathbb{E}\{[x(t)x^{*}(t + \tau)]\} & \mathbb{E}\{[x^{2}(t + \tau)]\} \\
							\end{matrix} \right].
						\end{align}
						
						
						Em sequência, podemos garantir a simetria quanto ao hermitiano se fizermos 
						
						\begin{align}
							&\mathbb{E}\{[x(t)x^{*}(t + \tau)]\} \overset{\Delta}{=} \mathbb{E}\{[x(t + \tau)x^{*}(t)]\}, \\
							&\mathbb{E}\{[x(t + \tau)x^{*}(t)]\} \overset{\Delta}{=} \mathbb{E}\{[x(t)x^{*}(t + \tau)]\},
						\end{align}
						
						mas, mais uma vez, considerando que o operador esperança é linear então as duas expressões são equivalentes. Já considerando a
						restrição imposta aos autovalores da matriz temos novamente
						
						\begin{align}
							&\mathbb{E}\{[x^{2}(t)]\} \mathbb{E}\{[x^{2}(t + \tau)]\} > \mathbb{E}\{[x(t)x^{*}(t + \tau)]]\}  \mathbb{E}\{[x(t + \tau)]x^{*}(t)]\}.
						\end{align}
					
				\end{enumerate}
			
			\item Assuma que a inversa $\mathbf{R}_{\mathbf{x}}^{-1}$ da matriz de autocorrelação de um vetor coluna $N$-dimensional exista. Mostre que
			
				\begin{align}
					\mathbb{E}\left\{\mathbf{x}^H \mathbf{R}_{\mathbf{x}}^{-1} \mathbf{x} \right\} = N
				\end{align}
				
				\textcolor{red}{Solução:}
				
				Inicialmente podemos escrever a expressão regular para a matriz de autocorrelação e supondo que de fato existe uma inversa bem definida 
				para a matriz de autocorrelação
				
				\begin{align}
					\mathbb{E}\{\mathbf{x} \mathbf{x}^{\text{H}}\} = \mathbf{R}_{x}, \\
					\mathbb{E}\{\mathbf{x} \mathbf{x}^{\text{H}}\}\mathbf{R}^{-1}_{x} = \mathbf{R}_{x}\mathbf{R}^{-1}_{x}, \\
					\mathbb{E}\{\mathbf{x} \mathbf{x}^{\text{H}}\mathbf{R}^{-1}_{x}\} = \mathbf{I}_{N}, \\
				\end{align}
				
				onde a última passagem foi possível pois consideramos que, embora de naturaze aleatória, a matriz de autocorrelação possui medicões fixas 
				após o término da coleta das amostas. Desse modo, podemos aplicar o operador traço de matriz e por meio das propriedades relevantes é possível
				alterar de forma conveniente a ordem do produto dos termos, somente ao aplicarmos uma permutação cíclica, obtendo portanto o seguinte resultado
				
				\begin{align}
					\text{Tr}\{\mathbb{E}\{\mathbf{x} \mathbf{x}^{\text{H}}\mathbf{R}^{-1}_{x}\}\} = \text{Tr}\{\mathbf{I}_{N}\}, \\
					\text{Tr}\{\mathbb{E}\{\mathbf{x}^{\text{H}}\mathbf{R}^{-1}_{x} \mathbf{x}\}\} = \text{Tr}\{\mathbf{I}_{N}\}, \\
					\text{Tr}\{\mathbb{E}\{\mathbf{x}^{\text{H}}\mathbf{R}^{-1}_{x} \mathbf{x}\}\} = N,
				\end{align}
			
			\item Mostre que as matrizes de correlação e covariância satisfazem as relações abaixo:
				
				\begin{enumerate}
					
					\item $\mathbf{R}_\mathbf{x} = \mathbf{C}_\mathbf{x} + {\mathbf{\mu}}_{\mathbf{x}}{\mathbf{\mu}}_{\mathbf{x}}^H$
					
						\textcolor{red}{Solução:}
						
						\begin{align}
							&C_{X} = \mathbb{E}\{[(x - \mu)(x - \mu)^{H}]\} = \mathbb{E}\{[xx^{H} -x\mu^{H} - \mu x^{H} + \mu \mu^{H}]\}, \\
							&C_{X} = \mathbb{E}\{[xx^{H}]\} -\mathbb{E}\{[x\mu^{H}]\} - \mathbb{E}\{[\mu x^{H}]\} + \mathbb{E}\{[\mu \mu^{H}]\}.
						\end{align}
						
						
						Considerando a matriz de correlação pode ser escrita por $R_{X} = \mathbb{E}\{[xx^{H}]\}$ e que o valor médio de um escalar é o próprio escalar temos
						
						\begin{align}
							&C_{X} = R_{X} - \mathbb{E}\{[x\mu^{H}]\} - \mathbb{E}\{[\mu x^{H}]\} + \mu \mu^{H}, \\
							&C_{X} = R_{X} - \mu^{H}\mathbb{E}\{[x]\} - \mu \mathbb{E}\{[x^{H}]\} + \mu \mu^{H}, \\
							&C_{X} = R_{X} - \mu \mu^{H} - \mu \mu^{H} + \mu \mu^{H}, \\
							&C_{X} = R_{X} - \mu \mu^{H}, \\
							&R_{X} = C_{X} + \mu \mu^{H}.
						\end{align}
					
					\item $\mathbf{C}_{\mathbf{x} + \mathbf{y}} = \mathbf{C}_{\mathbf{x}} +
					\mathbf{C}_{\mathbf{y}}$, para $\mathbf{x}$ e $\mathbf{y}$ descorrelacionados
					
						\textcolor{red}{Solução:}
						
						\begin{align}
							&\mathbf{C}_{\mathbf{x} + \mathbf{y}} = \mathbf{C}_{\mathbf{x}} + \mathbf{C}_{\mathbf{xy}} + \mathbf{C}_{\mathbf{yx}} + \mathbf{C}_{\mathbf{y}}.
						\end{align}
						
						Onde os termos de correlação cruzada $\mathbf{C}_{\mathbf{xy}}$ e $\mathbf{C}_{\mathbf{yx}}$ podem ser obtidos  da seguinte maneira
						
						\begin{align}
							&\mathbf{C_{xy}} = \mathbb{E}\{[x - \mu_{x}][y - \mu_{y}]\}, \\
							&\mathbf{C_{xy}} = \mathbb{E}\{[xy]\} - \mathbb{E}\{[x\mu_{y}]\} - \mathbb{E}\{[\mu_{x} y]\} + \mathbb{E}\{[\mu_{x} \mu_{y}]\}, \\
							&\mathbf{C_{xy}} = \mathbb{E}\{[xy]\} - \mu_{y} \mu_{x} - \mu_{x} \mu_{y} + \mu_{x} \mu_{y}, \\
							&\mathbf{C_{xy}} = \mathbb{E}\{[xy]\} + \mu_{x} \mu_{y}, \\
							&\mathbf{C_{xy}} = \mu_{x} \mu_{y}, \\
						\end{align}
						
						\begin{align}
							&\mathbf{C_{yx}} = \mathbb{E}\{[y - \mu_{y}][x - \mu_{x}]\}, \\
							&\mathbf{C_{yx}} = \mathbb{E}\{[yx]\} - \mathbb{E}\{[y\mu_{x}]\} - \mathbb{E}\{[\mu_{y} x]\} + \mathbb{E}\{[\mu_{y} \mu_{x}]\}, \\
							&\mathbf{C_{yx}} = \mathbb{E}\{[yx]\} - \mu_{y} \mu_{x} - \mu_{x} \mu_{y} + \mu_{y} \mu_{x}, \\
							&\mathbf{C_{yx}} = \mathbb{E}\{[yx]\} - \mu_{x} \mu_{y}, \\
							&\mathbf{C_{yx}} = - \mu_{x} \mu_{y}.
						\end{align}
						
						Portanto, é possível afirmar que a soma dos termos de correlação cruzada é nula e que a correlação da soma de dois vetores descorrelacionados é de fato 
						dada por $\mathbf{C}_{\mathbf{x} + \mathbf{y}} = \mathbf{C}_{\mathbf{x}} + \mathbf{C}_{\mathbf{y}}$
						
						\begin{align}
							&\mathbf{C_{xy}} + \mathbf{C_{yx}}  = \mu_{x} \mu_{y} - \mu_{x} \mu_{y}, \\
							&\mathbf{C_{xy}} + \mathbf{C_{yx}}  = 0.
						\end{align}
					
					
				\end{enumerate}
			
			\item Processos aleatórios $v_1(n)$ e $v_2(n)$  são independentes e têm a mesma função de correlação
			
				\begin{align}
					r_v(n_1,n_0) = 0.5\delta(n_1 - n_0)
				\end{align}
			 
			 	\begin{enumerate}
			 		
			 		\item Qual é a função de correlação do processo aleatório
			 		
				 		\begin{align}
				 			x(n) = v_1(n) + 2v_1(n + 1) + 3v_2(n-1) ?
				 		\end{align}
				 		
				 		Este é um processo WSS? Justifique.
				 		
				 		\textcolor{red}{Solução:}
				 		
				 		Para simplificar o desenvolvimento iremos considerar que os processos possuem media nula e são descorrelacionados
				 		
				 		\begin{align*}
				 			&r_{x} = \mathbb{E}\{x(n) x^{*}(n)\}, \\
				 			&r_{x} = \mathbb{E}\{[v_{1}(n) + 2v_{1}(n+1) + 3v_{2}(n-1)] [v_{1}(n) + 2v_{1}(n+1) + 3v_{2}(n-1)]^{*}(n)\}, \\
				 			&r_{x} = \mathbb{E}\{[v_{1}(n)v^{*}_{1}(n) + 2v_{1}(n)v^{*}_{1}(n+1) + 3v_{1}(n)v^{*}_{2}(n-1) + 2v_{1}(n + 1)v^{*}_{1}(n) + 4v_{1}(n+1)v^{*}_{1}(n+1) +
				 			6v_{1}(n+1)v^{*}_{2}(n-1)] + 3v_{2}(n-1)v^{*}_{1}(n) +  6v_{2}(n-1)v^{*}_{1}(n+1) + 9v_{2}(n-1)v^{*}_{2}(n-1)\}, \\
				 			&r_{x} = r_{v}(n,n) + 2r_{v}(n,n+1) + 0 + 2r_{v}(n+1,n) + 4r_{v}(n+1,n+1) + 0 + 0 + 0 + 9r_{v}(n-1,n-1), \\
				 			&r_{x} = r_{v}(n,n) + 2r_{v}(n,n+1) + 2r_{v}(n+1,n) + 4r_{v}(n+1,n+1) + 9r_{v}(n-1,n-1).
				 		\end{align*}
				 		
				 		Após uma breve análise na expressão $r_{v}(n_{1},n_{0})$ é possível verificar que determinados termos anulam-se ao considerar um único momento temporal devido a presença da função degrau unitário, permitindo a seguinte simplificação:
				 		
				 		\begin{align}
				 			&r_{x} = 2r_{v}(n,n+1) + 2r_{v}(n+1,n), \\
				 			&r_{x} = \delta(n - n - 1) + \delta(n + 1 - n).
				 		\end{align}
				 		
				 		Onde a generalização pode ser descrita por:
				 		
				 		\begin{align}
				 			&r_{x}(n_{1}, n_{2})= \delta(n_{1} - n_{2}) + \delta(n_{2} - n_{1}).  
				 		\end{align}
				 		
				 		Uma vez que a correlação é dependente apenas de um deslocamento temporal, então podemos classificar esse processo como WSS.
			 		
			 		\item Encontre a a matrix de correlação de um vetor aleatório consistindo de oito amostras consecutivas de
			 		$x(n)$.
			 		
				 		\textcolor{red}{Solução:}
				 		
				 		\begin{align}
				 			\mathbf{R}_{\mathbf{x}} = \left[
				 			\begin{matrix}
				 				2 & 0 & 0 & 0 & 0 & 0 & 0 & 0\\
				 				0 & 2 & 0 & 0 & 0 & 0 & 0 & 0\\
				 				0 & 0 & 2 & 0 & 0 & 0 & 0 & 0\\
				 				0 & 0 & 0 & 2 & 0 & 0 & 0 & 0\\
				 				0 & 0 & 0 & 0 & 2 & 0 & 0 & 0\\
				 				0 & 0 & 0 & 0 & 0 & 2 & 0 & 0\\
				 				0 & 0 & 0 & 0 & 0 & 0 & 2 & 0\\
				 				0 & 0 & 0 & 0 & 0 & 0 & 0 & 2
				 			\end{matrix}
				 			\right]
				 		\end{align}
				 		
				 		Para chegar a esse resultado foi utilizado a expressão para a correlação obtida no item anterior. Uma vez que a função delta de dirac terá um valor não nulo apenas quando o argumento for zero, isso irá acontecer apenas quando os momentos $n_{1}$ e $n_{2}$ forem iguais e isso irá acontecer apenas com os elementos da diagonal principal.
			 		
			 	\end{enumerate}
		 	
		\end{enumerate}
		
	\newpage
	\section*{Filtragem Linear Ótima}
	
		\begin{enumerate}
			
			\item Considere um problema de filtragem de Wiener conforme caracterizado a seguir. A matriz de correlação $\mathbf{R}_{\mathbf{x}}$ de um vetor de entrada $\mathbf{x}(n)$ é dada por
			
				\begin{align*}
					\mathbf{R}_{\mathbf{x}} = \left[ \begin{matrix}
						1 & {0.5}  \\   {0.5} & 1  \\ \end{matrix} \right].
				\end{align*}
				
				O vetor de correlação cruzada $\mathbf{p}_{\mathbf{x}d}$ entre o vetor de entrada $\mathbf{x}$ e a resposta desejada
				$d(n)$ é
				
				\begin{align*}
					\mathbf{p}_{\mathbf{x}d} &= \left[ \begin{matrix}   {0.5}  \\   {0.25}  \\ \end{matrix} \right]
				\end{align*}
				
				\begin{enumerate}
					
					\item Encontre o vetor de coeficientes do filtro de Wiener.
					
						\textcolor{red}{Solução:}
						
						Isso pode ser realizado de forma simples pelo uso da equação do filtro ótimo de wiener
						
						\begin{align}
							\mathbf{w}_{\text{opt}} &= \mathbf{R}^{-1}_{X} \mathbf{p}_{xd}, \\
							\mathbf{w}_{\text{opt}} &=  \left[ \begin{matrix} 1 & 0.5 \\ 0.5 & 1 \end{matrix} \right]  \left[ \begin{matrix} 0.5 \\ 0.25 \end{matrix} \right], \\
							\mathbf{w}_{\text{opt}} &=  \frac{4}{3} \left[ \begin{matrix} 1 & -0.5 \\ -0.5 & 1 \end{matrix} \right]  \left[ \begin{matrix} 0.5 \\ 0.25 \end{matrix} \right], \\
							\mathbf{w}_{\text{opt}} &= \left[ \begin{matrix} 0.5 \\ 0.0 \end{matrix} \right].
						\end{align}
					
					\item Qual é o mínimo erro médio quadrático fornecido por este filtro?
					
						\textcolor{red}{Solução:}
						
						\begin{align}
							\mathbb{E}\{e^{2}(n)\}  &=\sigma^{2}_{d} - 2\mathbf{w}^{T}\mathbf{p}_{xd} + w^{T}\mathbf{R}_{X}\mathbf{w}      
						\end{align}
						
						Ao aplicar o vetor descoberto no item anterior obtem-se o erro mínimo
						
						\begin{align}
							e_{min}  &=\sigma^{2}_{d} - 2\mathbf{w}^{T}_{opt}\mathbf{p}_{xd} + w^{T}_{opt}\mathbf{R}_{X}\mathbf{w}_{\text{opt}}, \\
							e_{min} &= \sigma^{2}_{d} - 2 \left[ \begin{matrix} 0.5 & 0.0 \end{matrix} \right] \left[ \begin{matrix} 0.5 \\ 0.25 \end{matrix} \right] + \left[ \begin{matrix} 0.5 & 0.0 \end{matrix} \right] \left[ \begin{matrix} 1 & -0.5 \\ -0.5 & 1 \end{matrix} \right]  \left[ \begin{matrix} 0.5  \\ 0.0 \end{matrix} \right], \\
							e_{min} &= \sigma^{2}_{d} - 2 * 0.25 + 0.25, \\
							e_{min} &= \sigma^{2}_{d} - 0.25. 
						\end{align}
						
						Dessa forma o erro é dependente da variância do sinal desejado.					
					
					\item Formule uma representação do filtro de Wiener em termos dos autovalores da matriz $\mathbf{R}_{\mathbf{x}}$
					e de seus autovetores associados.
					
						\textcolor{red}{Solução:}
						
						Utilizando a decomposição matricial em autovalores (EVD) é possível reescrever a matrix de correlação como abaixo
						
						\begin{align}
							\mathbf{R}_{X} &= \mathbf{Q} \mathbf{\Lambda} \mathbf{Q}^{-1}.
						\end{align}
						
						A matriz $\mathbf{\Lambda}$ contémm os autovalores $\lambda_{i}$ e a matriz $\mathbf{Q}$ os respectivos autovetores. Em posse dessa relação é possível reescrever a equação do fitro ótimo de wiener como
						
						\begin{align}
							\mathbf{w}_{\text{opt}} &= \mathbf{R}^{-1}_{X} \mathbf{p}_{xd}, \\
							\mathbf{w}_{\text{opt}} &= (\mathbf{Q} \mathbf{\Lambda} \mathbf{Q}^{-1})^{-1} \mathbf{p}_{xd}, \\
							\mathbf{w}_{\text{opt}} &= \mathbf{Q}^{-1} \mathbf{\Lambda}^{-1} \mathbf{Q} \mathbf{p}_{xd}.
						\end{align}
						
						É possível verificar de imediato que essa propriedade é bem útil uma vez que a inversa de uma matriz diagonal é certamente menos custosa que a inversa da matriz de autocorrelação completa.
					
				\end{enumerate}
			
			\item Mostre que a equação do erro mínimo pode se escrita da seguinte maneira
			
				\begin{align*}
					\mathbf{A}\left[ \begin{matrix}
						1  \\
						{ - \mathbf{w}}  \\
					\end{matrix} \right] = \left[ \begin{matrix}
						J_{\text{min}}  \\
						\mathbf{0}  \\
					\end{matrix} \right],
				\end{align*}
				
				em que $J_{\text{min}}$ é o mínimo erro médio quadrático, $\mathbf{w}$ é o filtro de Wiener, e $\mathbf{A}$ é a matriz de
				correlação do vetor aumentado
				
				\begin{align*}
					\left[ \begin{matrix}
						d(n)  \\
						\mathbf{x}(n)  \\
					\end{matrix} \right],
				\end{align*}
				
				em que $d(n)$ é o sinal desejado e $\mathbf{x(n)}$ é o sinal de entrada do filtro de Wiener.
				
				\textcolor{red}{Solução:}
				
				Inicialmente deve-se calcular a matriz de correlação do vetor aumentado
				
				\begin{align}
					\mathbf{A} &= \mathbb{E} \{ \left[ \begin{matrix} d(n) \\ x(n) \end{matrix} \right] \left[ \begin{matrix} d(n)^{T} & x(n)^{T} \end{matrix} \right] \}, \\
					\mathbf{A} &=   \left[ \begin{matrix} \mathbb{E}\{d(n) d(n)^{T}\} & \mathbb{E}\{d(n) x(n)^{T}\} \\ \mathbb{E}\{x(n) d(n)^{T}\} & \mathbb{E}\{x(n) x(n)^{T}\} \end{matrix} \right].
				\end{align}
				
				Dando as devidas nomeações aos termos a expressão acima reduz-se a
				
				\begin{align}
					\mathbf{A} &=  \left[ \begin{matrix} \sigma^{2}_{d} & \mathbf{p}_{xd}^{T} \\
						\mathbf{p}_{xd} & \mathbf{R}_{X} \end{matrix} \right].
				\end{align}
				
				Multiplicando-se a expressão acima pelo vetor $[1 - \mathbf{w}]^{T}$
				
				\begin{align}
					\mathbf{A} \left[ \begin{matrix} 1 \\ -\mathbf{w} \end{matrix} \right] &=   \left[ \begin{matrix} \sigma^{2}_{d} & \mathbf{p}_{xd}^{T} \\
						\mathbf{p}_{xd} & \mathbf{R}_{X} \end{matrix} \right] \left[ \begin{matrix} 1 \\ -\mathbf{w} \end{matrix} \right], \\
					\mathbf{A} \left[ \begin{matrix} 1 \\ -\mathbf{w} \end{matrix} \right] &=   \left[ \begin{matrix} \sigma^{2}_{d} - \mathbf{p}_{xd}^{T}\mathbf{w} \\
						\mathbf{p}_{xd} - \mathbf{R}_{X}\mathbf{w} \end{matrix} \right].
				\end{align}
				
				Por fim, resta apenas aplica a equação ótima do filtro de wiener $\mathbf{w}_{\text{opt}} = \mathbf{R}^{-1}_{X} \mathbf{p}_{xd}$
				
				\begin{align}
					\mathbf{A} \left[ \begin{matrix} 1 \\ -\mathbf{w} \end{matrix} \right] &=   \left[ \begin{matrix} \sigma^{2}_{d} - \mathbf{p}_{xd}^{T}\mathbf{R}^{-1}_{X} \mathbf{p}_{xd} \\
						\mathbf{p}_{xd} - \mathbf{R}_{X}\mathbf{R}^{-1}_{X} \mathbf{p}_{xd} \end{matrix} \right], \\
					\mathbf{A} \left[ \begin{matrix} 1 \\ -\mathbf{w} \end{matrix} \right] &=   \left[ \begin{matrix} \sigma^{2}_{d} - \mathbf{p}_{xd}^{T}\mathbf{R}^{-1}_{X} \mathbf{p}_{xd} \\
						\mathbf{p}_{xd} - \mathbf{I}_{X}\mathbf{p}_{xd} \end{matrix} \right], \\
					\mathbf{A} \left[ \begin{matrix} 1 \\ -\mathbf{w} \end{matrix} \right] &=   \left[ \begin{matrix} \sigma^{2}_{d} - \mathbf{p}_{xd}^{T
						}\mathbf{R}^{-1}_{X} \mathbf{p}_{xd} \\ 0 \end{matrix} \right], \\
					\mathbf{A} \left[ \begin{matrix} 1 \\ -\mathbf{w} \end{matrix} \right] &=   \left[ \begin{matrix} J_{min} \\ 0 \end{matrix} \right].
				\end{align}
			
			\item Em várias aplicações práticas há uma necessidade de cancelar ruído que foi adicionado a um sinal. Por exemplo, se estamos usando o telefone celular dentro de um ruído e o ruído do carro ou rádio é adicionado à mensagem que estamos tentando transmitir. A Figura abaixo ilustra as situações de contaminação de ruído. Calcule o filtro de Wiener (filtro ótimo) de tal configuração em relação às estatísticas dos sinais envolvidos que você dispõe (conhece).
			
				\begin{figure}[!ht]
					\centering
					\includegraphics[width=0.5\textwidth]{figs/cancelamento_ruido.png}
					\caption{Estrutura de Equalização de Canal}
				\end{figure}
				
				\textcolor{red}{Solução:}
				
				Inicialmente é necessário calcular a equação de erro do sistema aqui proposto
				
				\begin{align}
					e(n) &= x(n) - \hat{v_{1}} = x(n) - \mathbf{w}^{T}v_{2}(n)
				\end{align}
				
				Em seguida faz-necessário calcular a função mean square error(MSE) que é facilmente fornecida pela manipulação algébrica abaixo
				
				\begin{align}
					e^{2}(n) &= [x(n) - \mathbf{w}^{T}v_{2}(n)][x(n) - \mathbf{w}^{T}v_{2}(n)]^{T}, \\
					e^{2}(n) &= x^{2}(n) - 2x(n)\mathbf{w}^{T}v_{2}(n) + \mathbf{w}^{T}v_{2}(n)v_{2}^{T}\mathbf{w}.
				\end{align}
				
				Sendo considerado que o filtro apresenta coeficientes constantes é possível aplicar o operador Valor Esperado de forma a obter a seguinte relação
				
				\begin{align}
					\mathbb{E}\{e^{2}(n)\} = \mathbb{E}\{x^{2}(n)\} - 2\mathbf{w}^{T}\mathbb{E}\{x(n) v_{2}(n)\} + \mathbf{w}^{T}\mathbb{E}\{v_{2}(n)v_{2}(n)^{T}\} \mathbf{w},& \\
					\mathbb{E}\{e^{2}(n)\} = \sigma^{2}_{x} - 2\mathbf{w}^{T}\mathbf{p}_{xv_{2}} + \mathbf{w}^{T}\mathbf{R}_{v_{2}} \mathbf{w}.&
				\end{align}
				
				Por fim, basta encontrar o $\mathbf{w}$ que minimiza o MSE acima. Para chegar a esse fim, calcula-se o gradiante quanto ao $\mathbf{w}$ igualando-se o resultado da operação a zero
				
				\begin{align}
					\nabla_{\mathbf{w}} \mathbb{E}\{e^{2}(n)\} = - 2\mathbf{p}_{xv_{2}} + 2\mathbf{R}_{v_{2}} \mathbf{w} = 0,& \\
					-\mathbf{p}_{xv_{2}} + \mathbf{R}_{v_{2}} \mathbf{w} = 0,& \\
					\mathbf{R}_{v_{2}} \mathbf{w} = \mathbf{p}_{xv_{2}}.&
				\end{align}
				
				Utilizando a identidade matricial abaixo é possível resolver a equação acima para obter o seguinte resultado
				
				\begin{align}
					\mathbf{R}^{-1}_{v_{2}}\mathbf{R}_{v_{2}} \mathbf{w} &= \mathbf{R}^{-1}_{v_{2}}\mathbf{p}_{xv_{2}}, \\
					\mathbf{I}\mathbf{w} &= \mathbf{R}^{-1}_{v_{2}}\mathbf{p}_{xv_{2}}, \\ 
					\mathbf{w} &= \mathbf{R}^{-1}_{v_{2}}\mathbf{p}_{xv_{2}}. 
				\end{align}
				
				Onde é possível reescrever o termo final como
				
				\begin{align}
					\mathbf{w} = \mathbf{R}^{-1}_{v_{2}}(\mathbf{p}_{d} + \mathbf{p}_{v_{1}} + \mathbf{p}_{v_{2}}) 
				\end{align}
			
			\item Seja um processo estocástico dado pela expressão abaixo onde $S(n)$ é um processo estocástico WSS dado e $a$ é uma constante.
			
				\begin{align*}
					x(n) = s(n + a) + s(n-4a),
				\end{align*}
				
				Deseja-se filtrar o processo de tal forma obter-se um processo $D(s) = s(n -a)$, o qual também sabe-se que é um processo WSS. Suponha que o sinal $d(n)$ possua média nula e variância unitária.
			
				\begin{enumerate}
					
					\item Calcule o filtro, com dois coeficientes, que fornece a solução ótima em relação ao erro médio quadrático.
					
						\textcolor{red}{Solução:}
					
					\item Calcule o preditor direto ótimo de passo unitário, com dois coeficientes, que fornece a solução ótima em relação ao erro médio quadrático.
					
						\textcolor{red}{Solução:}
					
					\item Compare as soluções dos dois.
					
						\textcolor{red}{Solução:}
					
				\end{enumerate}
			
			\item Suponha que foram encontrados os seguintes coeficientes de autocorrelação: $r_x(0) = 1$ e $r_x(1) = 0$. Tais coeficientes foram obtidos de amostras corrompidas com ruído. Além disso, a variância do sinal desejado é $\sigma_d^2 =
			24.40$ e o vetor de correlação cruzada é $\mathbf{p}_{\mathbf{x}d} = [2 \ \ 4.5]^T$. Encontre:
			
				\begin{enumerate}
					
					\item O valor dos coeficientes do filtro de Wiener.
					
						\textcolor{red}{Solução:}
						
						A partir dos coeficientes fornecidos é possível escrever a matrix de correlação necessário para o filtro ótimo de wiener como uma matriz identidade de ordem 2
						
						\begin{align}
							\mathbf{R}_{X} = \left[ \begin{matrix} 1 & 0 \\ 0 & 1 \end{matrix} \right]
						\end{align}
						
						Ao utilizar a solução fechada do problema chega-se ao seguinte vetor resultado
						
						\begin{align}
							\mathbf{w}_{\text{opt}} &= \mathbf{R}^{-1}_{x} \mathbf{p}_{xd}, \\
							\mathbf{w}_{\text{opt}} &=  \left[ \begin{matrix} 1 & 0 \\ 0 & 1 \end{matrix} \right]  \left[ \begin{matrix} 2 \\ 4.5 \end{matrix} \right], \\
							\mathbf{w}_{\text{opt}} &= \left[ \begin{matrix} 2 \\ 4.5 \end{matrix} \right].
						\end{align}
					
					\item A superfície definida por $J(\mathbf{w})$. Faça um gráfico da mesma.
					
						\textcolor{red}{Solução:}
						
						Para obter a expressão que define a superfície basta desenvolver a expressão para o erro médio
						
						\begin{align}
							\mathbf{J}(w) &= \mathbb{E}\{e^{2}(n)\}, \\
							\mathbf{J}(w) &= \sigma^{2}_{d} - 2\mathbf{w}^{T}\mathbf{p}_{xd} + w^{T}\mathbf{R}_{X}\mathbf{w}. \label{eq:mse}   
						\end{align}
						
						Substituindo os valores encontrados anteriormente na expressão da superfície
						
						\begin{align}
							\mathbf{J}(w_{0}, w_{1}) &= 24.40 - 2 \left[ \begin{matrix} w_{0}  w_{1} \end{matrix} \right] \left[ \begin{matrix} 2 \\ 4.5 \end{matrix} \right] + \left[ \begin{matrix} w_{0}  w_{1} \end{matrix} \right] \left[ \begin{matrix} 1 & 0 \\ 0 & 1 \end{matrix} \right]  \left[ \begin{matrix} w_{0}  \\ w_{1} \end{matrix} \right], \\
							\mathbf{J}(w_{0},w_{1}) &= 24.40 - 4w_{0} - 9w_{1} + w^{2}_{0} + w^{2}_{1}.
						\end{align}
						
						Utilizando um software gráfico é possível obter a Figura \ref{fig:01} onde é traçada a superfície de erro MSE expressa na Equação (\ref{eq:mse}).
					
						\begin{figure}[!ht]
							\centering
							\includegraphics[width=0.5\textwidth]{figs/superficie-de-erro.png}
							\caption{Superfície de erro MSE}
							\label{fig:01}
						\end{figure}
					
				\end{enumerate}
			
		\end{enumerate}
	
	\newpage
	\section*{Algoritmos Recursivos}
	
		\begin{enumerate}
			
			\item Deseja-se minimizar a função objetivo $\mathbb{E}\{e^{4}(n)\}$ utilizando-se um algoritmo do gradiente estocástico do tipo LMS. O algoritmo resultando é chamado de algoritmo least mean fourth (LMF). Derive tal algoritmo. Derive também o filtro ótimo para tal critério e compare as soluções.
			
				\textcolor{red}{Solução:}
				
				Podemos inicialmente definir a função erro para esse filtro como
				
				\begin{align}
					\mathbf{e}(n) &= d(n) - y(n), \\
					\mathbf{e}(n) &= d(n) - \mathbf{w}^{\text{T}}(n)\mathbf{x}(n),
				\end{align}
				
				e em sequência define-se a seguinte função objetivo que podemos simplifcar utilizando expansão polinomial de newton em conjunto com propriedades do operador transposto
				
				\begin{align*}
					\notag \mathbb{E}\{e^{4}(n)\} &= \mathbb{E}\{\left[d(n) - \mathbf{w}^{\text{T}}(n)\mathbf{x}(n)\right]^{4}\}, \\
					\notag \mathbb{E}\{e^{4}(n)\} &= \mathbb{E}\{d^{4}(n)\} - 4\mathbb{E}\{d^{3}(n)\mathbf{w}^{\text{T}}(n)\mathbf{x}(n)\} + 6\mathbb{E}\{d^{2}(n)\left[\mathbf{w}^{\text{T}}(n)\mathbf{x}(n)\right]^{2}\} \\
					&-4\mathbb{E}\{d(n)\left[\mathbf{w}^{\text{T}}(n)\mathbf{x}(n)\right]^{3}\} + \mathbb{E}\{\left[\mathbf{w}^{\text{T}}(n)\mathbf{x}(n)\right]^{4}\}, \\
					\notag \mathbb{E}\{e^{4}(n)\} &= \mathbb{E}\{d^{4}(n)\} - 4\mathbb{E}\{d^{3}(n)\mathbf{w}^{\text{T}}(n)\mathbf{x}(n)\} + 6\mathbb{E}\{d^{2}(n)\left[\mathbf{w}^{\text{T}}(n)\mathbf{x}(n) \mathbf{x}^{\text{T}}(n)\mathbf{w}(n)\right]\} \\&
					-4 \mathbb{E}\{\mathbf{w}^{\text{T}}(n) d(n) \mathbf{x}(n) \mathbf{x}^{\text{T}}(n) \mathbf{x}(n) \mathbf{w}(n) \mathbf{w}^{\text{T}}(n) \} + \mathbb{E}\{\mathbf{w}^{\text{T}}(n)\mathbf{x}(n) \mathbf{x}^{\text{T}}(n) \mathbf{w}(n) \mathbf{w}^{\text{T}}(n)\mathbf{x}(n) \mathbf{x}^{\text{T}}(n)\mathbf{w}(n)\}, \\
					\notag \mathbb{E}\{e^{4}(n)\} &= \mathbb{E}\{d^{4}(n)\} - 4 \mathbf{w}^{\text{T}}(n)\mathbb{E}\{d^{3}(n)\mathbf{x}(n)\} + 6 \mathbf{w}^{\text{T}}(n) \mathbb{E}\{d^{2}(n)\mathbf{x}(n) \mathbf{x}^{\text{T}}(n)\} \mathbf{w}(n) \\
					&- 4\mathbf{w}^{\text{T}}(n) \mathbb{E}\{d(n) \mathbf{x}(n) \mathbf{x}^{\text{T}}(n) \mathbf{x}(n)\} \mathbf{w}(n) \mathbf{w}^{\text{T}}(n) + \mathbf{w}^{\text{T}}(n) \mathbf{w}(n) \mathbb{E}\{\mathbf{x}(n) \mathbf{x}^{\text{T}}(n) \mathbf{x}(n) \mathbf{x}^{\text{T}}(n)\} \mathbf{w}^{\text{T}}(n) \mathbf{w}(n), \\
				\end{align*}
			
			\item Considere o uso de um a sequência de ruído branco com média nula e variância $\sigma^{2}$ como entrada do algoritmo LMS. Avalie
				
				\begin{enumerate}
					
					\item a condição para convergência do algoritmo em média.
					
						\textcolor{red}{Solução:}
						
						A condição de convergência está diretamente associada com o erro nos coeficientes do filtro adaptativo para cada iteração.
						Desse modo, podemos iniciar o estudo desse tópico com a seguinte expressão
						
						\begin{align}
							\Delta \mathbf{w} = \mathbf{w}(k) - \mathbf{w}_{\text{opt}},
						\end{align}
					
					\item o erro em excesso em média quadrática.
					
						\textcolor{red}{Solução:}
					 
				\end{enumerate}

			\item Avalie a questão anterior para o caso do algoritmo LMS-Normalizado. Compare os dois casos.
				
			\textcolor{red}{Solução:}
						
			\item Considere um sinal branco gaussiano de variância unitária transmitido por um canal de comunicação de função de transferência $H(z) = 1 + 1.6z^{-1}$. Para compensar este
			canal utiliza-se um equalizador dado por $W(z) = w_{0} + w_{1}z^{-1}$ .
			
			
				\begin{enumerate}
					
					\item Forneça o equalizador ótimo segundo o critério de Wiener. Esboce a posição dos zeros do canal e do equalizador no plano Z.
					
						\textcolor{red}{Solução:}
						
						Considerando um sinal gaussiano branco $x(n)$ a saída do canal pode ser prontamente obtida por
						
						\begin{align}
							y(n) = x(n) + 1.6 x(n - 1),
						\end{align}
						
						e a matriz de autocorrelação será então dada por
						
						\begin{align}
							\mathbf{R}_{y} =
							\begin{bmatrix}
								\mathbb{E}\{y(n)y^{\text{H}}(n)\} & \mathbb{E}\{y(n)y^{\text{H}}(n - 1)\} \\
								\mathbb{E}\{y(n - 1)y^{\text{H}}(n)\} & \mathbb{E}\{y(n - 1)y^{\text{H}}(n - 1)\}
							\end{bmatrix},
						\end{align}
						
						onde podemos calcular os valores teóricos para as correlações da seguinte forma se assumirmos que existe independência entre amostras distintas e que o sinal é média nula 
						
						\begin{align}
							\mathbb{E}\{y(n)y^{\text{H}}(n)\} &= \mathbb{E}\{ \mathbf{x}^{2}(n) + 1.6 \mathbf{x}(n) \mathbf{x}^{\text{H}}(n - 1) + 1.6 \mathbf{x}(n - 1) \mathbf{x}^{\text{H}} (n) + 2.56 \mathbf{x}^{2}(n - 1) \} = 3.56, \\
							\mathbb{E}\{y(n)y^{\text{H}}(n - 1)\} &= \mathbb{E}\{ \mathbf{x}(n) \mathbf{x}^{\text{H}}(n - 1) + 1.6 \mathbf{x}(n) \mathbf{x}^{\text{H}}(n - 2) + 1.6 \mathbf{x}(n - 1) \mathbf{x}^{\text{H}} (n - 1) + 2.56 \mathbf{x}(n - 1) \mathbf{x}^{\text{H}}(n - 2) \} = 1.60, \\
							\mathbb{E}\{y(n - 1)y^{\text{H}}(n)\} &= \mathbb{E}\{ \mathbf{x}(n - 1) \mathbf{x}^{\text{H}}(n) + 1.6 \mathbf{x}(n - 1) \mathbf{x}^{\text{H}}(n - 1) + 1.6 \mathbf{x}(n - 2) \mathbf{x}^{\text{H}} (n) + 2.56 \mathbf{x}(n - 2) \mathbf{x}^{\text{H}}(n - 1) \} = 1.60, \\
							\mathbb{E}\{y(n - 1)y^{\text{H}}(n - 1)\} &= \mathbb{E}\{ \mathbf{x}^{2}(n - 1) + 1.6 \mathbf{x}(n - 1) \mathbf{x}^{\text{H}}(n - 2) + 1.6 \mathbf{x}(n - 2) \mathbf{x}^{\text{H}} (n - 1) + 2.56 \mathbf{x}(n - 2)^{2} \} = 3.56, \\
						\end{align}
						
						podendo assim descrever a matriz de autocorrelação teórica e sua inversa como
						
						\begin{align}
							\mathbf{R}_{y} =
							\begin{bmatrix}
								3.56 & 1.60 \\
								1.60 & 3.56
							\end{bmatrix},
						\end{align}
						
						\begin{align}
							\mathbf{R}^{-1}_{y} = \frac{1}{3.56^{2} - 1.6^{2}}
							\begin{bmatrix}
								3.56 &  -1.60 \\
								-1.60 & 3.56
							\end{bmatrix} =
							\begin{bmatrix}
								0.35 &  -0.16 \\
								-0.16 & 0.35
							\end{bmatrix}.
						\end{align}
						
						Já o vetor de correlação cruzada teórico pode ser descrito por
						
						\begin{align}
							\mathbf{p}_{yd} =
							\begin{bmatrix}
								\mathbb{E}\{y(n)d(n)\} \\
								\mathbb{E}\{y(n - 1)d(n)\}
							\end{bmatrix} = 
							\begin{bmatrix}
								1 \\
								0
							\end{bmatrix},
						\end{align}
						
						pois queremos que o sinal de saída tenha a maior correlação possivel com o sinal desejado de um mesmo instante mas continue sendo independente de um sinal de um instante temporal diferente.
						Desse modo, podemos obter o equalizador ótimo segundo o critério de Wiener como
						
						\begin{align}
							\mathbf{w}_{\text{opt}} = \mathbf{R}^{-1}_{y} \mathbf{p}_{yd} = \begin{bmatrix}
								0.35 \\
								-0.16
							\end{bmatrix}.
						\end{align}
						
						Por fim, abaixo segue o traçado para os zeros das funções de transferência tanto do canal quanto do filtro ótimo em azul e em vermelho, respectivamente.
						
						\begin{figure}[!ht]
							\centering
							\includegraphics[width=0.5\textwidth]{figs/plano_z.png}
							\caption{Zeros do canal e do equalizador no plano-z.}
						\end{figure}
					
					\item Obtenha o filtro de erro de predição direta de passo unitário, correspondente ao sinal à saída do canal. Calcule os zeros deste filtro e compare com os do equalizador.
					
						\textcolor{red}{Solução:}
					
					\item Obtenha as trajetórias sobre as curvas de nível, tendo condições iniciais nulas para os coeficientes do equalizador, para os seguintes algoritmos: (a) Algoritmo de Newton, (b) Gradiente Determinístico, (c) Least Means Square e (d) Least Means Square Normalizado
						
						\textcolor{red}{Solução:}
						
						Antes de tudo é necessário definir a superfície de erro que servirá como referência para traçar as curvas de nível. Desse modo, podemos prontamente
						definir a superfície MSE como
						
						\begin{align}
							\mathbf{J}(w) &= \mathbb{E}\{e^{2}(n)\}, \\
							\mathbf{J}(w) &= \sigma^{2}_{d} - 2\mathbf{w}^{T}\mathbf{p}_{xd} + w^{T}\mathbf{R}_{X}\mathbf{w}.   
						\end{align}
						
						Desse modo, temos as seguintes curvas de nível
						
					\item Obtenha também a evolução do erro quadrático médio para cada um dos algoritmos anteriores.
					
						\textcolor{red}{Solução:}

					\item Qual o número de condicionamento para o problema em questão?
					
						\textcolor{red}{Solução:}
						VERIFICAR OS AUTOVALORES DA MATRIZ DE CORRELACAO
					
					\item Qual deveria ser o canal para que o número de condicionamento fosse menor/maior que 5?
					Comente os resultados.
					
						\textcolor{red}{Solução:}
						LEMBRAR QUE O DETERMINANTE É O PRODUTO ENTRE OS AUTOVALORES DE UMA MATRIZ E O NUMERO DE CONDICIONAMENTO ESTA ASSOCIADO COM A RAZAO MAXIMO/MINIMO DOS AUTOVALORES.
						
				\end{enumerate}
			
			\item Utilize o algoritmo LMS para identificar um sistema com a função de transferência dada abaixo.
			
				\begin{align}
					H(z) = \frac{1 - z^{-12}}{1 - z^{-1}}
				\end{align}
			
				\begin{enumerate}
					
					\item Calcule o limite superior para $\mu$ (ou seja $\mu_{\text{max}}$) para garantir a estabilidade do algoritmo.
					
						\textcolor{red}{Solução:}
					
					\item Execute o algoritmo para $\frac{\mu_{\text{max}}}{2}$, $\frac{\mu_{\text{max}}}{10}$ e $\frac{\mu_{\text{max}}}{50}$. Comente sobre o comportamento da convergência de cada caso.
						
						\textcolor{red}{Solução:}

					\item Meça o desajuste (misadjustment ) em cada exemplo e comparar com os resultados obtidos pela solução teórica (Eq. (3.50) do livro texto)					

						\textcolor{red}{Solução:}

					\item Mostre o gráfico da resposta em frequência do filtro FIR em qualquer uma das iterações após a convergência ser obtida e compare com o sistema desconhecido.
					
						\textcolor{red}{Solução:}

				\end{enumerate}
			
			\item Seja o canal de comunicações dado por
			
				\begin{align}
					H(z) = 0.5 + 1.2z^{-1} + 1.5z^{-2} + z^{-3},
				\end{align}
				
				e deseja-se projetar um equalizar para o mesmo. A estrutura do equalizador é mostrada na Figura abaixo. Os símbolos $s(n)$ são transmitidos através de um canal e corrompidos por ruído aditivo gaussiano branco complexo $v(n)$. O sinal recebido $x(n)$ é processado pelo equalizador FIR para gerar estimativas $\overset{\sim}{s}(n - \delta)$, as quais são passados por um dispositivo decisor gerando  símbolos $\hat{s}(n - \delta)$. O equalizador possui dois modos de operação: um modo de treinamento durante o qual uma versão atrasada e  replicada da sequência de entrada é usada como o sinal de referência (desejado) e um modo dirigido por decisão no qual a saída do dispositivo de decisão substitui a sequência de referência. O sinal de entrada $s(n)$ é escolhido de uma constelação QAM (por exemplo, 4-QAM, 16-QAM, 64-QAM ou 256-QAM).
				
				\begin{figure}[!ht]
					\centering
					\includegraphics[width=0.85\textwidth]{figs/equalizador_linear.png}
					\caption{Equalizador Linear}
				\end{figure}
			
				\begin{enumerate}
					
					\item Faça um programa que treine o filtro adaptativo com 500 símbolos de uma constelação 4-QAM, seguindo de uma operação dirigida por decisão de 5000 símbolos de uma constelação 16-QAM. Escolha a variância do ruído $\sigma^{2}_{v}$ de maneira que ela promova uma relação sinal ruído de 30 db na entrada do equalizador. Note que os símbolos escolhidos não têm variância unitária. Por esta razão, a a variância do ruído necessita ser ajustada adequadamente para cada uma das diferentes modulações (constelações) QAM para fornecer o nível de SNR desejado. Escolha $\delta = 15$ e o comprimento do equalizador M = 15. Mostre os gráficos da evolução temporal de $s(n)$, $x(n)$ e $\overset{\sim}{s}(n - \delta)$. Use o LMS-normalizado com um fator de passo de $\mu = 0.4$.
								
						\textcolor{red}{Solução:}

					\item Para os mesmos parâmetros do item (a), plote e compare os gráficos de evolução que seriam resultante se o equalizador fosse treinado com 150, 300 e 500 iterações. Use o LMS com um
					$\mu = 0.001$.
					
						\textcolor{red}{Solução:}

					\item Assuma agora que os dados transmitidos foram gerados de uma constelação 256-QAM ao invés de 16-QAM. Plote os gráficos da evolução do sinal na saída do equalizador quando treinado
					usando o LMS-normalizado e 500 símbolos de treinamento.					
					
						\textcolor{red}{Solução:}

					\item Gerar as curvas de taxa de erro de símbolo (SER, do inglês Symbol Error Rate) versus SNR na entrada do equalizador para símbolos de constelações 4, 16, 64 e 256-QAM. Faça SNR variar
					de 5dB a 30dB.
					
						\textcolor{red}{Solução:}

				\end{enumerate}
				
		\end{enumerate}
	
\end{document}

