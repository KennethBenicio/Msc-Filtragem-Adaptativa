\documentclass[a4paper,10pt]{article}
\usepackage[portuguese]{babel}

\usepackage{cite}
\usepackage{float}
\usepackage{xcolor}
\usepackage{graphicx}
\usepackage{fancyhdr}
\usepackage[shortlabels]{enumitem}
\usepackage{amstext,amsmath,amssymb}

\usepackage{sectsty}
\sectionfont{\LARGE}

\setlength{\oddsidemargin}{0cm} %
\setlength{\evensidemargin}{0cm} %
\setlength{\topmargin}{0cm} %
\setlength{\textwidth}{16cm} %
\setlength{\textheight}{22.5cm} %

\pagestyle{empty}
\newcommand{\assunto}{Filtragem Adaptativa}

\sloppy

\begin{document}
	
	\thispagestyle{empty}

\begin{center}
  
    \includegraphics[scale=0.10]{figs/icon.png}
    
    \LARGE{Universidade Federal do Ceará}
    
    \LARGE{Centro de Tecnologia}
    
    \LARGE{Departamento de Engenharia de Teleinformática}
    
    \LARGE{Engenharia de Teleinformática}
    
    \vspace{180pt}
      
    \LARGE{Filtragem Adaptativa}
      
    \LARGE{Listas de Exercícios Propostos}
      
    \vspace{100pt}
    
\end{center}

\vspace{25pt}

\begin{flushleft}
	\begin{tabbing}
		Student \qquad Kenneth Brenner dos Anjos Benício – 519189\\
	   \qquad\qquad\qquad\= \\
		Professor\> Charles Casimiro Cavalcante e Guilherme Barreto\\
		Course \> Filtragem Adaptativa - TIP7188\\
	\end{tabbing}
\end{flushleft}

\vspace{25pt}

\begin{center}
    Fortaleza, 2021
\end{center}
	
	\section*{Estatísticas de Segunda Ordem}
	
		\begin{enumerate}
			
			\item Determine a média e a função de autocorrelação para o processo aleatório em que $v(n)$ é uma sequência de variáveis aleatórias independentes com média $\mu$ e variância $\sigma^2$. $x(n)$ é estacionário? Justifique.
			
				\begin{align}
					&x(n) = v(n) + 3 v(n-1),
				\end{align}
				
				\textcolor{red}{Solução:}
				
				\paragraph{}Para o cálculo da média é possível escrever
				
				\begin{align}
					&E[x(n)] = E[v(n) + 3v(n-1)], 
				\end{align}
				
				mas uma vez que todas as variáveis aleatórias possuem mesma média o resultado será
				
				\begin{align}
					&E[x(n)] = \mu + 3\mu = 4\mu.  
				\end{align}
				
				\paragraph{}De forma semelhante a variância pode ser obtida por 
				
				\begin{align}
					&\mathbb{E}\{[(x(n) - \mu_{X})]^2\}, \\
					&\mathbb{E}\{[x(n) - 4\mu]^2\} = \mathbb{E}\{[v(n) - 3v(n-1) -4\mu]^{2}\}, \\
					&\mathbb{E}\{[x(n) - 4\mu]^2\} = \mathbb{E}\{[v(n) - \mu + 3v(n-1) -3\mu]^{2}\},
				\end{align}
				
				e ao desenvolver o valor quadrado da expressão chega-se em
				
				\begin{align}
					&\mathbb{E}\{[x(n) - 4\mu]^2\} = \mathbb{E}\{ [v(n) - \mu]^{2} + 9[v(n-1) - \mu]^{2} + 6[v(n) - \mu][v(n-1) - \mu] \}, \\
					&\mathbb{E}\{[x(n) - 4\mu]^2\} = \sigma^{2} + 9\sigma^{2} + \mathbb{E}\{6[v(n) - \mu][v(n-1) - \mu]\}.
				\end{align}
				
				\paragraph{}Sabendo que os sinais são descorrelacionados uma vez que temos duas variáveis aleatórias independentes temos
				
				\begin{align}
					&6\mathbb{E}\{[v(n) - \mu][v(n-1) - \mu]\} = \mathbb{E}\{v(n)v(n-1) - \mu v(n) -\mu v(n-1) + \mu^{2}\}, \\
					&6\mathbb{E}\{[v(n) - \mu][v(n-1) - \mu]\} = \mathbb{E}\{v(n)v(n-1)\} - \mu^{2} -\mu^{2} + \mu^{2}, \\
					&6\mathbb{E}\{[v(n) - \mu][v(n-1) - \mu]\} = \mathbb{E}\{v(n)\} \mathbb{E}\{v(n-1)\} - \mu^{2}, \\
					&6\mathbb{E}\{[v(n) - \mu][v(n-1) - \mu]\} = \mu^{2} - \mu^{2} = 0,
				\end{align}
				
				\paragraph{}Portanto, a média pode ser dada por
				
				\begin{align}
					&\mathbb{E}\{[x(n) - 4\mu]^2\} = 10\sigma^{2}. 
				\end{align}
				
				\paragraph{}Além disso, a função de correlação pode ser adquirida por meio da seguinte operação 
				
				\begin{align}
					&\mathbb{E}\{x(n)x^{*}(n)\} = \mathbb{E}\{[v(n) + 3v(n-1)][v(n) + 3v(n-1)]^{*}\}, \\
					&\mathbb{E}\{x(n)x^{*}(n)\} = \mathbb{E}\{[v(n)v^{*}(n)] + 3[v(n)v^{*}(n-1)] + 3[v(n-1)v^{*}(n)] + [9v(n-1)v^{*}(n-1)]\},
				\end{align}
				
				onde é possível reescrever a equação acima ao considerarmos novamente a descorrelação entre as variáveis aleatórias 
				
				\begin{align}
					&\mathbb{E}\{x(n)x^{*}(n)\} = \mathbb{E}\{\mu^{2} + 3\mu^{2} + 3\mu^{2} + 9\mu^{2}\}, \\
					&\mathbb{E}\{x(n)x^{*}(n)\} = 16\mu^{2}.
				\end{align}
				
				\paragraph{}Uma vez que as estatísticas de primeira e de segunda ordem são independentes quanto ao deslocamento temporal é possível afirmar que o processo descrito é Estacionário no Sentido Amplo(WSS). Entretanto, não é possível afirmar que o processo é Estacionário no Sentido Estrito(SSS) uma vez que não se tem conhecimento da função que descreve a densidade de probabilidade desse sistema.
			
			\item Sejam os processos aleatórios $x(n)$ e $y(n)$ definidos por
			
				\begin{align} 
					&x(n) = v_1(n) + 3v_2(n-1), \\
					&y(n) = v_2(n + 1) + 3v_1(n-1),
				\end{align}
				
				em que $v_1(n)$ e $v_2(n)$ são processos de ruído branco independentes cada um com variância igual a $0,5$.
				
				\begin{enumerate}
					
					\item Quais são as funções de autocorrelação de $x$ e de $y$? Os processos são WSS?
					
						\textcolor{red}{Solução:}
						
						\paragraph{}Antes de tudo é de interesse definir o processo estocástico de ruído branco. Um processo desse tipo é formado por um número qualquer de amostras
						independentes entre si e com média nula. Desse modo, uma vez que a média de processos de ruído branco são nulas é possível obter facilmente a média dos novos processos formados pela combinação dos ruídos
						
						\begin{align}
							&\mathbb{E}\{x(n)\} = \mathbb{E}\{v_{1}(n) + 3v_{2}(n-1)\} = \mu_{1} + 3\mu_{2} = 0, \\
							&\mathbb{E}\{y(n)\} = \mathbb{E}\{v_{2}(n+1) + 3v_{1}(n-1)\} = \mu_{2} + 3\mu_{1} = 0.
						\end{align}
						
						\paragraph{}Para a variância teremos uma resposta similar
						
						\begin{align*}
							&\mathbb{E}\{[x(n) - \mu]^{2}\} = \mathbb{E}\{[x(n) - 0]^{2}\} = \mathbb{E}\{[x(n)]^{2}\}, \\
							&\mathbb{E}\{[x(n) - \mu]^{2}\} = \mathbb{E}\{[v_{1}(n) + 3v_{2}(n-1)]^{2}\} = \mathbb{E}\{ [v^{2}_{1}(n)] + 6[v_{1}(n)v_{2}(n-1)] + 9[v^{2}_{2}(n-1)]\}, \\
							&\mathbb{E}\{[y(n) - \mu]^{2}\} = \mathbb{E}\{[y(n) - 0]^{2}\} = \mathbb{E}\{[y(n)]^{2}\}, \\
							&\mathbb{E}\{[y(n) - \mu]^{2}\} = \mathbb{E}\{[v_{2}(n+1) + 3v_{1}(n-1)]^{2}\} = \mathbb{E}\{ [v^{2}_{2}(n+1)] + 6[v_{2}(n+1)v_{1}(n-1)] + 9[v^{2}_{1}(n-1)]\}.
						\end{align*}
						
						\paragraph{}Rescrevendo a expressão para obter a formula da variância e utilizando o fato dos processos de ruído serem descorrelacionados uma nova expressão é obtida
						
						\begin{align}
							&\mathbb{E}\{[x(n) - \mu]^{2}\} = \mathbb{E}\{ [v^{2}_{1}(n) - 0] + 0 + 9[v^{2}_{2}(n-1) - 0]\}, \\
							&\mathbb{E}\{[x(n) - \mu]^{2}\} = \sigma^{2}_{1} + 9\sigma^{2}_{2} = 10*0.5 = 5, \\
							&\mathbb{E}\{[y(n) - \mu]^{2}\} = \mathbb{E}\{ [v^{2}_{2}(n+1) - 0] + 0 + 9[v^{2}_{1}(n-1) - 0]\}, \\
							&\mathbb{E}\{[y(n) - \mu]^{2}\} = \sigma^{2}_{2} + 9\sigma^{2}_{1} = 10*0.5 = 5.
						\end{align}
						
						\paragraph{}A função de autocorrelação para $X$ é calculada como se segue 
						
						\begin{align}
							r_{x}(\tau) &= \mathbb{E}\{x(n)x(n + \tau)\} = \mathbb{E}\{[v_{1}(n) + 3v_{2}(n-1)][v_{1}(n + \tau) + 3v_{2}(n-1 + \tau)]\}, \\
							r_{x}(\tau) &= \mathbb{E}\{[v_{1}(n)v_{1}(n + \tau)] + 3[v_{1}(n)v_{2}(n-1 + \tau)] + 3[v_{2}(n-1)v_{1}(n + \tau)] \\ 
							&+ 9[v_{2}(n-1)v_{2}(n-1 + \tau)] \}.    
						\end{align}
						
						\paragraph{}Sabendo que amostras de um mesmo processo são descorrelacionadas e que ambos os processos também são descorrelacionados, além de termos conhecimento 
						de que ambos possuem média nula, então 
						
						\begin{align}
							&\mathbb{E}\{[v_{1}(n)v_{1}(n + \tau)]\} = \mathbb{E}\{[v_{1}(n)]\} \mathbb{E}\{[v_{1}(n + \tau)]\} = 0, \\ 
							&3\mathbb{E}\{[v_{1}(n)v_{2}(n-1 + \tau)]\} = 3\mathbb{E}\{[v_{1}(n)]\} \mathbb{E}\{[v_{2}(n-1 + \tau)]\} = 0, \\
							&3\mathbb{E}\{[v_{2}(n-1)v_{1}(n + \tau)]\} = 3\mathbb{E}\{[v_{2}(n-1)]\} \mathbb{E}\{[v_{1}(n + \tau)]\} = 0, \\
							&9\mathbb{E}\{[v_{2}(n-1)v_{2}(n-1 + \tau)]\} = 9\mathbb{E}\{[v_{2}(n-1)]\} \mathbb{E}\{[v_{2}(n-1 + \tau)]\} = 0, \\ 
							&r_{x}(\tau) = 0.
						\end{align}
						
						\paragraph{}Repetindo o procedimento para $Y$ e omitindo informações redundantes chega-se ao seguinte resultado
						
						\begin{align}
							&r_{y}(\tau) = 0.
						\end{align}
						
						\paragraph{}Sendo as estatísticas de primeira e de segunda ordem independentes quanto ao tempo para os dois processos então é possível afirmar que eles são WSS.
					
					\item Qual é a função de correlação cruzada $r_{xy}(n_1,n_0)$? Estes processos são conjuntamente estacionários (no sentido amplo)? Justifique.
					
						\textcolor{red}{Solução:}
						
						\begin{align}
							r_{x,y}(n_{1},n_{0}) &= \mathbb{E}\{[x(n_{1})y^{*}(n_{0})]\} = \mathbb{E}\{[v_{1}(n_{1}) + 3v_{2}(n_{1}-1)][v_{2}(n_{0}+1) + 3v_{1}(n_{0}-1)]^{*}\}, \\
							r_{x,y}(n_{1},n_{0}) &= \mathbb{E}\{[v_{1}(n_{1})v^{*}_{2}(n_{0}+1)] + 3[v_{1}(n_{1})v^{*}_{1}(n_{0}-1)] + 3[v_{2}(n_{1}-1)v^{*}_{2}(n_{0}+1)] \\ 
							&+ 9[v_{2}(n_{1}-1)v^{*}_{1}(n_{0}-1)]\}.  
						\end{align}
						
						\paragraph{}Novamente considerando que os processos de ruído branco são descorrelacionadas entre si e que as amostras individuais são também independentes então podemos simplificar a expressão acima da seguinte forma
						
						\begin{align}
							&r_{x,y}(n_{1},n_{0}) = 0 
						\end{align}
						
						\paragraph{}Sendo assim, os processos podem ser considerados conjuntamente estacionários uma vez que a correlação cruzada é independente do instante temporal e que os processos que compoem o processo conjunto podem 
						ser considerados estacionários em sentido amplo isoladamente. Entretanto, vale notar que foi possível realizar tal afirmação apenas por termos conhecimento da relação de independência entre os processos 
						de ruído branco, pois de outro modo não seria possível realizar afirmações quanto a correlação cruzada do processo conjunto.
					
				\end{enumerate}
			
			\item Quais as condições que os elementos de uma matriz
			
				\begin{align}
					\mathbf{R} = \left[
					\begin{matrix}
						a & b  \\
						c & d  \\
					\end{matrix} \right]
				\end{align}
				
				devem satisfazer tal que $\mathbf{R}$ seja uma matriz de autocorrelação válida de
			
				\begin{enumerate}
					
					\item Um vetor aleatório bidimensional?
					
						\textcolor{red}{Solução:}
						
						\paragraph{}Em suma, precisamos estar atentos às seguintes propriedades para garantir que temos em mãos uma matriz de autocorrelação válida
						
						
						- $\mathbf{R_{x}} = \mathbf{R^{H}_{x}}$
						
						- $\mathbf{a^{H}} \mathbf{R_{xa}} \geq 0$
						
						- $\mathbf{Ax} = \lambda \mathbf{x}, \forall \lambda \geq 0 \text{ and } \mathbf{x} \in \mathbb{R}$
						
						\paragraph{}Desse modo, considerando um vetor aleatório bidimensional descrito por $\mathbf{X} = (x_{1},x_{2})$ podemos então
						
						\begin{align}
							\mathbf{R} &= \left[ 
							\begin{matrix}
								\mathbb{E}\{[x^{2}_{1}]\} & \mathbb{E}\{[x_{1}x^{*}_{2}]\} \\
								\mathbb{E}\{[x_{2}x^{*}_{1}]\} & \mathbb{E}\{[x^{2}_{2}]\} \\
							\end{matrix} \right], \\
							\mathbf{R}^{\text{H}} &= \left[ 
							\begin{matrix}
								\mathbb{E}\{[x^{2}_{1}]\} & \mathbb{E}\{[x_{2}x^{*}_{1}]\} \\
								\mathbb{E}\{[x_{1}x^{*}_{2}]\} & \mathbb{E}\{[x^{2}_{2}]\} \\
							\end{matrix} \right].
						\end{align}
						
						\paragraph{}Inicialmente, podemos garantir a simetria quanto ao hermitiano se fizermos 
						
						\begin{align}
							&\mathbb{E}\{[x_{1}x^{*}_{2}]\} \overset{\Delta}{=} \mathbb{E}\{[x_{2}x^{*}_{1}]\},
							&\mathbb{E}\{[x_{2}x^{*}_{1}]\} \overset{\Delta}{=} \mathbb{E}\{[x_{1}x^{*}_{2}]\}.
						\end{align}
						
						\paragraph{}Entretanto, sabemos que o operador esperança é linear tornando assim as expressões acima equivalentes. Em continuidade do problema, podemos 
						considerar que a restrição as quais os autovalores estão submetidos pode ser facilmente atingida ao garantirmos que o determinante da matriz 
						de correlação seja maior que a nulidade
						
						\begin{align}
							&\mathbb{E}\{[x^{2}_{1}]\} \mathbb{E}\{[x^{2}_{1}]\} - \mathbb{E}\{[x_{1}x_{2}]\} \mathbb{E}\{[x_{2}x_{1}]\} > 0, \\
							&\mathbb{E}\{[x_{1}^{2}]\} \mathbb{E}\{[x^{2}_{2}]\} > \mathbb{E}\{[x_{1}x_{2}]\} \mathbb{E}\{[x_{2}x_{1}]\}.
						\end{align}
				
					\item Um processo estocástico estacionário escalar?
					
						\textcolor{red}{Solução:}
						
						\paragraph{}Considerando um processo estocástico estacionário escalar do tipo $\mathbf{X}_{t} = x(t)$ e uma versão atrasada desse
						processo definida por $\mathbf{X}_{t + \tau} = x(t + \tau)$ temos que a matriz de correlação pode ser escrita da seguinte forma
						
						\begin{align}
							\mathbf{R} &= \left[ 
							\begin{matrix}
								\mathbb{E}\{[x^{2}(t)_{1}]\} & \mathbb{E}\{[x(t)x^{*}(t + \tau)]\} \\
								\mathbb{E}\{[x(t + \tau)x^{*}(t)]\} & \mathbb{E}\{[x^{2}(t + \tau)]\} \\
							\end{matrix} \right], \\
							\mathbf{R}^{\text{H}} &= \left[ 
							\begin{matrix}
								\mathbb{E}\{[x^{2}(t)_{1}]\} & \mathbb{E}\{[x(t + \tau)x^{*}(t)]\} \\
								\mathbb{E}\{[x(t)x^{*}(t + \tau)]\} & \mathbb{E}\{[x^{2}(t + \tau)]\} \\
							\end{matrix} \right].
						\end{align}
						
						
						\paragraph{}Em sequência, podemos garantir a simetria quanto ao hermitiano se fizermos 
						
						\begin{align}
							&\mathbb{E}\{[x(t)x^{*}(t + \tau)]\} \overset{\Delta}{=} \mathbb{E}\{[x(t + \tau)x^{*}(t)]\}, \\
							&\mathbb{E}\{[x(t + \tau)x^{*}(t)]\} \overset{\Delta}{=} \mathbb{E}\{[x(t)x^{*}(t + \tau)]\},
						\end{align}
						
						mas, mais uma vez, considerando que o operador esperança é linear então as duas expressões são equivalentes. Já considerando a
						restrição imposta aos autovalores da matriz temos novamente
						
						\begin{align}
							&\mathbb{E}\{[x^{2}(t)]\} \mathbb{E}\{[x^{2}(t + \tau)]\} > \mathbb{E}\{[x(t)x^{*}(t + \tau)]]\}  \mathbb{E}\{[x(t + \tau)]x^{*}(t)]\}.
						\end{align}
					
				\end{enumerate}
			
			\item Assuma que a inversa $\mathbf{R}_{\mathbf{x}}^{-1}$ da matriz de autocorrelação de um vetor coluna $N$-dimensional exista. Mostre que
			
				\begin{align}
					\mathbb{E}\left\{\mathbf{x}^H \mathbf{R}_{\mathbf{x}}^{-1} \mathbf{x} \right\} = N
				\end{align}
				
				\textcolor{red}{Solução:}
				
				\paragraph{}Inicialmente podemos escrever a expressão regular para a matriz de autocorrelação e supondo que de fato existe uma inversa bem definida 
				para a matriz de autocorrelação
				
				\begin{align}
					\mathbb{E}\{\mathbf{x} \mathbf{x}^{\text{H}}\} = \mathbf{R}_{x}, \\
					\mathbb{E}\{\mathbf{x} \mathbf{x}^{\text{H}}\}\mathbf{R}^{-1}_{x} = \mathbf{R}_{x}\mathbf{R}^{-1}_{x}, \\
					\mathbb{E}\{\mathbf{x} \mathbf{x}^{\text{H}}\mathbf{R}^{-1}_{x}\} = \mathbf{I}_{N}, \\
				\end{align}
				
				\paragraph{}Desse modo, podemos aplicar o operador traço de matriz e por meio da propriedade de permutação cíclica desse operador chegamos ao seguinte resultado
				
				\begin{align}
					\text{Tr}\{\mathbb{E}\{\mathbf{x} \mathbf{x}^{\text{H}}\mathbf{R}^{-1}_{x}\}\} = \text{Tr}\{\mathbf{I}_{N}\}, \\
					\text{Tr}\{\mathbb{E}\{\mathbf{x}^{\text{H}}\mathbf{R}^{-1}_{x} \mathbf{x}\}\} = \text{Tr}\{\mathbf{I}_{N}\}, \\
					\text{Tr}\{\mathbb{E}\{\mathbf{x}^{\text{H}}\mathbf{R}^{-1}_{x} \mathbf{x}\}\} = N,
				\end{align}
			
				onde a ultima expressão se justifica pois temos uma matriz identidade de ordem $N$ no lado direito. Desse modo, seu traço é dado por $\sum^{N}_{i = 1} 1 = N$.
				
			\item Mostre que as matrizes de correlação e covariância satisfazem as relações abaixo:
				
				\begin{enumerate}
					
					\item $\mathbf{R}_\mathbf{x} = \mathbf{C}_\mathbf{x} + {\mathbf{\mu}}_{\mathbf{x}}{\mathbf{\mu}}_{\mathbf{x}}^H$
					
						\textcolor{red}{Solução:}
						
						\begin{align}
							&C_{X} = \mathbb{E}\{[(x - \mu)(x - \mu)^{H}]\} = \mathbb{E}\{[xx^{H} -x\mu^{H} - \mu x^{H} + \mu \mu^{H}]\}, \\
							&C_{X} = \mathbb{E}\{[xx^{H}]\} -\mathbb{E}\{[x\mu^{H}]\} - \mathbb{E}\{[\mu x^{H}]\} + \mathbb{E}\{[\mu \mu^{H}]\}.
						\end{align}
						
						
						\paragraph{}Considerando a matriz de correlação pode ser escrita por $R_{X} = \mathbb{E}\{[xx^{H}]\}$ e que o valor médio de um escalar é o próprio escalar temos
						
						\begin{align}
							&C_{X} = R_{X} - \mathbb{E}\{[x\mu^{H}]\} - \mathbb{E}\{[\mu x^{H}]\} + \mu \mu^{H}, \\
							&C_{X} = R_{X} - \mu^{H}\mathbb{E}\{[x]\} - \mu \mathbb{E}\{[x^{H}]\} + \mu \mu^{H}, \\
							&C_{X} = R_{X} - \mu \mu^{H} - \mu \mu^{H} + \mu \mu^{H}, \\
							&C_{X} = R_{X} - \mu \mu^{H}, \\
							&R_{X} = C_{X} + \mu \mu^{H}.
						\end{align}
					
					\item $\mathbf{C}_{\mathbf{x} + \mathbf{y}} = \mathbf{C}_{\mathbf{x}} +
					\mathbf{C}_{\mathbf{y}}$, para $\mathbf{x}$ e $\mathbf{y}$ descorrelacionados
					
						\textcolor{red}{Solução:}
						
						\begin{align}
							&\mathbf{C}_{\mathbf{x} + \mathbf{y}} = \mathbf{C}_{\mathbf{x}} + \mathbf{C}_{\mathbf{xy}} + \mathbf{C}_{\mathbf{yx}} + \mathbf{C}_{\mathbf{y}}.
						\end{align}
						
						\paragraph{}Onde os termos de correlação cruzada $\mathbf{C}_{\mathbf{xy}}$ e $\mathbf{C}_{\mathbf{yx}}$ podem ser obtidos  da seguinte maneira
						
						\begin{align}
							&\mathbf{C_{xy}} = \mathbb{E}\{[x - \mu_{x}][y - \mu_{y}]\}, \\
							&\mathbf{C_{xy}} = \mathbb{E}\{[xy]\} - \mathbb{E}\{[x\mu_{y}]\} - \mathbb{E}\{[\mu_{x} y]\} + \mathbb{E}\{[\mu_{x} \mu_{y}]\}, \\
							&\mathbf{C_{xy}} = \mathbb{E}\{[xy]\} - \mu_{y} \mu_{x} - \mu_{x} \mu_{y} + \mu_{x} \mu_{y}, \\
							&\mathbf{C_{xy}} = \mathbb{E}\{[xy]\} + \mu_{x} \mu_{y}, \\
							&\mathbf{C_{xy}} = \mu_{x} \mu_{y}, \\
						\end{align}
						
						\begin{align}
							&\mathbf{C_{yx}} = \mathbb{E}\{[y - \mu_{y}][x - \mu_{x}]\}, \\
							&\mathbf{C_{yx}} = \mathbb{E}\{[yx]\} - \mathbb{E}\{[y\mu_{x}]\} - \mathbb{E}\{[\mu_{y} x]\} + \mathbb{E}\{[\mu_{y} \mu_{x}]\}, \\
							&\mathbf{C_{yx}} = \mathbb{E}\{[yx]\} - \mu_{y} \mu_{x} - \mu_{x} \mu_{y} + \mu_{y} \mu_{x}, \\
							&\mathbf{C_{yx}} = \mathbb{E}\{[yx]\} - \mu_{x} \mu_{y}, \\
							&\mathbf{C_{yx}} = - \mu_{x} \mu_{y}.
						\end{align}
						
						\paragraph{}Portanto, é possível afirmar que a soma dos termos de correlação cruzada é nula e que a correlação da soma de dois vetores descorrelacionados é de fato 
						dada por $\mathbf{C}_{\mathbf{x} + \mathbf{y}} = \mathbf{C}_{\mathbf{x}} + \mathbf{C}_{\mathbf{y}}$
						
						\begin{align}
							&\mathbf{C_{xy}} + \mathbf{C_{yx}}  = \mu_{x} \mu_{y} - \mu_{x} \mu_{y}, \\
							&\mathbf{C_{xy}} + \mathbf{C_{yx}}  = 0.
						\end{align}
					
					
				\end{enumerate}
			
			\item Processos aleatórios $v_1(n)$ e $v_2(n)$  são independentes e têm a mesma função de correlação
			
				\begin{align}
					r_v(n_1,n_0) = 0.5\delta(n_1 - n_0)
				\end{align}
			 
			 	\begin{enumerate}
			 		
			 		\item Qual é a função de correlação do processo aleatório
			 		
				 		\begin{align}
				 			x(n) = v_1(n) + 2v_1(n + 1) + 3v_2(n-1) ?
				 		\end{align}
				 		
				 		\paragraph{}Este é um processo WSS? Justifique.
				 		
				 		\textcolor{red}{Solução:}
				 		
				 		\paragraph{}Para simplificar o desenvolvimento iremos considerar que os processos possuem media nula e são descorrelacionados
				 		
				 		\begin{align*}
				 			r_{x} &= \mathbb{E}\{x(n) x^{*}(n)\}, \\
				 			r_{x} &= \mathbb{E}\{[v_{1}(n) + 2v_{1}(n+1) + 3v_{2}(n-1)] [v_{1}(n) + 2v_{1}(n+1) + 3v_{2}(n-1)]^{*}(n)\}, \\
				 			r_{x} &= \mathbb{E}\{[v_{1}(n)v^{*}_{1}(n) + 2v_{1}(n)v^{*}_{1}(n+1) + 3v_{1}(n)v^{*}_{2}(n-1) + 2v_{1}(n + 1)v^{*}_{1}(n) + 4v_{1}(n+1)v^{*}_{1}(n+1) \\
							&+6v_{1}(n+1)v^{*}_{2}(n-1)] + 3v_{2}(n-1)v^{*}_{1}(n) +  6v_{2}(n-1)v^{*}_{1}(n+1) + 9v_{2}(n-1)v^{*}_{2}(n-1)\}, \\
				 			r_{x} &= r_{v}(n,n) + 2r_{v}(n,n+1) + 0 + 2r_{v}(n+1,n) + 4r_{v}(n+1,n+1) + 0 + 0 + 0 + 9r_{v}(n-1,n-1), \\
				 			r_{x} &= r_{v}(n,n) + 2r_{v}(n,n+1) + 2r_{v}(n+1,n) + 4r_{v}(n+1,n+1) + 9r_{v}(n-1,n-1).
				 		\end{align*}
				 		
				 		\paragraph{}Após uma breve análise na expressão $r_{v}(n_{1},n_{0})$ é possível verificar que determinados termos anulam-se ao considerar um único momento temporal devido a presença da função degrau unitário, permitindo a seguinte simplificação:
				 		
				 		\begin{align}
				 			&r_{x} = 2r_{v}(n,n+1) + 2r_{v}(n+1,n), \\
				 			&r_{x} = \delta(n - n - 1) + \delta(n + 1 - n).
				 		\end{align}
				 		
				 		\paragraph{}Onde a generalização pode ser descrita por:
				 		
				 		\begin{align}
				 			&r_{x}(n_{1}, n_{2})= \delta(n_{1} - n_{2}) + \delta(n_{2} - n_{1}).  
				 		\end{align}
				 		
				 		\paragraph{}Uma vez que a correlação é dependente apenas de um deslocamento temporal, então podemos classificar esse processo como WSS.
			 		
			 		\item Encontre a a matrix de correlação de um vetor aleatório consistindo de oito amostras consecutivas de
			 		$x(n)$.
			 		
				 		\textcolor{red}{Solução:}
				 		
				 		\begin{align}
				 			\mathbf{R}_{\mathbf{x}} = \left[
				 			\begin{matrix}
				 				2 & 0 & 0 & 0 & 0 & 0 & 0 & 0\\
				 				0 & 2 & 0 & 0 & 0 & 0 & 0 & 0\\
				 				0 & 0 & 2 & 0 & 0 & 0 & 0 & 0\\
				 				0 & 0 & 0 & 2 & 0 & 0 & 0 & 0\\
				 				0 & 0 & 0 & 0 & 2 & 0 & 0 & 0\\
				 				0 & 0 & 0 & 0 & 0 & 2 & 0 & 0\\
				 				0 & 0 & 0 & 0 & 0 & 0 & 2 & 0\\
				 				0 & 0 & 0 & 0 & 0 & 0 & 0 & 2
				 			\end{matrix}
				 			\right]
				 		\end{align}
				 		
				 		\paragraph{}Para chegar a esse resultado foi utilizado a expressão para a correlação obtida no item anterior. Uma vez que a função delta de dirac terá um valor não nulo apenas quando o argumento for zero, isso irá acontecer apenas quando os momentos $n_{1}$ e $n_{2}$ forem iguais e isso irá acontecer apenas com os elementos da diagonal principal.
			 		
			 	\end{enumerate}
		 	
		\end{enumerate}
		
	\newpage
	\section*{Filtragem Linear Ótima}
	
		\begin{enumerate}
			
			\item Considere um problema de filtragem de Wiener conforme caracterizado a seguir. A matriz de correlação $\mathbf{R}_{\mathbf{x}}$ de um vetor de entrada $\mathbf{x}(n)$ é dada por
			
				\begin{align*}
					\mathbf{R}_{\mathbf{x}} = \left[ \begin{matrix}
						1 & {0.5}  \\   {0.5} & 1  \\ \end{matrix} \right].
				\end{align*}
				
				\paragraph{}O vetor de correlação cruzada $\mathbf{p}_{\mathbf{x}d}$ entre o vetor de entrada $\mathbf{x}$ e a resposta desejada
				$d(n)$ é
				
				\begin{align*}
					\mathbf{p}_{\mathbf{x}d} &= \left[ \begin{matrix}   {0.5}  \\   {0.25}  \\ \end{matrix} \right]
				\end{align*}
				
				\begin{enumerate}
					
					\item Encontre o vetor de coeficientes do filtro de Wiener.
					
						\textcolor{red}{Solução:}
						
						\paragraph{}Isso pode ser realizado de forma simples pelo uso da equação do filtro ótimo de wiener
						
						\begin{align}
							\mathbf{w}_{\text{opt}} &= \mathbf{R}^{-1}_{X} \mathbf{p}_{xd}, \\
							\mathbf{w}_{\text{opt}} &=  \left[ \begin{matrix} 1 & 0.5 \\ 0.5 & 1 \end{matrix} \right]  \left[ \begin{matrix} 0.5 \\ 0.25 \end{matrix} \right], \\
							\mathbf{w}_{\text{opt}} &=  \frac{4}{3} \left[ \begin{matrix} 1 & -0.5 \\ -0.5 & 1 \end{matrix} \right]  \left[ \begin{matrix} 0.5 \\ 0.25 \end{matrix} \right], \\
							\mathbf{w}_{\text{opt}} &= \left[ \begin{matrix} 0.5 \\ 0.0 \end{matrix} \right].
						\end{align}
					
					\item Qual é o mínimo erro médio quadrático fornecido por este filtro?
					
						\textcolor{red}{Solução:}
						
						\begin{align}
							\mathbb{E}\{e^{2}(n)\}  &=\sigma^{2}_{d} - 2\mathbf{w}^{T}\mathbf{p}_{xd} + w^{T}\mathbf{R}_{X}\mathbf{w}      
						\end{align}
						
						\paragraph{}Ao aplicar o vetor descoberto no item anterior obtem-se o erro mínimo
						
						\begin{align}
							e_{min}  &=\sigma^{2}_{d} - 2\mathbf{w}^{T}_{opt}\mathbf{p}_{xd} + w^{T}_{opt}\mathbf{R}_{X}\mathbf{w}_{\text{opt}}, \\
							e_{min} &= \sigma^{2}_{d} - 2 \left[ \begin{matrix} 0.5 & 0.0 \end{matrix} \right] \left[ \begin{matrix} 0.5 \\ 0.25 \end{matrix} \right] + \left[ \begin{matrix} 0.5 & 0.0 \end{matrix} \right] \left[ \begin{matrix} 1 & -0.5 \\ -0.5 & 1 \end{matrix} \right]  \left[ \begin{matrix} 0.5  \\ 0.0 \end{matrix} \right], \\
							e_{min} &= \sigma^{2}_{d} - 2 * 0.25 + 0.25, \\
							e_{min} &= \sigma^{2}_{d} - 0.25. 
						\end{align}
						
						\paragraph{}Dessa forma o erro é dependente da variância do sinal desejado.					
					
					\item Formule uma representação do filtro de Wiener em termos dos autovalores da matriz $\mathbf{R}_{\mathbf{x}}$
					e de seus autovetores associados.
					
						\textcolor{red}{Solução:}
						
						\paragraph{}Utilizando a decomposição matricial em autovalores (EVD) é possível reescrever a matrix de correlação como abaixo
						
						\begin{align}
							\mathbf{R}_{X} &= \mathbf{Q} \mathbf{\Lambda} \mathbf{Q}^{-1}.
						\end{align}
						
						\paragraph{}A matriz $\mathbf{\Lambda}$ contémm os autovalores $\lambda_{i}$ e a matriz $\mathbf{Q}$ os respectivos autovetores. Em posse dessa relação é possível reescrever a equação do fitro ótimo de wiener como
						
						\begin{align}
							\mathbf{w}_{\text{opt}} &= \mathbf{R}^{-1}_{X} \mathbf{p}_{xd}, \\
							\mathbf{w}_{\text{opt}} &= (\mathbf{Q} \mathbf{\Lambda} \mathbf{Q}^{-1})^{-1} \mathbf{p}_{xd}, \\
							\mathbf{w}_{\text{opt}} &= \mathbf{Q}^{-1} \mathbf{\Lambda}^{-1} \mathbf{Q} \mathbf{p}_{xd}.
						\end{align}
						
						\paragraph{}É possível verificar de imediato que essa propriedade é bem útil uma vez que a inversa de uma matriz diagonal é certamente menos custosa que a inversa da matriz de autocorrelação completa.
					
				\end{enumerate}
			
			\item Mostre que a equação do erro mínimo pode se escrita da seguinte maneira
			
				\begin{align*}
					\mathbf{A}\left[ \begin{matrix}
						1  \\
						{ - \mathbf{w}}  \\
					\end{matrix} \right] = \left[ \begin{matrix}
						J_{\text{min}}  \\
						\mathbf{0}  \\
					\end{matrix} \right],
				\end{align*}
				
				em que $J_{\text{min}}$ é o mínimo erro médio quadrático, $\mathbf{w}$ é o filtro de Wiener, e $\mathbf{A}$ é a matriz de
				correlação do vetor aumentado
				
				\begin{align*}
					\left[ \begin{matrix}
						d(n)  \\
						\mathbf{x}(n)  \\
					\end{matrix} \right],
				\end{align*}
				
				em que $d(n)$ é o sinal desejado e $\mathbf{x(n)}$ é o sinal de entrada do filtro de Wiener.
				
				\textcolor{red}{Solução:}
				
				\paragraph{}Inicialmente deve-se calcular a matriz de correlação do vetor aumentado
				
				\begin{align}
					\mathbf{A} &= \mathbb{E} \{ \left[ \begin{matrix} d(n) \\ x(n) \end{matrix} \right] \left[ \begin{matrix} d(n)^{T} & x(n)^{T} \end{matrix} \right] \}, \\
					\mathbf{A} &=   \left[ \begin{matrix} \mathbb{E}\{d(n) d(n)^{T}\} & \mathbb{E}\{d(n) x(n)^{T}\} \\ \mathbb{E}\{x(n) d(n)^{T}\} & \mathbb{E}\{x(n) x(n)^{T}\} \end{matrix} \right].
				\end{align}
				
				\paragraph{}Dando as devidas nomeações aos termos a expressão acima reduz-se a
				
				\begin{align}
					\mathbf{A} &=  \left[ \begin{matrix} \sigma^{2}_{d} & \mathbf{p}_{xd}^{T} \\
						\mathbf{p}_{xd} & \mathbf{R}_{X} \end{matrix} \right].
				\end{align}
				
				\paragraph{}Multiplicando-se a expressão acima pelo vetor $[1 - \mathbf{w}]^{T}$
				
				\begin{align}
					\mathbf{A} \left[ \begin{matrix} 1 \\ -\mathbf{w} \end{matrix} \right] &=   \left[ \begin{matrix} \sigma^{2}_{d} & \mathbf{p}_{xd}^{T} \\
						\mathbf{p}_{xd} & \mathbf{R}_{X} \end{matrix} \right] \left[ \begin{matrix} 1 \\ -\mathbf{w} \end{matrix} \right], \\
					\mathbf{A} \left[ \begin{matrix} 1 \\ -\mathbf{w} \end{matrix} \right] &=   \left[ \begin{matrix} \sigma^{2}_{d} - \mathbf{p}_{xd}^{T}\mathbf{w} \\
						\mathbf{p}_{xd} - \mathbf{R}_{X}\mathbf{w} \end{matrix} \right].
				\end{align}
				
				\paragraph{}Por fim, resta apenas aplica a equação ótima do filtro de wiener $\mathbf{w}_{\text{opt}} = \mathbf{R}^{-1}_{X} \mathbf{p}_{xd}$
				
				\begin{align}
					\mathbf{A} \left[ \begin{matrix} 1 \\ -\mathbf{w} \end{matrix} \right] &=   \left[ \begin{matrix} \sigma^{2}_{d} - \mathbf{p}_{xd}^{T}\mathbf{R}^{-1}_{X} \mathbf{p}_{xd} \\
						\mathbf{p}_{xd} - \mathbf{R}_{X}\mathbf{R}^{-1}_{X} \mathbf{p}_{xd} \end{matrix} \right], \\
					\mathbf{A} \left[ \begin{matrix} 1 \\ -\mathbf{w} \end{matrix} \right] &=   \left[ \begin{matrix} \sigma^{2}_{d} - \mathbf{p}_{xd}^{T}\mathbf{R}^{-1}_{X} \mathbf{p}_{xd} \\
						\mathbf{p}_{xd} - \mathbf{I}_{X}\mathbf{p}_{xd} \end{matrix} \right], \\
					\mathbf{A} \left[ \begin{matrix} 1 \\ -\mathbf{w} \end{matrix} \right] &=   \left[ \begin{matrix} \sigma^{2}_{d} - \mathbf{p}_{xd}^{T
						}\mathbf{R}^{-1}_{X} \mathbf{p}_{xd} \\ 0 \end{matrix} \right], \\
					\mathbf{A} \left[ \begin{matrix} 1 \\ -\mathbf{w} \end{matrix} \right] &=   \left[ \begin{matrix} J_{min} \\ 0 \end{matrix} \right].
				\end{align}
			
			\item Em várias aplicações práticas há uma necessidade de cancelar ruído que foi adicionado a um sinal. Por exemplo, se estamos usando o telefone celular dentro de um ruído e o ruído do carro ou rádio é adicionado à mensagem que estamos tentando transmitir. A Figura abaixo ilustra as situações de contaminação de ruído. Calcule o filtro de Wiener (filtro ótimo) de tal configuração em relação às estatísticas dos sinais envolvidos que você dispõe (conhece).
			
				\begin{figure}[H]
					\centering
					\includegraphics[width=0.5\textwidth]{figs/cancelamento_ruido.png}
					\caption{Estrutura de Equalização de Canal}
				\end{figure}
				
				\textcolor{red}{Solução:}
				
				\paragraph{}Inicialmente é necessário calcular a equação de erro do sistema aqui proposto
				
				\begin{align}
					e(n) &= x(n) - \hat{v_{1}} = x(n) - \mathbf{w}^{T}v_{2}(n)
				\end{align}
				
				\paragraph{}Em seguida faz-necessário calcular a função mean square error(MSE) que é facilmente fornecida pela manipulação algébrica abaixo
				
				\begin{align}
					e^{2}(n) &= [x(n) - \mathbf{w}^{T}v_{2}(n)][x(n) - \mathbf{w}^{T}v_{2}(n)]^{T}, \\
					e^{2}(n) &= x^{2}(n) - 2x(n)\mathbf{w}^{T}v_{2}(n) + \mathbf{w}^{T}v_{2}(n)v_{2}^{T}\mathbf{w}.
				\end{align}
				
				\paragraph{}Sendo considerado que o filtro apresenta coeficientes constantes é possível aplicar o operador Valor Esperado de forma a obter a seguinte relação
				
				\begin{align}
					\mathbb{E}\{e^{2}(n)\} = \mathbb{E}\{x^{2}(n)\} - 2\mathbf{w}^{T}\mathbb{E}\{x(n) v_{2}(n)\} + \mathbf{w}^{T}\mathbb{E}\{v_{2}(n)v_{2}(n)^{T}\} \mathbf{w},& \\
					\mathbb{E}\{e^{2}(n)\} = \sigma^{2}_{x} - 2\mathbf{w}^{T}\mathbf{p}_{xv_{2}} + \mathbf{w}^{T}\mathbf{R}_{v_{2}} \mathbf{w}.&
				\end{align}
				
				\paragraph{}Por fim, basta encontrar o $\mathbf{w}$ que minimiza o MSE acima. Para chegar a esse fim, calcula-se o gradiante quanto ao $\mathbf{w}$ igualando-se o resultado da operação a zero
				
				\begin{align}
					\nabla_{\mathbf{w}} \mathbb{E}\{e^{2}(n)\} = - 2\mathbf{p}_{xv_{2}} + 2\mathbf{R}_{v_{2}} \mathbf{w} = 0,& \\
					-\mathbf{p}_{xv_{2}} + \mathbf{R}_{v_{2}} \mathbf{w} = 0,& \\
					\mathbf{R}_{v_{2}} \mathbf{w} = \mathbf{p}_{xv_{2}}.&
				\end{align}
				
				\paragraph{}Utilizando a identidade matricial abaixo é possível resolver a equação acima para obter o seguinte resultado
				
				\begin{align}
					\mathbf{R}^{-1}_{v_{2}}\mathbf{R}_{v_{2}} \mathbf{w} &= \mathbf{R}^{-1}_{v_{2}}\mathbf{p}_{xv_{2}}, \\
					\mathbf{I}\mathbf{w} &= \mathbf{R}^{-1}_{v_{2}}\mathbf{p}_{xv_{2}}, \\ 
					\mathbf{w} &= \mathbf{R}^{-1}_{v_{2}}\mathbf{p}_{xv_{2}}. 
				\end{align}
				
				\paragraph{}Onde é possível reescrever o termo final como
				
				\begin{align}
					\mathbf{w} = \mathbf{R}^{-1}_{v_{2}}(\mathbf{p}_{d} + \mathbf{p}_{v_{1}} + \mathbf{p}_{v_{2}}) 
				\end{align}
			
			\item Seja um processo estocástico dado pela expressão abaixo onde $S(n)$ é um processo estocástico WSS dado e $a$ é uma constante.
			
				\begin{align*}
					x(n) = s(n + a) + s(n-4a),
				\end{align*}
				
				Deseja-se filtrar o processo de tal forma obter-se um processo $D(s) = s(n -a)$, o qual também sabe-se que é um processo WSS. Suponha que o sinal $d(n)$ possua média nula e variância unitária.
			
				\begin{enumerate}
					
					\item Calcule o filtro, com dois coeficientes, que fornece a solução ótima em relação ao erro médio quadrático.
					
						\textcolor{red}{Solução:}

						\paragraph{}O filtro linear ótimo que minimiza o erro médio quadrático é descrito pela solução das equações de Wiener. Portanto, inicialmente definir a matriz de autocorrelação para o processo descrito por $x(n)$

						\begin{align} 
							\mathbf{R}_{x} = 
							\begin{bmatrix}
								\mathbb{E}\{x(n)x^{*}(n)\} & \mathbb{E}\{x(n-1)x^{*}(n)\} \\
								\mathbb{E}\{x(n)x^{*}(n-1)\}  & \mathbb{E}\{x(n-1)x^{*}(n-1)\} 
							\end{bmatrix},
						\end{align}
					
						se considerarmos que o processo $S(n)$ é WSS com variância $\sigma^{2}_{s}$ podemos calcular as correlações como se segue

						\begin{align*} 
							\mathbb{E}\{x(n)x^{*}(n)\} &= \mathbb{E}\{ s(n - a) s^{*}(n - a) + s(n - a) s^{*}(n - 4a) + s(n - 4a) s^{*}(n - a) \\
							&+ s(n - 4a) s^{*}(n - 4a) \} = \sigma^{2}_{s} + 0 + 0 + \sigma^{2}_{s} = 2\sigma^{2}_{s} , \\
							\mathbb{E}\{x(n-1)x^{*}(n)\} &= \mathbb{E}\{ s(n - 1 -a) s^{*}(n - a) + s(n - 1 - a) s^{*}(n - 4a) + s(n - 1 - 4a) s^{*}(n - a) \\
							&+ s(n - 1 - 4a) s^{*}(n - 4a) \} = 0 + 0 + 0 + 0 = 0, \\
							\mathbb{E}\{x(n)x^{*}(n-1)\} &= \mathbb{E}\{ s(n - a) s^{*}(n - 1 - a) + s(n - a) s^{*}(n - 1 - 4a) + s(n - 4a) s^{*}(n - 1 - a) \\
							&+ s(n - 4a) s^{*}(n - 1 - 4a) \} = 0 + 0 + 0 + 0 = 0, \\
							\mathbb{E}\{x(n-1)x^{*}(n-1)\} &= \mathbb{E}\{ s(n - 1 -a) s^{*}(n - 1 - a) + s(n - 1 - a) s^{*}(n - 1 - 4a) \\
							&+ s(n - 1 - 4a) s^{*}(n - 1 - a) + s(n - 1 - 4a) s^{*}(n - 1 - 4a) \} = \sigma^{2}_{s} + 0 + 0 + \sigma^{2}_{s} = 2\sigma^{2}_{s}, \\
						\end{align*}

						obtendo assim
						
						\begin{align} 
							\mathbf{R}_{x} = 
							\begin{bmatrix}
								2 \sigma^{2}_{s} & 0 \\
								0  & 2 \sigma^{2}_{s}
							\end{bmatrix}.
						\end{align}

						\paragraph{}Entretanto, considerando que o processo $D(n)$ têm média nula então temos na verdade um vetor de correlação cruzada nulo. Desse modo, o filtro linear
						ótimo para esse processo seria o próprio vetor nulo. Sendo assim

						\begin{align} 
							\mathbf{w}_{\text{opt}} = \mathbf{R}^{-1}_{x} \mathbf{p}_{xd} = 
							\begin{bmatrix}
								2 \sigma^{2}_{s} & 0 \\
								0  & 2 \sigma^{2}_{s}
							\end{bmatrix}
							\begin{bmatrix}
								0 \\
								0 
							\end{bmatrix} = 
							\begin{bmatrix}
								0 \\
								0 
							\end{bmatrix},  
						\end{align}
						


					\item Calcule o preditor direto ótimo de passo unitário, com dois coeficientes, que fornece a solução ótima em relação ao erro médio quadrático.
					
						\textcolor{red}{Solução:}
						
						\paragraph{}O filtro preditor direto é aquele que utiliza o conhecimento de amostras passadas para predizer amostras futuras de algum processo. Desse modo, podemos definir o filtro direto de passo unitário como

						\begin{align}
							\hat{x}(n) &= \sum^{M + k - 1}_{i = k} w_{f,i} x(n - i) = \sum^{2}_{i = 1} w_{f,i} x(n - i), \\
							\hat{x}(n) &= \mathbf{w}^{\text{T}}_{f} \mathbf{x}(n - 1),
						\end{align}

						onde o erro quadrático médio pode ser dado por

						\begin{align}
							\mathbb{E}\{e^{2}(n)\} = \mathbb{E}\{(x(n) - \hat{x}(n) )^{2}\} = \mathbf{r}_{x}(0) - 2 \mathbf{w}^{\text{T}}_{f} \mathbf{r}_{x,f} + \mathbf{w}^{\text{T}}_{f} \mathbf{R}_{x} \mathbf{w}_{f},
						\end{align}

						e, portanto, de forma similar a solução de Wiener para o filtro linear ótimo temos

						\begin{align}
							\mathbf{w}_{f,\text{opt}} = \mathbf{R}^{-1}_{x} \mathbf{r}_{x,f}.
						\end{align}

						\paragraph{}Onde mais uma vez teremos a mesma matriz de autocorrelação definida no item anterior, mas agora teremos um vetor de correlação cruzada definido por

						\begin{align}
							\mathbf{r}_{x,f} = 
							\begin{bmatrix}
								r_{x}(1) \\
								r_{x}(2)
							\end{bmatrix} =
							\begin{bmatrix}
								\mathbb{E}\{x(n) x(n - 1)\} \\
								\mathbb{E}\{x(n) x(n - 2)\}
							\end{bmatrix} = 
							\begin{bmatrix}
								0 \\
								0
							\end{bmatrix}.
						\end{align}
						
						\paragraph{}Desse modo, temos resultado idêntico ao item anterior com a solução ótima sendo o próprio vetor nulor.

					\item Compare as soluções dos dois.
					
						\textcolor{red}{Solução:} 

						\paragraph{}Os resultados são contraintuitivos e exibem limitações de ordem técnica que podemos encontrar durante a solução de certos problemas. Talvez uma
						solução para esse problema fosse de alguma forma obter as correlações reais entre amostras dos processos descritos por $X(n)$ e $D(n)$. Entretanto, não
						consigo visualizar com tanta clareza como tal procedimento poderia ser realizado.
					
				\end{enumerate}
			
			\item Suponha que foram encontrados os seguintes coeficientes de autocorrelação: $r_x(0) = 1$ e $r_x(1) = 0$. Tais coeficientes foram obtidos de amostras corrompidas com ruído. Além disso, a variância do sinal desejado é $\sigma_d^2 =
			24.40$ e o vetor de correlação cruzada é $\mathbf{p}_{\mathbf{x}d} = [2 \ \ 4.5]^T$. Encontre:
			
				\begin{enumerate}
					
					\item O valor dos coeficientes do filtro de Wiener.
					
						\textcolor{red}{Solução:}
						
						\paragraph{}A partir dos coeficientes fornecidos é possível escrever a matrix de correlação necessário para o filtro ótimo de wiener como uma matriz identidade de ordem 2
						
						\begin{align}
							\mathbf{R}_{X} = \left[ \begin{matrix} 1 & 0 \\ 0 & 1 \end{matrix} \right]
						\end{align}
						
						\paragraph{}Ao utilizar a solução fechada do problema chega-se ao seguinte vetor resultado
						
						\begin{align}
							\mathbf{w}_{\text{opt}} &= \mathbf{R}^{-1}_{x} \mathbf{p}_{xd} = \left[ \begin{matrix} 1 & 0 \\ 0 & 1 \end{matrix} \right]  \left[ \begin{matrix} 2 \\ 4.5 \end{matrix} \right] = \left[ \begin{matrix} 2 \\ 4.5 \end{matrix} \right].
						\end{align}
					
					\item A superfície definida por $J(\mathbf{w})$. Faça um gráfico da mesma.
					
						\textcolor{red}{Solução:}
						
						\paragraph{}Para obter a expressão que define a superfície basta desenvolver a expressão para o erro médio
						
						\begin{align}
							\mathbf{J}(w) &= \mathbb{E}\{e^{2}(n)\} = \sigma^{2}_{d} - 2\mathbf{w}^{T}\mathbf{p}_{xd} + w^{T}\mathbf{R}_{X}\mathbf{w}. \label{eq:mse}   
						\end{align}
						
						\paragraph{}Substituindo os valores encontrados anteriormente na expressão da superfície
						
						\begin{align}
							\mathbf{J}(w_{0}, w_{1}) &= 24.40 - 2 \left[ \begin{matrix} w_{0}  w_{1} \end{matrix} \right] \left[ \begin{matrix} 2 \\ 4.5 \end{matrix} \right] + \left[ \begin{matrix} w_{0}  w_{1} \end{matrix} \right] \left[ \begin{matrix} 1 & 0 \\ 0 & 1 \end{matrix} \right]  \left[ \begin{matrix} w_{0}  \\ w_{1} \end{matrix} \right], \\
							\mathbf{J}(w_{0},w_{1}) &= 24.40 - 4w_{0} - 9w_{1} + w^{2}_{0} + w^{2}_{1}.
						\end{align}
						
						\paragraph{}Utilizando um software gráfico é possível obter a Figura \ref{fig:01} onde é traçada a superfície de erro MSE expressa na Equação (\ref{eq:mse}).
					
						\begin{figure}[H]
							\centering
							\includegraphics[width=0.5\textwidth]{figs/superficie-de-erro.png}
							\caption{Superfície de erro MSE}
							\label{fig:01}
						\end{figure}
					
				\end{enumerate}
			
		\end{enumerate}
	
	\newpage
	\section*{Algoritmos Recursivos}
	
		\begin{enumerate}
			
			\item Deseja-se minimizar a função objetivo $\mathbb{E}\{e^{4}(n)\}$ utilizando-se um algoritmo do gradiente estocástico do tipo LMS. O algoritmo resultando é chamado de algoritmo least mean fourth (LMF). Derive tal algoritmo. Derive também o filtro ótimo para tal critério e compare as soluções.
			
				\textcolor{red}{Solução:}
				
				\paragraph{}Podemos inicialmente definir a função erro para esse filtro como
				
				\begin{align}
					e(n) &= d(n) - y(n), \\
					e(n) &= d(n) - \mathbf{w}^{\text{T}}(n)\mathbf{x}(n),
				\end{align}
				
				e para obtermos a nova expressão de recurssão precisamos primeiro obter o equacionamento para o vetor gradiente instantâneo de $\mathbb{E}\{e^{4}(n)\}$. Isso pode ser prontamente obtido por meio do auxílio de uma derivação implícita

				\begin{align}
					\nabla_{\mathbf{w}} \mathbb{E}\{e^{4}(n)\} &= \frac{\partial \mathbb{E}\{e^{4}(n)\}}{\partial \mathbf{w}} = \mathbb{E}\{ \frac{\partial e^{4}(n)}{\partial \mathbf{w}}\} = \mathbb{E}\{ \frac{\partial e^{4}(n)}{\partial e(n)} \frac{\partial e(n) }{\partial \mathbf{w}}\}, \\
					\nabla_{\mathbf{w}} \mathbb{E}\{e^{4}(n)\} &= \mathbb{E}\{4 e^{3}(n) \frac{\partial (d(n) - \mathbf{w}^{\text{T}}(n)\mathbf{x}(n)) }{\partial \mathbf{w}}\} = \mathbb{E}\{4 e^{3}(n) (0 - \mathbf{x}(n))\}, \\
					\nabla_{\mathbf{w}} \mathbb{E}\{e^{4}(n)\} &= - 4 \mathbb{E}\{e^{3}(n) \mathbf{x}(n)\}.
				\end{align}
			
				\paragraph{}Desse modo, para que a minimização de $\mathbb{E}\{e^{4}(n)\}$ seja atingida precisamos garantir apenas que o vetor $x(n)$ tenha entradas ortogonais ao vetor erro $e(n)$. Assim, é necessário que a seguinte equação seja verdade

				\begin{align}
					\mathbb{E}\{e^{3}(n) \mathbf{x}(n)\} &= 0, \\
					\mathbb{E}\{(d(n) - \mathbf{w}^{\text{T}}(n)\mathbf{x}(n))^{3} \mathbf{x}(n)\} &= 0, 
				\end{align}

				onde é possível demonstrar que existe convergência em média para essa expressão se definirmos o passo de aprendizado no seguinte intervalo\footnote{A demonstração da propriedade foi extensivamente explicada no artigo \textit{The Least Mean Square Fourth (LMF) Algorithm and Its Family} de 1984 por Eugene Walash e Bernard Widrow.}

				\begin{align}
					1 < \mu < \frac{1}{6 \sigma^{2}_{z} \lambda_{\text{max}}},
				\end{align}

				onde $\sigma^{2}_{z}$ é a variância do ruído presente e $\lambda_{\text{max}}$ é o maior autovalor da matriz de autocorrelação $\mathbf{R}_{x}$. Por fim, a partir dessas observações podemos escrever a expressão de recurssão para o LMF utilizando 
				a expressão padrão para o algoritmo do gradiente descendente\footnote{Essa expressão é brevemente introduzida no livro texto da disciplina.}

				\begin{align}
					\mathbf{w}(n + 1) &= \mathbf{w}(n) - \mu \mathbf{g}_{w}(n), \\
					\mathbf{w}(n + 1) &= \mathbf{w}(n) + 4 \mu e^{3}(n) \mathbf{x}(n),
				\end{align}

				onde sabemos que o erro é dado por $e(n) = d(n) - \mathbf{w}^{\text{T}}(n)\mathbf{x}(n)$.

			\item Considere o uso de um a sequência de ruído branco com média nula e variância $\sigma^{2}$ como entrada do algoritmo LMS. Avalie
				
				\begin{enumerate}
					
					\item a condição para convergência do algoritmo em média.
					
						\textcolor{red}{Solução:}
						
						\paragraph{}A condição de convergência está diretamente associada com o erro nos coeficientes do filtro adaptativo para cada iteração.
						Desse modo, podemos iniciar o estudo desse tópico com a seguinte expressão que relaciona o erro dos coeficientes do filtro de um iteração
						$k$ para a solução ótima
						
						\begin{align}
							\Delta \mathbf{w}(n) = \mathbf{w}(n) - \mathbf{w}_{\text{opt}},
						\end{align}

						e assim podemos reescrever a função de recurssão do LMS do seguinte modo

						\begin{align}
							\Delta \mathbf{w}(n + 1) &= \Delta \mathbf{w}(n) + 2 \mu e(n) \mathbf{x}(n), \\
							\Delta \mathbf{w}(n + 1) &= \Delta \mathbf{w}(n) + 2 \mu \mathbf{x}(n) \left[\mathbf{x}^{\text{T}}(n)\mathbf{w}_{\text{opt}} + z(n) - \mathbf{x}^{\text{T}}(n)\mathbf{w}(n)\right], \\
							\Delta \mathbf{w}(n + 1) &= \Delta \mathbf{w}(n) + 2 \mu \mathbf{x}(n) \left[e_{\text{opt}}(n) - \mathbf{x}^{\text{T}}(n) \Delta \mathbf{w}(n)\right], \\
							\Delta \mathbf{w}(n + 1) &= \left[ \mathbf{I} - 2 \mu \mathbf{x}(n) \mathbf{x}^{\text{T}}(n) \right] \Delta \mathbf{w}(n) + 2 \mu e_{\text{opt}}(n) \mathbf{x}(n)
						\end{align}
					
						onde $z(n) \in \mathcal{N}(0,\sigma^{2})$ e $e_{\text{opt}}(n) = z(n)$. Indicando que, para o caso ideal, teríamos que nos preocupar apenas com o erro introduzido pelas componentes ruidosas do sistema.
						Ademais, também utilizamos o fato de que $e(n) = e^{\text{T}}(n) = \mathbf{w}^{\text{T}}_{\text{opt}}\mathbf{x}(n) + z(n) - \mathbf{w}^{\text{T}}(n)\mathbf{x}(n) = \mathbf{x}^{\text{T}}(n)\mathbf{w}_{\text{opt}} + z(n) - \mathbf{x}^{\text{T}}(n)\mathbf{w}(n)$.
						Em sequência, podemos escrever o valor esperado para o erro como

						\begin{align}
							\mathbb{E}\{\Delta \mathbf{w}(n + 1)\} &= \mathbb{E}\{\left[ \mathbf{I} - 2 \mu \mathbf{x}(n) \mathbf{x}^{\text{T}}(n) \right] \Delta \mathbf{w}(n) + 2 \mu e_{\text{opt}}(n) \mathbf{x}(n)\}, \\
							\mathbb{E}\{\Delta \mathbf{w}(n + 1)\} &= \mathbb{E}\{\left[ \mathbf{I} - 2 \mu \mathbf{x}(n) \mathbf{x}^{\text{T}}(n) \right] \Delta \mathbf{w}(n)\} + 2 \mu \mathbb{E}\{e_{\text{opt}}(n) \mathbf{x}(n)\}
						\end{align}
						
						onde utilizamos a propriedade de linearidade do operador valor esperado. Ao assumirmos que $\mathbf{x}(n)$ é simultaneamente ortogonal aos elementos $e_{\text{opt}}(n)$ e $\Delta \mathbf{w}(n)$ podemos simplificar a expressão acima por
						
						\begin{align}
							\mathbb{E}\{\Delta \mathbf{w}(n + 1)\} &= \left[ \mathbf{I} - 2 \mu \mathbb{E}\{\mathbf{x}(n) \mathbf{x}^{\text{T}}(n)\} \right] \mathbb{E}\{\Delta \mathbf{w}(n)\}, \\
							\mathbb{E}\{\Delta \mathbf{w}(n + 1)\} &= \left( \mathbf{I} - 2 \mu \mathbf{R}_{x} \right) \mathbb{E}\{\Delta \mathbf{w}(n)\},
						\end{align}

						onde podemos utilizar a noção de que $\mathbb{E}\{ \mathbf{I} \} = \mathbf{I}$ uma vez que temos um operador linear. Em continuidade, podemos supor que existe uma matriz $\mathbf{Q}$ unitária que diagonaliza $\mathbf{R}_{x}$
						de tal modo que temos 

						\begin{align}
							\mathbb{E}\{ \mathbf{Q}^{\text{T}} \Delta \mathbf{w}(n + 1) \} &= \left( \mathbf{I} - 2 \mu \mathbf{Q}^{\text{T}} \mathbf{R}_{x} \right) \mathbf{I} \mathbb{E}\{ \Delta \mathbf{w}(n)\}, \\
							\mathbb{E}\{ \mathbf{Q}^{\text{T}} \Delta \mathbf{w}(n + 1) \} &= \left( \mathbf{I} - 2 \mu \mathbf{Q}^{\text{T}} \mathbf{R}_{x} \right) \mathbf{Q} \mathbf{Q}^{\text{T}} \mathbb{E}\{ \Delta \mathbf{w}(n)\}, \\
							\mathbb{E}\{ \mathbf{Q}^{\text{T}} \Delta \mathbf{w}(n + 1) \} &= \left( \mathbf{I} - 2 \mu \mathbf{Q}^{\text{T}} \mathbf{R}_{x} \mathbf{Q} \right) \mathbb{E}\{ \mathbf{Q}^{\text{T}} \Delta \mathbf{w}(n)\}, \\
							\mathbb{E}\{\Delta \mathbf{w}'(n + 1)\} &= \left( \mathbf{I} - 2 \mu \mathbf{\Lambda} \right) \mathbb{E}\{\Delta \mathbf{w}'(n)\}.
						\end{align}

						\paragraph{}Podemos, por fim, expandir os termos à esquerda e chegar na seguinte expressão para a análise do comportamento de convergência dos coeficientes de filtro

						\begin{align}
							\mathbb{E}\{ \Delta \mathbf{w}'(n + 1) \} &= \left( \mathbf{I} - 2 \mu \mathbf{\Lambda} \right)^{n + 1} \mathbb{E}\{\Delta \mathbf{w}'(0)\}, \\
							\mathbb{E}\{ \Delta \mathbf{w}'(n + 1) \} &= 
							\begin{bmatrix}
								(1 - 2 \mu \lambda_{1})^{n + 1} & 0 & \cdots & 0 \\
								0 & (1 - 2 \mu \lambda_{2})^{n + 1} & \cdots & \vdots \\
								\vdots & \vdots & \ddots & \vdots \\
								0 & 0 & \cdots & (1 - 2 \mu \lambda_{N})^{n + 1}
							\end{bmatrix} 
							\mathbb{E}\{ \Delta \mathbf{w}'(0)\},	
						\end{align}

						onde temos que $\lambda_{n} \forall n \in \{1, \cdots, N\}$ são os autovalores da matriz de autocorrelação. Portanto, vemos que para garantir a estabilidade da convergência é necessário 
						apenas que o passo de aprendizado do algoritmo seja definido pela seguinte inequação

						\begin{align}
							0 < \mu < \frac{1}{\lambda}_{\text{max}},
						\end{align}

						pois desse modo conseguimos garantir que os valores da diagonal irão tender a zero a medida que o número de iterações do algoritmo tende ao infinito. Ademais, é interessante ressaltar que a escolha 
						do valor de $\mu$ deve também levar em consideração o espalhamento de energia da matriz de correlação. Dessa forma, se não há grande diferença entre
						os valores númericos dos autovalores, então seria aconselhável escolher um passo de aprendizado muito menor do que aquele definido pelo limite superior da expressão obtida acima.
						
						\item o erro em excesso em média quadrática. 
					
						\textcolor{red}{Solução:}
						
						\paragraph{}O erro em excesso é normalmente ocasionado pelos termos ruidosos presentes no gradiente, impedindo que os coeficientes convirjam de forma exata para a solução ótima. 
						Podemos iniciar essa análise escrevendo a equação para o erro de estimação para um determinado instante $n$ como se segue

						\begin{align}
							e(n) &= d(n) - \mathbf{w}^{\text{T}}_{\text{opt}} \mathbf{x}(n) - \Delta \mathbf{w}^{\text{T}}(n) \mathbf{x}(n), \\
							e(n) &= e_{\text{opt}}(n) - \Delta \mathbf{w}^{\text{T}}(n) \mathbf{x}(n),
						\end{align}

						onde o erro quadrático é expresso por

						\begin{align}
							e^{2}(n) &= e^{2}_{\text{opt}}(n) - 2 e_{\text{opt}}(n) \Delta \mathbf{w}^{\text{T}}(n) \mathbf{x}(n) + \Delta \mathbf{w}^{\text{T}}(n) \mathbf{x}(n) \mathbf{x}^{\text{T}}(n) \Delta \mathbf{w}(n) ,
						\end{align}

						se definirmos o erro quadrático médio como $\xi(n) = \mathbb{E}\{e^{2}(n)\}$ e o erro ótimo, leia-se mínimo, como $\xi_{\text{min}} = \mathbb{E}\{e^{2}_{\text{opt}}(n)\}$
						então podemos escrever
						
						\begin{align}
							\xi(n) = \xi_{\text{min}} - 2 \mathbb{E}\{e_{\text{opt}}(n) \Delta \mathbf{w}^{\text{T}}(n) \mathbf{x}(n)\} + \mathbb{E}\{\Delta \mathbf{w}^{\text{T}}(n) \mathbf{x}(n) \mathbf{x}^{\text{T}}(n) \Delta \mathbf{w}(n)\},
						\end{align}

						considerando mais uma vez que $\mathbf{x}$ é ortogonal ao $\Delta \mathbf{w}^{\text{T}}(n)$ e ao $e_{\text{opt}}(n)$, simultaneamente, podemos simplificar a expressão ainda mais 

						\begin{align}
							\xi(n) = \xi_{\text{min}} - 2 \mathbb{E}\{\Delta \mathbf{w}^{\text{T}}(n)\} \mathbb{E}\{e_{\text{opt}}(n) \mathbf{x}(n)\} + \mathbb{E}\{\Delta \mathbf{w}^{\text{T}}(n) \mathbf{x}(n) \mathbf{x}^{\text{T}}(n) \Delta \mathbf{w}(n)\},
						\end{align}

						onde podemos utilizar a propriedade $\mathbb{E}\{\Delta \mathbf{w}^{\text{T}}(n) \mathbf{x}(n) \mathbf{x}^{\text{T}}(n) \Delta \mathbf{w}(n)\} = \text{tr}(\mathbb{E}\{\Delta \mathbf{w}^{\text{T}}(n) \mathbf{x}(n) \mathbf{x}^{\text{T}}(n) \Delta \mathbf{w}(n)\})$ uma vez
						que sabemos que o traço de um escalar é o próprio escalar. Sendo assim, utilizando a propriedade cíclica do operador traço escrevemos

						\begin{align}
							\xi(n) &= \xi_{\text{min}} - 2 \mathbb{E}\{\Delta \mathbf{w}^{\text{T}}(n)\} \mathbb{E}\{e_{\text{opt}}(n) \mathbf{x}(n)\} + \text{tr}(\mathbb{E}\{\Delta \mathbf{w}^{\text{T}}(n) \mathbf{x}(n) \mathbf{x}^{\text{T}}(n) \Delta \mathbf{w}(n)\}), \\
							\xi(n) &= \xi_{\text{min}} - 2 \mathbb{E}\{\Delta \mathbf{w}^{\text{T}}(n)\} \mathbb{E}\{e_{\text{opt}}(n)\mathbf{x}(n)\} + \mathbb{E}\{\text{tr}[\Delta \mathbf{w}^{\text{T}}(n) \mathbf{x}(n) \mathbf{x}^{\text{T}}(n) \Delta \mathbf{w}(n)]\}, \\
							\xi(n) &= \xi_{\text{min}} - 2 \mathbb{E}\{\Delta \mathbf{w}^{\text{T}}(n)\} \mathbb{E}\{e_{\text{opt}}(n)\mathbf{x}(n)\} + \mathbb{E}\{\text{tr}[\mathbf{x}(n) \mathbf{x}^{\text{T}}(n) \Delta \mathbf{w}(n) \Delta \mathbf{w}^{\text{T}}(n)]\}, \\
							\xi(n) &= \xi_{\text{min}} + \mathbb{E}\{\text{tr}[\mathbf{x}(n) \mathbf{x}^{\text{T}}(n) \Delta \mathbf{w}(n) \Delta \mathbf{w}^{\text{T}}(n)]\}, \\
							\xi(n) &= \xi_{\text{min}} + \text{tr}(\mathbb{E}\{\mathbf{x}(n) \mathbf{x}^{\text{T}}(n) \Delta \mathbf{w}(n) \Delta \mathbf{w}^{\text{T}}(n)\}), \\
							\xi(n) &= \xi_{\text{min}} + \text{tr}(\mathbb{E}\{\mathbf{x}(n) \mathbf{x}^{\text{T}}(n)\} \mathbb{E}\{\Delta \mathbf{w}(n) \Delta \mathbf{w}^{\text{T}}(n)\}), \\
							\xi(n) &= \xi_{\text{min}} + \text{tr}(\mathbf{R}_{x} \mathbb{E}\{\Delta \mathbf{w}(n) \Delta \mathbf{w}^{\text{T}}(n)\}), \\
							\xi(n) &= \xi_{\text{min}} + \text{tr}(\mathbb{E}\{\mathbf{R}_{x} \Delta \mathbf{w}(n) \Delta \mathbf{w}^{\text{T}}(n)\}), 
						\end{align}

						onde o processo justifica-se pela intercambialidade entre os operadores traço e valor esperador e pela ortogonalidade mencionada anteriormente. Em sequência podemos definir o erro em excesso por

						\begin{align}
							\xi(n) - \xi_{\text{min}} &= \text{tr}(\mathbb{E}\{\mathbf{R}_{x} \Delta \mathbf{w}(n) \Delta \mathbf{w}^{\text{T}}(n)\}), \\
							\Delta \xi(n) &= \text{tr}(\mathbb{E}\{\mathbf{R}_{x} \Delta \mathbf{w}(n) \Delta \mathbf{w}^{\text{T}}(n)\}), \label{excesso1}
						\end{align}

						e adicionalmente, pela definição de uma transformação de similaridade onde garantimos que exista $\mathbf{Q} \mathbf{Q}^{\text{T}} = \mathbf{I}$ capaz de diagonalizar a matriz de autocorrelação, podemos reescrever a Equação (\ref{excesso1}) da seguinte forma

						\begin{align}
							\Delta \xi(n) &= \text{tr}(\mathbb{E}\{\mathbf{Q} \mathbf{Q}^{\text{T}} \mathbf{R}_{x} \mathbf{Q} \mathbf{Q}^{\text{T}} \Delta \mathbf{w}(n) \Delta \mathbf{w}^{\text{T}}(n) \mathbf{Q} \mathbf{Q}^{\text{T}}\}), \\
							\Delta \xi(n) &= \text{tr}(\mathbb{E}\{\mathbf{Q} \mathbf{\Lambda} \mathbf{Q}^{\text{T}} \Delta \mathbf{w}(n) \Delta \mathbf{w}^{\text{T}}(n) \mathbf{Q} \mathbf{Q}^{\text{T}}\}), \\
							\Delta \xi(n) &= \text{tr}(\mathbb{E}\{\mathbf{Q} \mathbf{\Lambda} \text{cov} \left[\Delta \mathbf{w}'(n)\right] \mathbf{Q}^{\text{T}}\}), \\
							\Delta \xi(n) &= \text{tr}(\mathbb{E}\{\mathbf{\Lambda} \text{cov} \left[\Delta \mathbf{w}'(n)\right] \mathbf{Q}^{\text{T}} \mathbf{Q}\}), \\
							\Delta \xi(n) &= \text{tr}(\mathbb{E}\{\mathbf{\Lambda} \text{cov} \left[\Delta \mathbf{w}'(n)\right]\}), \label{excesso2}
						\end{align}

						onde é sabido que $\text{cov} \left[\Delta \mathbf{w}(n)\right] = \mathbb{E}\{(\mathbf{w}(n) - \mathbf{w}_{\text{opt}}) (\mathbf{w}(n) - \mathbf{w}_{\text{opt}})^{\text{T}}\}$ e $\Delta \mathbf{w}'(n) = \mathbf{Q}^{\text{T}} \Delta \mathbf{w}(n) \Delta \mathbf{w}^{\text{T}}(n) \mathbf{Q}$.
						Enfim, a partir das orientações presentes no livro texto da disciplina a respeito da matriz de covariância do vetor de erros é possível ainda escrever a Equação (\ref{excesso2}) como

						\begin{align}
							\Delta \xi(n) &= \sum^{N}_{i = 1} \lambda_{i} v_{i}'(n) = \mathbf{\lambda}^{\text{T}} \mathbf{v}'(n),
						\end{align}

						onde $\mathbf{\lambda}$ é um vetor contendo todos os autovalores da matriz de autocorrelação e $\mathbf{v}(n)$ é um vetor que contém os elementos da diagonal de $\text{cov} \left[\Delta \mathbf{w}'(n)\right]$. Em suma, podemos 
						expressar o $i$th elemento do vetor diagonal $\mathbf{v}'(n + 1)$, seguindo orientações disponíveis no livro texto da disciplina, com o seguinte equacionamento

						\begin{align}
							v_{i}'(n + 1) &= (1 - 4 \mu \lambda_{i} + 8 \mu^{2} \lambda^{2}_{i}) v_{i}'(n) + 4 \mu^{2} \lambda_{i} \sum^{N}_{j = 0} \lambda_{j} v_{j}'(n) + 4 \mu^{2} \sigma^{2}_{z} \lambda_{i}.
						\end{align}

						\paragraph{}Entretanto, se considerarmos que $v_{i}'(n + 1) \approx v_{i}'(n)$ quando $n \rightarrow \infty$ podemos simplificar a expressão, além de que 
						também podemos realizar uma operação de soma total dos parâmetros para obter o erro em excesso total

						\begin{align}
							\notag &\sum^{N}_{i = 1} v_{i}'(n) = \sum^{N}_{i = 1} (1 - 4 \mu \lambda_{i} + 8 \mu^{2} \lambda^{2}_{i}) v_{i}'(n) + 4 \mu^{2} \sum^{N}_{i = 1} \lambda_{i} \sum^{N}_{j = 0} \lambda_{j} v_{j}'(n) + 4 \mu^{2} \sigma^{2}_{z} \sum^{N}_{i = 1} \lambda_{i}, \\
							\notag &\sum^{N}_{i = 1} v_{i}'(n) = \sum^{N}_{i = 1} v_{i}'(n) - 4 \mu \sum^{N}_{i = 1} \lambda_{i} v_{i}'(n)  + 8 \mu^{2} \sum^{N}_{i = 1} \lambda^{2}_{i} v_{i}'(n) + 4 \mu^{2} \sum^{N}_{i = 1} \lambda_{i} \sum^{N}_{j = 0} \lambda_{j} v_{j}'(n) + 4 \mu^{2} \sigma^{2}_{z} \sum^{N}_{i = 1} \lambda_{i}, \\
							\notag &\sum^{N}_{i = 1} v_{i}'(n) - \sum^{N}_{i = 1} v_{i}'(n) + 4 \mu \sum^{N}_{i = 1} \lambda_{i} v_{i}'(n) - 4 \mu^{2} \sum^{N}_{i = 1} \lambda_{i} \sum^{N}_{j = 0} \lambda_{j} v_{j}'(n) = 4 \mu^{2} \sigma^{2}_{z} \sum^{N}_{i = 1} \lambda_{i} + 8 \mu^{2} \sum^{N}_{i = 1} \lambda^{2}_{i} v_{i}'(n), \\
							\notag &4 \mu \sum^{N}_{j = 1} \lambda_{j} v_{j}'(n)(1 - \mu \sum^{N}_{i = 1} \lambda_{i}) = 4 \mu (\mu \sigma^{2}_{z} \sum^{N}_{i = 1} \lambda_{i} + 2 \mu \sum^{N}_{i = 1} \lambda^{2}_{i} v_{i}'(n)), \\
							\notag &\sum^{N}_{j = 1} \lambda_{j} v_{j}'(n) = \frac{\mu \sigma^{2}_{z} \sum^{N}_{i = 1} \lambda_{i} + 2 \mu \sum^{N}_{i = 1} \lambda^{2}_{i} v_{i}'(n)}{1 - \mu \sum^{N}_{i = 1} \lambda_{i}}, \\
							\notag &\sum^{N}_{j = 1} \lambda_{j} v_{j}'(n) \approx \frac{\mu \sigma^{2}_{z} \sum^{N}_{i = 1} \lambda_{i} }{1 - \mu \sum^{N}_{i = 1} \lambda_{i}}, \\
							&\sum^{N}_{j = 1} \lambda_{j} v_{j}'(n) = \frac{\mu \sigma^{2}_{z} \text{tr}(\mathbf{R}_{x})}{1 - \mu \text{tr}(\mathbf{R}_{x})},
						\end{align}

						onde foi considerado que o termo $2 \mu \sum^{N}_{i = 1} \lambda^{2}_{i} v_{i}'(n)$ apresenta contribuição insignificante para o valor absoluto do numerador. Entretanto, é mencionado no livro texto que tal aproximação é de prova complexa, mas que normalmente 
						se verifica verdadeira para pequenos valores do passo de aprendizado $\mu$. Portanto, o erro em excesso pode ser prontamente descrito pela expressão que se segue

						\begin{align}
							\xi_{\text{excesso}} &= \underset{n \rightarrow \infty}{\text{lim}} \Delta \xi(n) \approx \frac{\mu \sigma^{2}_{z} \text{tr}(\mathbf{R}_{x})}{1 - \mu \text{tr}(\mathbf{R}_{x})}.
						\end{align}

						\paragraph{}Vale ainda expor que podemos considerar $1 - \mu \text{tr}(\mathbf{R}_{x}) \approx 1$ para valores muito pequenos de $\mu$, obtendo assim uma versão aproximada para o erro em excesso dada por
						
						\begin{align}
							\xi_{\text{excesso}} &= \underset{n \rightarrow \infty}{\text{lim}} \Delta \xi(n) \approx \mu \sigma^{2}_{z} \text{tr}(\mathbf{R}_{x}).
						\end{align}

				\end{enumerate}

			\item Avalie a questão anterior para o caso do algoritmo LMS-Normalizado. Compare os dois casos.
				
			\textcolor{red}{Solução:}

			\paragraph{}Em acordância com o livro texto da disciplina a expressão de atualização dos coeficientes de filtro para o algoritmo NLMS é dada por

			\begin{align}
				\mathbf{w}(k+1) = \mathbf{w}(k) + \frac{\mu_{norm}}{\gamma + \mathbf{x}^{\text{T}}(k) \mathbf{x}(k)} \mathbf{e}(k) \mathbf{x}(k),
			\end{align}

			e considerando que o valor médio do passo de aprendizado aplicado na direção LMS $2 \mathbf{e}(k) \mathbf{x}(k)$ é descrito por $\frac{\mu_{norm}}{2 \text{tr}(\mathbf{R_{xx}})}$, então é possível chegar
			no seguinte limite superior para o valor de convergência se compararmos diretamente as fórmulas de atualização do LMS padrão com o LMS normalizado

			\begin{align}
				0 &< \frac{\mu_{norm}}{2 \text{tr}(\mathbf{R_{xx}})} < \frac{1}{\text{tr}(\mathbf{R_{xx}})}, \\
				0 &< \mu_{norm} < 2, 
			\end{align}
						
			\item Considere um sinal branco gaussiano de variância unitária transmitido por um canal de comunicação de função de transferência $H(z) = 1 + 1.6z^{-1}$. Para compensar este
			canal utiliza-se um equalizador dado por $W(z) = w_{0} + w_{1}z^{-1}$ .
			
			
				\begin{enumerate}
					
					\item Forneça o equalizador ótimo segundo o critério de Wiener. Esboce a posição dos zeros do canal e do equalizador no plano Z.
					
						\textcolor{red}{Solução:}
						
						\paragraph{}Considerando um sinal gaussiano branco $x(n)$ a saída do canal pode ser prontamente obtida por
						
						\begin{align}
							y(n) = x(n) + 1.6 x(n - 1),
						\end{align}
						
						e a matriz de autocorrelação será então dada por
						
						\begin{align}
							\mathbf{R}_{y} =
							\begin{bmatrix}
								\mathbb{E}\{y(n)y^{\text{H}}(n)\} & \mathbb{E}\{y(n)y^{\text{H}}(n - 1)\} \\
								\mathbb{E}\{y(n - 1)y^{\text{H}}(n)\} & \mathbb{E}\{y(n - 1)y^{\text{H}}(n - 1)\}
							\end{bmatrix},
						\end{align}
						
						onde podemos calcular os valores teóricos para as correlações da seguinte forma se assumirmos que existe independência entre amostras distintas e que o sinal é média nula 
						
						\begin{align*}
							\mathbb{E}\{y(n)y^{\text{H}}(n)\} &= \mathbb{E}\{ \mathbf{x}^{2}(n) + 1.6 \mathbf{x}(n) \mathbf{x}^{\text{H}}(n - 1) + 1.6 \mathbf{x}(n - 1) \mathbf{x}^{\text{H}} (n) \\
							&+ 2.56 \mathbf{x}^{2}(n - 1) \} = 3.56, \\
							\mathbb{E}\{y(n)y^{\text{H}}(n - 1)\} &= \mathbb{E}\{ \mathbf{x}(n) \mathbf{x}^{\text{H}}(n - 1) + 1.6 \mathbf{x}(n) \mathbf{x}^{\text{H}}(n - 2) + 1.6 \mathbf{x}(n - 1) \mathbf{x}^{\text{H}} (n - 1) \\ 
							&+ 2.56 \mathbf{x}(n - 1) \mathbf{x}^{\text{H}}(n - 2) \} = 1.60, \\
							\mathbb{E}\{y(n - 1)y^{\text{H}}(n)\} &= \mathbb{E}\{ \mathbf{x}(n - 1) \mathbf{x}^{\text{H}}(n) + 1.6 \mathbf{x}(n - 1) \mathbf{x}^{\text{H}}(n - 1) + 1.6 \mathbf{x}(n - 2) \mathbf{x}^{\text{H}} (n) \\ 
							&+ 2.56 \mathbf{x}(n - 2) \mathbf{x}^{\text{H}}(n - 1) \} = 1.60, \\
							\mathbb{E}\{y(n - 1)y^{\text{H}}(n - 1)\} &= \mathbb{E}\{ \mathbf{x}^{2}(n - 1) + 1.6 \mathbf{x}(n - 1) \mathbf{x}^{\text{H}}(n - 2) + 1.6 \mathbf{x}(n - 2) \mathbf{x}^{\text{H}} (n - 1) \\ 
							&+ 2.56 \mathbf{x}(n - 2)^{2} \} = 3.56, \\
						\end{align*}
						
						podendo assim descrever a matriz de autocorrelação teórica e sua inversa como
						
						\begin{align}
							\mathbf{R}_{y} =
							\begin{bmatrix}
								3.56 & 1.60 \\
								1.60 & 3.56
							\end{bmatrix},
						\end{align}
						
						\begin{align}
							\mathbf{R}^{-1}_{y} = \frac{1}{3.56^{2} - 1.6^{2}}
							\begin{bmatrix}
								3.56 &  -1.60 \\
								-1.60 & 3.56
							\end{bmatrix} =
							\begin{bmatrix}
								0.35 &  -0.16 \\
								-0.16 & 0.35
							\end{bmatrix}.
						\end{align}
						
						\paragraph{}Já o vetor de correlação cruzada teórico pode ser descrito por
						
						\begin{align}
							\mathbf{p}_{yd} =
							\begin{bmatrix}
								\mathbb{E}\{y(n)d(n)\} \\
								\mathbb{E}\{y(n - 1)d(n)\}
							\end{bmatrix} = 
							\begin{bmatrix}
								1 \\
								0
							\end{bmatrix},
						\end{align}
						
						pois queremos que o sinal de saída tenha a maior correlação possivel com o sinal desejado de um mesmo instante mas continue sendo independente de um sinal de um instante temporal diferente.
						Desse modo, podemos obter o equalizador ótimo segundo o critério de Wiener como
						
						\begin{align}
							\mathbf{w}_{\text{opt}} = \mathbf{R}^{-1}_{y} \mathbf{p}_{yd} = \begin{bmatrix}
								0.35 \\
								-0.16
							\end{bmatrix}.
						\end{align}
						
						\paragraph{}Por fim, abaixo segue o traçado para os zeros das funções de transferência tanto do canal quanto do filtro ótimo em azul e em vermelho, respectivamente.
						
						\begin{figure}[H]
							\centering
							\includegraphics[width=0.5\textwidth]{figs/plano_z.png}
							\caption{Zeros do canal e do equalizador no plano-z.}
						\end{figure}
					
					\item Obtenha o filtro de erro de predição direta de passo unitário, correspondente ao sinal à saída do canal. Calcule os zeros deste filtro e compare com os do equalizador.
					
						\textcolor{red}{Solução:}

						\paragraph{}De forma semelhante ao abordado anteriormente temos que o filtro de predição direta de passo unitário é dado por
						
						\begin{align}
							\hat{x}(n) &= \sum^{M + k - 1}_{i = k} w_{f,i} x(n - i) = \sum^{2}_{i = 1} w_{f,i} x(n - i) = \mathbf{w}^{\text{T}}_{f} \mathbf{x}(n - 1),
						\end{align}

						onde o erro quadrático médio é dado por

						\begin{align}
							\mathbb{E}\{e^{2}(n)\} = \mathbb{E}\{(x(n) - \hat{x}(n) )^{2}\} = \mathbf{r}_{x}(0) - 2 \mathbf{w}^{\text{T}}_{f} \mathbf{r}_{x,f} + \mathbf{w}^{\text{T}}_{f} \mathbf{R}_{x} \mathbf{w}_{f},
						\end{align}

						onde a solução ótima será dada por

						\begin{align}
							\mathbf{w}_{f,\text{opt}} = \mathbf{R}^{-1}_{x} \mathbf{r}_{x,f}.
						\end{align}

						\paragraph{}Teremos assim que a mesma matriz de autocorrelação e o vetor de correlação cruzada serão definidos por

						\begin{align}
							\mathbf{R}_{y} &=
							\begin{bmatrix}
								3.56 & 1.60 \\
								1.60 & 3.56
							\end{bmatrix}, \\
							\mathbf{r}_{y,f} &= 
							\begin{bmatrix}
								r_{y}(1) \\
								r_{y}(2)
							\end{bmatrix} =
							\begin{bmatrix}
								\mathbb{E}\{y(n) y(n - 1)\} \\
								\mathbb{E}\{y(n) y(n - 2)\}
							\end{bmatrix} = 
							\begin{bmatrix}
								1.60 \\
								0
							\end{bmatrix}.
						\end{align}

						\paragraph{}Por essa razão temos

						\begin{align}
							\mathbf{w}_{f,\text{opt}} = 
							\begin{bmatrix}
								0.35 & -0.16 \\
								-0.16 & 0.35
							\end{bmatrix}
							\begin{bmatrix}
								1.60 \\
								0
							\end{bmatrix} =
							\begin{bmatrix}
								0.56 \\
								-0.26
							\end{bmatrix}.
						\end{align}

						\paragraph{}Em seguida, podemos obter os zeros do filtro como

						\begin{align}
							W(z) &= 0.56 - 0.26 z^{-1}, \\
							0 &= 0.56 - 0.26 z^{-1}, \\
							z &= 0.45,
						\end{align}

						que é o mesmo zero do equalizador definido anteriormente.

					\newpage
					\item Obtenha as trajetórias sobre as curvas de nível, tendo condições iniciais nulas para os coeficientes do equalizador, para os seguintes algoritmos: (a) Algoritmo de Newton, (b) Gradiente Determinístico, (c) Least Means Square e (d) Least Means Square Normalizado
						
						\textcolor{red}{Solução:}
						
						\paragraph{}Antes de tudo é necessário definir a superfície de erro que servirá como referência para traçar as curvas de nível. Desse modo, podemos prontamente
						definir a superfície MSE como
						
						\begin{align}
							\mathbf{J}(w) &= \mathbb{E}\{e^{2}(n)\}, \\
							\mathbf{J}(w) &= \sigma^{2}_{d} - 2\mathbf{w}^{T}\mathbf{p}_{xd} + w^{T}\mathbf{R}_{X}\mathbf{w}. \label{mseopt}  
						\end{align}
						
						\paragraph{}Desse modo, nas Figuras \ref{fig:newton_contour}, \ref{fig:gradient_contour}, \ref{fig:lms_contour} e \ref{fig:nlms_contour} temos as curvas de convergência do algoritmo sobre a superficie MSE descrita na Equação (\ref{mseopt}). 
						Como é possível visualizar para os métodos determinísticos a convergência ocorre de forma muito mais organizada e suave. Esse resultado é previsível visto que a convergência é guiada 
						pelo conhecimento dos coeficientes ideais do filtro. Ademais, ainda é possível confirmar que o algoritmo do Gradiante Descendente de fato possue maior latência para convergir até o ponto 
						ótimo quando comparado ao Método de Newton. Entretanto, apesar de convergir mais rapidamente o Método de Newton calcula diretamente uma inversão de matriz o que é computacionalmente custoso.
						
						\paragraph{}Já para os métodos estocásticos é possível ver uma certa desorientação na convergência dos coeficientes do filtro. Isso se deve pois os coeficientes são calculados com base numa aproximação das 
						estatísticas instantâneas do sinal. Entre os dois algoritmos é possivel visualizar que o LMS apresenta um comportamento de convergência mais estável, enquanto é possível notar alguns outliers no NLMS que são resultados das 
						iterações iniciais do filtro.

					\item Obtenha também a evolução do erro quadrático médio para cada um dos algoritmos anteriores.
						
						\textcolor{red}{Solução:}

						\paragraph{}A evolução do erro quadrático médio está disponível nas Figuras \ref{fig:newton_mse}, \ref{fig:gradient_mse}, \ref{fig:lms_mse} e \ref{fig:nlms_mse} para cada um dos algoritmos requeridos.
						Em relação à evolução do MSE todos os algoritmos apresentaram desempenhos similares, muito embora o NLMS tenha apresentado um desempenho levemente superior aos outros três. Existem algumas explicações que podem
						ser alencadas para explicar tal comportamento dos métodos aplicados. Entre os mais evidentes é a presença do ruído no sistema, pois foi utilizado uma SNR de 30dB na transmissão do sinal e isso pode acarretar em 
						falhas durante o processo de filtragem. Ademais, também posso citar a ordem utilizada para os filtros que foi definida como $M = 2$ e assim o algoritmo utiliza poucas informações para atualizar os seus coeficientes a cada iteração. 
						
						\paragraph{}Por fim, também é de suma importância citar os efeitos do canal para o processo de filtragem. O canal utilizado tem uma resposta em frequência com memória e isso prejudica o processo de filtragem. Isso pode ser visto com mais clareza
						na Figura \ref{fig:nlms_awgn} onde tracei novamente a evolução do MSE para o NLMS mas agora considerando um cenário onde a resposta em frequência do canal é unitária e a SNR de transmissão se mantém em 30 dB. Nessa nova figura 
						podemos visualizar um comportamento mais eficiente do filtro e assim confirmar o impacto que a transmissão pelo canal sugerido tem para os algoritmos de filtragem implementados. Dessa forma, é necessário analisar se os algoritmos aqui 
						implementados de fato conseguem acompanhar a evolução do sistema desfazendo as alterações inseridas pelos mais diversos fenômenos. Uma forma interessante de analisar o desempenho da filtragem, e nao apenas a convergência dos coeficientes 
						do filtro, seria a análise da taxa de erro símbolo (SER) ou taxa erro bit (BER). Entretanto, isso foge ao escopo do que foi pedido nesse problema.
					
					\item Qual o número de condicionamento para o problema em questão?
					
						\textcolor{red}{Solução:}
						
						\paragraph{}O número de condicionamento pode ser prontamente obtido pela expressão

						\begin{align}
							\mathbb{C} (\mathbf{R}_{x}) = \frac{\lambda_{\text{max}}}{\lambda_{\text{min}}},
						\end{align}
					
						onde $\lambda_{\text{max}}$ e $\lambda_{\text{min}}$ são os autovalores máximo e mínimo da matriz de autocorrelação, respectivamente. Por meio de um software
						matemático foi possível obter os seguinte autovalores para a matriz de autocorrelação teórica

						\begin{align}
							\mathbb{C} (\mathbf{R}_{x}) = \frac{5.16}{1.96} = 2.63,
						\end{align}

						onde talvez seja importante ressaltar que também poderiamos ter obtido os autovalores resolvendo a equação do polinômio
						característico da matriz de autocorrelação que é dada por

						\begin{align}
							\lambda^{2} - 7.12 \lambda + 10.11 = 0. 
						\end{align}

					\item Qual deveria ser o canal para que o número de condicionamento fosse menor/maior que 5?
					Comente os resultados.
					
						\textcolor{red}{Solução:}
						
						\paragraph{}Inicialmente podemos escrever a matriz de correlação contabilizando a contribuição dos coeficientes do canal para os elementos individuais
						
						\begin{align}
							\mathbf{R}_{y} =
							\begin{bmatrix}
								a_{0} + a^{2}_{1} & a_{1}\\
								a_{1} & a_{0} + a^{2}_{1}
							\end{bmatrix},
						\end{align}

						onde a função de transferência do canal seria dada por $H(z) = a_{0} + a_{1}z^{-1}$. A partir dessa matriz de autocorrelação genérica podemos então 
						definir o seguinte polinômio característico

						\begin{align}
							&(\lambda - a_{0} + a^{2}_{1})^{2} - a^{2}_{1} = 0, \\
							&\lambda^{2} \underbrace{- 2 (a_{0} + a^{2}_{1})}_{b} \lambda + \underbrace{(a_{0} + a^{2}_{1})^{2} - a^{2}_{1}}_{c} = 0, \\
							&\lambda^{2} + b \lambda + c = 0,
						\end{align}

						onde sabemos que a solução é facilmente obtida pela fórmula de Bháskara. A partir disso podemos definir o número de condicionamento como

						\begin{align}
							\mathbb{C} (\mathbf{R}_{x}) &= \frac{\lambda_{\text{max}}}{\lambda_{\text{min}}}, \\
							\mathbb{C} (\mathbf{R}_{x}) &= \frac{- b + \sqrt{b^{2} - 4c}}{- b - \sqrt{b^{2} - 4c}}, \\
							\mathbb{C} (\mathbf{R}_{x}) &= \frac{2 (a_{0} + a^{2}_{1}) + \sqrt{4 (a_{0} + a^{2}_{1})^{2} - 4 (a_{0} + a^{2}_{1})^{2} + 4 a^{2}_{1}}}{2 (a_{0} + a^{2}_{1}) - \sqrt{4 (a_{0} + a^{2}_{1})^{2} - 4 (a_{0} + a^{2}_{1})^{2} + 4 a^{2}_{1}}}, \\
							\mathbb{C} (\mathbf{R}_{x}) &= \frac{2 (a_{0} + a^{2}_{1}) + 2a_{1}}{2 (a_{0} + a^{2}_{1}) - 2a_{1}}, \\
							\mathbb{C} (\mathbf{R}_{x}) &= \frac{a_{0} + a^{2}_{1} + a_{1}}{a_{0} + a^{2}_{1} - a_{1}},
						\end{align}

						assim temos agora uma fórmula para o número de condicionamento da matriz de autocorrelação com base nos coeficientes de canal. A partir disso basta que as seguintes inequações sejam atendidas para que
						obtenhamos um número de condicionamento maior ou menor do que o requerido

						\begin{align}
							a_{0} + a^{2}_{1} + a_{1} &\geq 5 (a_{0} + a^{2}_{1} - a_{1}), \\
							a_{0} + a^{2}_{1} + a_{1} &\leq 5 (a_{0} + a^{2}_{1} - a_{1}).
						\end{align}

				\end{enumerate}
			
			\item Utilize o algoritmo LMS para identificar um sistema com a função de transferência dada abaixo.
			
				\begin{align}
					H(z) = \frac{1 - z^{-12}}{1 - z^{-1}}
				\end{align}
			
				\begin{enumerate}
					
					\item Calcule o limite superior para $\mu$ (ou seja $\mu_{\text{max}}$) para garantir a estabilidade do algoritmo.
					
						\textcolor{red}{Solução:}
					
						\paragraph{}Para garantirmos a estabilidade do algoritmo precisamos apenas obter o valor numérico do maior autovalor definido pela matriz de autocorrelação do problema.
						Desse modo, podemos simplificar a função de transferência por meio da seguinte manipulação algébrica

						\begin{align*}
							H(z) &= \frac{1 - z^{-12}}{1 - z^{-1}}, \\
							H(z) &= \frac{(1 - z^{-1})(1 + z^{-1} + z^{-2} + z^{-3} + z^{-4} + z^{-5} + z^{-6} + z^{-7} + z^{-8} + z^{-9} + z^{-10} + z^{-11})}{1 - z^{-1}}, \\
							H(z) &= 1 + z^{-1} + z^{-2} + z^{-3} + z^{-4} + z^{-5} + z^{-6} + z^{-7} + z^{-8} + z^{-9} + z^{-10} + z^{-11},
						\end{align*}

						e tomando a transformada z inversa da função de transferência chega-se a seguinte saída de um sinal transmitido por esse canal

						\begin{align*}
							y(n) &= x(n) + x(n-1) + x(n-2) + x(n-3) + x(n-4) + x(n-5) + x(n-6) + x(n-7) \\
							&+ x(n-8) + x(n-9) + x(n-10) - x(n-11), 
						\end{align*}

						\paragraph{}Em sequência é possível utilizar um software matemático para obter uma estimação para a matriz de autocorrelação

						\begin{figure}[H]
							\centering
							\includegraphics[width=1\textwidth]{figs/Rxx.png}
						\caption{Matriz de autocorrelação estimada após 10000 realizações para retirada do comportamento médio}
						\label{fig:rxx}
						\end{figure}

						onde a análise de autovalores de $\mathbf{R}_{xx}$ resulta no seguinte intervalo de convergência para o passo de 
						aprendizado

						\begin{align}
							0 < \mu < \frac{1}{\lambda_{\text{max}}} = \frac{1}{97} \approx 0.01,
						\end{align}

						e assim $\mu_{\text{max}} \approx 0.01$.

					\item Execute o algoritmo para $\frac{\mu_{\text{max}}}{2}$, $\frac{\mu_{\text{max}}}{10}$ e $\frac{\mu_{\text{max}}}{50}$. Comente sobre o comportamento da convergência de cada caso.
						
						\textcolor{red}{Solução:}

						\paragraph{}Nas Figuras \ref{fig:mu_2}, \ref{fig:mu_10} e \ref{fig:mu_50} podemos verificar o comportamento da convergência do algoritmo para esse problema. A príncipio é possível 
						confirmar que existe uma diminuta piora quanto ao desempenho dos algoritmos a medida que o passo de aprendizado descrece. Isso pode ser explicado pois a medida que 
						$\mu$ fica menor a flexibilidade de adaptação do algoritmo é reduzida. Assim, para passos de aprendizado muito pequenos é mais difícil para o filtro conseguir acompanhar
						as mudanças no canal provocadas pelo impacto de componentes ruidosas e pela resposta em frequência do canal. 

					\item Meça o desajuste (misadjustment) em cada exemplo e comparar com os resultados obtidos pela solução teórica (Eq. (3.50) do livro texto)					

						\textcolor{red}{Solução:}

						\paragraph{}O desajuste pode ser aproximado por

						\begin{align}
							M = \frac{\xi_{\text{excesso}}}{\xi_{\text{min}}} &\approx \frac{\mu \text{tr}(\mathbf{R}_{x})}{1 - \mu \text{tr}(\mathbf{R}_{x})},
						\end{align}

						e a partir dessa expressão foi possivel obter a seguinte tabela

						\begin{table}[H]
							\centering
							\begin{tabular}{|l|l|l|}
								\hline
							 	& Empírico & Téorico \\ \hline
							 	$\frac{\mu_{\text{max}}}{2}$  & $ -1.3865 $ &  $ -1.3846 $ \\ \hline
							 	$\frac{\mu_{\text{max}}}{10}$ & $ +2.5392 $ & $ +2.5714 $ \\ \hline
							 	$\frac{\mu_{\text{max}}}{50}$ & $ +0.1675 $ & $ +0.1682 $ \\ \hline
							\end{tabular}
						\end{table}

						\paragraph{}Os resultados foram obtidos por uso de software matemático e os códigos estão disponíveis juntamente com este relatório.

					\item Mostre o gráfico da resposta em frequência do filtro FIR em qualquer uma das iterações após a convergência ser obtida e compare com o sistema desconhecido.
					
						\textcolor{red}{Solução:}

						\paragraph{}A resposta em frequência do filtro está disponível nas Figuras \ref{fig:filter_response_2}, \ref{fig:filter_response_10} e \ref{fig:filter_response_50} para os casos $\frac{\mu_{\text{max}}}{2}$, $\frac{\mu_{\text{max}}}{10}$ e $\frac{\mu_{\text{max}}}{50}$, respectivamente. 
						É possível ver que a resposta em frequência do filtro tende a se aproximar da resposta em frequência do sistema quanto maior o passo de aprendizado, pois quanto maior $\mu$ mais facilidade tem o filtro em acompanhar as variações do canal. 
						No caso extremo onde temos um passo de aprendizado $\frac{\mu_{\text{max}}}{50}$ é possível verificar que a resposta em frequência do filtro é uma versão consideravelmente amortecida da resposta original do canal. 
						Ademais, na Figura \ref{fig:L3Q5_t} vemos a evolução temporal do filtro para as primeiras 100 amostras quando temos a resposta em frequência dada na Figura \ref{fig:filter_response_2}.

				\end{enumerate}
			
			\item Seja o canal de comunicações dado por
			
				\begin{align}
					H(z) = 0.5 + 1.2z^{-1} + 1.5z^{-2} + z^{-3},
				\end{align}
				
				e deseja-se projetar um equalizar para o mesmo. A estrutura do equalizador é mostrada na Figura abaixo. Os símbolos $s(n)$ são transmitidos através de um canal e corrompidos por ruído aditivo gaussiano branco complexo $v(n)$. O sinal recebido $x(n)$ é processado pelo equalizador FIR para gerar estimativas $\overset{\sim}{s}(n - \delta)$, as quais são passados por um dispositivo decisor gerando  símbolos $\hat{s}(n - \delta)$. O equalizador possui dois modos de operação: um modo de treinamento durante o qual uma versão atrasada e  replicada da sequência de entrada é usada como o sinal de referência (desejado) e um modo dirigido por decisão no qual a saída do dispositivo de decisão substitui a sequência de referência. O sinal de entrada $s(n)$ é escolhido de uma constelação QAM (por exemplo, 4-QAM, 16-QAM, 64-QAM ou 256-QAM).
				
				\begin{figure}[H]
					\centering
					\includegraphics[width=0.85\textwidth]{figs/equalizador_linear.png}
					\caption{Equalizador Linear}
				\end{figure}
			
				\begin{enumerate}
					
					\item Faça um programa que treine o filtro adaptativo com 500 símbolos de uma constelação 4-QAM, seguindo de uma operação dirigida por decisão de 5000 símbolos de uma constelação 16-QAM. Escolha a variância do ruído $\sigma^{2}_{v}$ de maneira que ela promova uma relação sinal ruído de 30 db na entrada do equalizador. Note que os símbolos escolhidos não têm variância unitária. Por esta razão, a a variância do ruído necessita ser ajustada adequadamente para cada uma das diferentes modulações (constelações) QAM para fornecer o nível de SNR desejado. Escolha $\delta = 15$ e o comprimento do equalizador M = 15. Mostre os gráficos da evolução temporal de $s(n)$, $x(n)$ e $\overset{\sim}{s}(n - \delta)$. Use o LMS-normalizado com um fator de passo de $\mu = 0.4$.
								
						\textcolor{red}{Solução:}

						\paragraph{}Os resultados estão nas Figuras \ref{fig:L3Q6A1} e \ref{fig:L3Q6A2}. A evolução temporal do MSE apresenta comportamento similar aos resultados apresentados anteriormente com
						o MSE possuindo uma grande variação mesmo após a convergência. Por fim, também é possível ver na figura seguinte o impacto no filtro na filtragem de um sinal modulado por 16-QAM. É visível 
						que a estimação conseguiu separar o sinal transmitido em diferentes regiões de decisão e assim seria necessário ao fim desse processo passar esse sinal filtrado por um decisor para que fosse
						obtida uma aproximação do sinal originalmente trnasmitido. Ademais, foi implementado um simples processo de aprendizado utilizando algumas amostras de sinais 4-QAM para o treinamento do filtro com
						o objetivo de facilitar ou acelerar a convergência quando o sinal 16-QAM fosse filtrado. 
						
						\paragraph{}Além disso, na Figura \ref{fig:L3Q6A3} é possível acompanhar a evolução temporal do sinal filtrado e do sinal 
						original para dois momentos distintos. No primeiro, temos a adaptação do filtro para as primeiras 100 amostras e previsivelmente confirmamos que o filtro possuí ainda dificuldades em acompanhar o canal 
						nas primeiras amostras. Já no segundo momento temos as últimas 100 amostras do filtro e podemos entender a partir disso que o filtro agora tem capacidade de acompanhar o canal de forma mais eficiente, embora
						ainda existam momentos onde erros ocorrem. Esses erros podem ser ocasionados pelas interferências ruidosas que provocam erros de detecção na hora de avaliar as constelações aproximadas que saem do processo de filtragem.

					\item Para os mesmos parâmetros do item (a), plote e compare os gráficos de evolução que seriam resultante se o equalizador fosse treinado com 150, 300 e 500 iterações. Use o LMS com um
					$\mu = 0.001$.
					
						\textcolor{red}{Solução:}

						\paragraph{}O resultado está na Figura \ref{fig:L3Q6B}. Particularmente não pude distinguir notáveis diferenças de desempenho ao considerar diferentes tamanhos de sequências de treinamento.

					\item Assuma agora que os dados transmitidos foram gerados de uma constelação 256-QAM ao invés de 16-QAM. Plote os gráficos da evolução do sinal na saída do equalizador quando treinado
					usando o LMS-normalizado e 500 símbolos de treinamento.					
					
						\textcolor{red}{Solução:}

						\paragraph{}Os resultados estão nas Figuras \ref{fig:L3Q6C1} e \ref{fig:L3Q6C2}. É possível verificar na primeira figura que a evolução do MSE aconteceu apesar de uma considerável variação ao final
						da convergência. Já em relação a figura seguinte, onde transmitimos um sinal 256-QAM e utilizamos um filtro treinado por um sinal 4-QAM, é possível verificar que inicialmente o filtro demonstrou dificuldade em acompanhar
						a evolução do canal. Por fim, em um segundo momento é apresentado a evolução temporal considerando as ultimas amostras do sinal e aqui podemos visualizar que, embora ainda existam uma quantidade considerável de erros por se 
						tratar de uma modulação de ordem elevada, o filtro consegue se aproximar com um pouco mais de facilidade do sinal original. 
						
						\paragraph{}Adicionalmente, para termos uma melhor interpretação do desempenho desse filtro seria necessário utilizar 
						uma métrica como a SER ou BER e comparar o impacto de diferentes modulações no desempenho do algoritmo de filtragem. Isso se deve pois quando utilizamos um decisor o símbolo filtrado pode ser alocado para uma região da constelação
						da modulação a qual não pertence. Entretanto, tais erros de decisão so são aparente quando utilizamos uma métrica capaz de capturar tal fenômeno.

					\item Gerar as curvas de taxa de erro de símbolo (SER, do inglês Symbol Error Rate) versus SNR na entrada do equalizador para símbolos de constelações 4, 16, 64 e 256-QAM. Faça SNR variar
					de 5dB a 30dB.
					
						\textcolor{red}{Solução:}

						\paragraph{}Os resultados estão na Figura \ref{fig:L3Q6D}. Assim, podemos analisar o real desempenho do filtro quando associado a um equalizador que desconhece o sinal verdadeiro. 
						É possível verificar que a medida que a ordem de modulação aumenta o desempenho sofre uma considerável piora. Existem duas principais razões para explicar esse comportamento. A primeira vem justamente 
						do tamanho das constelações digitais que a medida que a ordem aumenta tem uma menor regiao de decisão associada aos seus símbolos. Desse modo, o sinal transmitido fica sujeito a interferências de componentes 
						ruidosas do canal pois as divisórias entre as diversas regiões de decisão dos símbolos da modulação ficam menores e erros de decisão irão acontecer com maior frequência A segunda razão poderia ser explicada pelo 
						fato do filtro ter sido inicialmente treinado utilizando símbolos modulados com 4-QAM, mas provavelmente isso é insuficiente quando a ordem de modulação do sinal transmitido cresce.
						Ademais, vale ressaltar que não houve erros para SNR $= 30$ dB quando se transmitiu sinais 4-QAM, entretanto como foram utilizados apenas 1000 realizações de Monte Carlo não foi
						possível capturar erros para esse ponto específico de SNR. 

				\end{enumerate}
				
				\newpage
				\begin{figure}[H]
					\centering
					\includegraphics[width=0.75\textwidth]{figs/newton_contour.png}
					\caption{Caminho de convergência na superficie MSE para o Método de Newton. $\text{Amostras} = 1000$, $M = 2$, $\mu = 10^{-3}$}
					\label{fig:newton_contour}
				\end{figure}
				
				\begin{figure}[H]
					\centering
					\includegraphics[width=0.75\textwidth]{figs/gradient_contour.png}
					\caption{Caminho de convergência na superficie MSE para o Gradiente Determinístico. $\text{Amostras} = 1000$, $M = 2$, $\mu = 10^{-3}$}
					\label{fig:gradient_contour}
				\end{figure}

				\begin{figure}[H]
					\centering
					\includegraphics[width=0.75\textwidth]{figs/lms_contour.png}
					\caption{Caminho de convergência na superficie MSE para o LMS. $\text{Amostras} = 1000$, $M = 2$, $\mu = 10^{-3}$}
					\label{fig:lms_contour}
				\end{figure}

				\begin{figure}[H]
					\centering
					\includegraphics[width=0.75\textwidth]{figs/nlms_contour.png}
					\caption{Caminho de convergência na superficie MSE para o NLMS. $\text{Amostras} = 1000$, $M = 2$, $\mu = 0.5$}
					\label{fig:nlms_contour}
				\end{figure}

				\begin{figure}[H]
					\centering
					\includegraphics[width=0.75\textwidth]{figs/newton_mse.png}
					\caption{Evolução do MSE para o Método de Newton. $\text{Amostras} = 1000$, $M = 2$, $\mu = 10^{-3}$}
					\label{fig:newton_mse}
				\end{figure}
				
				\begin{figure}[H]
					\centering
					\includegraphics[width=0.75\textwidth]{figs/gradient_mse.png}
					\caption{Evolução do MSE para o Gradiente Determinístico. $\text{Amostras} = 1000$, $M = 2$, $\mu = 10^{-3}$}
					\label{fig:gradient_mse}
				\end{figure}

				\begin{figure}[H]
					\centering
					\includegraphics[width=0.75\textwidth]{figs/lms_mse.png}
					\caption{Evolução do MSE para o LMS. $\text{Amostras}  = 1000$, $M = 2$, $\mu = 10^{-3}$}
					\label{fig:lms_mse}
				\end{figure}

				\begin{figure}[H]
					\centering
					\includegraphics[width=0.75\textwidth]{figs/nlms_mse.png}
					\caption{Evolução do MSE para o NLMS. $\text{Amostras} = 1000$, $M = 2$, $\mu = 0.5$}
					\label{fig:nlms_mse}
				\end{figure}

				\begin{figure}[H]
					\centering
					\includegraphics[width=0.75\textwidth]{figs/nlms_awgn.png}
					\caption{Evolução do MSE para o NLMS em um Canal AWGN 30dB. $\text{Amostras} = 1000$, $M = 15$, $\mu = 0.5$}
					\label{fig:nlms_awgn}
				\end{figure}

				\begin{figure}[H]
					\centering
					\includegraphics[width=0.75\textwidth]{figs/L3Q5_mu_2.png}
					\caption{$\text{Amostras} = 1000$, $M = 15$, $\mu = \frac{\mu_{\text{max}}}{2}$}
					\label{fig:mu_2}
				\end{figure}

				\begin{figure}[H]
					\centering
					\includegraphics[width=0.75\textwidth]{figs/L3Q5_mu_10.png}
					\caption{$\text{Amostras} = 1000$, $M = 15$, $\mu = \frac{\mu_{\text{max}}}{10}$}
					\label{fig:mu_10}
				\end{figure}

				\begin{figure}[H]
					\centering
					\includegraphics[width=0.75\textwidth]{figs/L3Q5_mu_50.png}
					\caption{$\text{Amostras} = 1000$, $M = 15$, $\mu = \frac{\mu_{\text{max}}}{50}$}
					\label{fig:mu_50}
				\end{figure}

				\begin{figure}[H]
					\centering
					\includegraphics[width=0.75\textwidth]{figs/L3Q5_filter_response_2.png}
					\caption{$\text{Amostras} = 1000$, $M = 15$, $\mu = \frac{\mu_{\text{max}}}{2}$}
					\label{fig:filter_response_2}
				\end{figure}

				\begin{figure}[H]
					\centering
					\includegraphics[width=0.75\textwidth]{figs/L3Q5_filter_response_10.png}
					\caption{$\text{Amostras} = 1000$, $M = 15$, $\mu = \frac{\mu_{\text{max}}}{10}$}
					\label{fig:filter_response_10}
				\end{figure}

				\begin{figure}[H]
					\centering
					\includegraphics[width=0.75\textwidth]{figs/L3Q5_filter_response_50.png}
					\caption{$\text{Amostras} = 1000$, $M = 15$, $\mu = \frac{\mu_{\text{max}}}{50}$}
					\label{fig:L3Q5_t}
				\end{figure}

				\begin{figure}[H]
					\centering
					\includegraphics[width=0.75\textwidth]{figs/L3Q5_t.png}
					\caption{$\text{Amostras} = 1000$, $M = 15$, $\mu = \frac{\mu_{\text{max}}}{2}$}
					\label{fig:filter_response_50}
				\end{figure}

				\begin{figure}[H]
					\centering
					\includegraphics[width=0.75\textwidth]{figs/L3Q6_A_mse.png}
					\caption{$\text{Amostras} = 5000$, $M = 15$, $\mu = 0.4$}
					\label{fig:L3Q6A1}
				\end{figure}

				\begin{figure}[H]
					\centering
					\includegraphics[width=0.75\textwidth]{figs/L3Q6_A_c.png}
					\caption{$\text{Amostras} = 5000$, $M = 15$, $\mu = 0.4$}
					\label{fig:L3Q6A2}
				\end{figure}

				\begin{figure}[H]
					\centering
					\includegraphics[width=0.75\textwidth]{figs/L3Q6_A_t.png}
					\caption{$\text{Amostras} = 5000$, $M = 15$, $\mu = 0.4$}
					\label{fig:L3Q6A3}
				\end{figure}

				\begin{figure}[H]
					\centering
					\includegraphics[width=0.75\textwidth]{figs/L3Q6_B_t.png}
					\caption{$\text{Amostras} = 5000$, $M = 15$, $\mu = 0.001$}
					\label{fig:L3Q6B}
				\end{figure}

				\begin{figure}[H]
					\centering
					\includegraphics[width=0.75\textwidth]{figs/L3Q6_C_mse.png}
					\caption{$\text{Amostras} = 5000$, $M = 15$, $\mu = 0.4$}
					\label{fig:L3Q6C1}
				\end{figure}

				\begin{figure}[H]
					\centering
					\includegraphics[width=0.75\textwidth]{figs/L3Q6_C_t.png}
					\caption{$\text{Amostras} = 5000$, $M = 15$, $\mu = 0.4$}
					\label{fig:L3Q6C2}
				\end{figure}

				\begin{figure}[H]
					\centering
					\includegraphics[width=0.75\textwidth]{figs/L3Q6_D_ser.png}
					\caption{$\text{Amostras} = 5000$, $M = 15$, $\mu = 0.4$}
					\label{fig:L3Q6D}
				\end{figure}

		\end{enumerate}
	
	\newpage
	\section*{Método de Mínimos Quadrados}

		\begin{enumerate}
			
			\item O algoritmo RLS é utilizado para prever o sinal $x(n) = \text{cos} (\frac{\pi n}{3})$ usando um filtro FIR de segunda ordem com o primero 
			coeficiente fixo em $1$. Dado $\lambda = 0.98$, calcule o sinal de saída $y(n)$ e os coeficientes do filtro nas primeiras $10$ iterações. 
			Note que a meta é minimizar $\mathbb{E}\{y^{2}(n)\}$. Inicie com $\boldsymbol{w} = [1 0 0]^{\text{T}}$ e $\delta = 100$.
 
				\textcolor{red}{Solução:}

				\begin{table}[H]
					\centering
					\begin{tabular}{|l|l|l|l|}
						\hline
						Iterations & $w_{0}$ & $w_{1}$ & $w_{2}$ \\ 
						\hline 
						i = 1 & 1.0000 & 0.000000 & 0.000000 \\ \hline
						i = 2 & 1.0703 & 0.094579 & 0.087444 \\ \hline
						i = 3 & 0.63832 & 0.67774 & -0.22783 \\ \hline
						i = 4 & 0.25922 & 0.43298 & 0.28991 \\ \hline
						i = 5 & 0.26607 & 0.4316 & 0.28539 \\ \hline
						i = 6 & 0.29342 & 0.40148 & 0.30207 \\ \hline
						i = 7 & 0.29611 & 0.40434 & 0.29579 \\ \hline
						i = 8 & 0.3016 & 0.42552 & 0.27103 \\ \hline
						i = 9 & 0.41718 & 0.35248 & 0.23766 \\ \hline
						i = 10 & 0.38104 & 0.34597 & 0.29742 \\ \hline 
					\end{tabular}
				\end{table}

				\begin{figure}[H]
					\centering
					\includegraphics[width=0.65\textwidth]{figs/L4Q1.png}
					\caption{Primeiro coeficiente livre para adaptação com $\text{Amostras} = 100$, $M = 2$, $\lambda = 0.98$}
					\label{fig:L4Q1}
				\end{figure}

				\paragraph{} Na tabela acima foram disponibilizados os coeficientes de filtro para as 10 primeiras iterações como foi pedido. 
				A tabela e o algoritmo de filtragem foram implementados utilzando software matemático e os códigos foram disponibilizados junto com 
				o relatório. É possível observar nessas primeiras 10 iterações a convergência dos coeficientes de filtro rumo a estabilidade que poderia
				ser observada se mais algumas iterações fossem disponibilizadas na tabela. 
				
				\paragraph{}Ademais, decidi apresentar o traçado do sinal senoidal transmitido 
				comparando-o diretamente com o sinal filtrado na Figura \ref{fig:L4Q1}. É possível conferir que o sinal filtrado começa com uma amplitude bastante
				distante do sinal verdadeiro, mas a meddida que mais amostras são utilizadas para a adaptação do sinal a resposta em aplitudade do sinal filtrado se
				aproxima do comportamento ideal. Existem algumas oscilações ao longo das 100 amostras, mas sem mudanças abruptas. É necessário apenas chamar atenção 
				para o comportamento repentino apresentado no final do processo que é uma consequência da minha implementação para o RLS. Considero que o filtro inicia 
				com sua janela totalmente preenchida o que por consequência induz a interupção do algoritmo quando a janela chega a sua última seção onde está totalmente preenchida.

				\paragraph{}Por fim, na tabela abaixo e na Figura \ref{fig:L4Q2} apresento os resultados para o cenário onde o primeiro coeficiente do filtro RLS é mantido fixo em 1.
				Podemos ver de mais imediato que os coeficientes que estão livres para a adaptação apresentam considerável instabilidade quando comparados com o cenário onde os três 
				coeficientes estão livres. Consequentemente, o filtro tem uma dificuldade bem maior em acompanhar as mudanças do sinal de referência quando comparamos com o
				resultado da Figura \ref{fig:L4Q1}.

				\begin{table}[H]
					\centering
					\begin{tabular}{|l|l|l|l|}
						\hline
						Iterations & $w_{0}$ & $w_{1}$ & $w_{2}$ \\ 
						\hline 
						i = 1 & 1 & 0 & 0 \\ \hline 
						i = 2 & 1 & -0.043423 & -0.044953 \\ \hline 
						i = 3 & 1 & -0.031108 & -0.038147 \\ \hline 
						i = 4 & 1 & -0.030954 & -0.037966 \\ \hline 
						i = 5 & 1 & -0.021256 & -0.027157 \\ \hline 
						i = 6 & 1 & -0.016635 & -0.01361 \\ \hline 
						i = 7 & 1 & -0.0091408 & -0.0052317 \\ \hline 
						i = 8 & 1 & -0.011328 & -0.0066525 \\ \hline 
						i = 9 & 1 & -0.017649 & -0.00074347 \\ \hline 
						i = 10 & 1 & -0.0095932 & -0.012379 \\ \hline 
					\end{tabular}
				\end{table}
				
				\begin{figure}[H]
					\centering
					\includegraphics[width=0.65\textwidth]{figs/L4Q2.png}
					\caption{Primeiro coeficiente fixo com $\text{Amostras} = 100$, $M = 2$, $\lambda = 0.98$}
					\label{fig:L4Q2}
				\end{figure}

			\item Seja $\epsilon(n)$ que denota um erro de estimação a priori
			
				\begin{align}
					\epsilon(n) = d(n) - \boldsymbol{w}^{\text{H}}(n - 1) \boldsymbol{x}(n),
				\end{align}

				em que $d(n)$ é a resposta desejada, $x(n)$ é o vetor de entrada do filtro e $\boldsymbol{w}(n - 1)$ é a estimativa
				anterior do vetor de coeficientes do filtro. Seja $e(n)$ o erro de estimação a posteriori

				\begin{align}
					e(n) = d(n) - \boldsymbol{w}^{\text{H}}(n) \boldsymbol{x}(n),
				\end{align}

				em que $\boldsymbol{w}(n)$ é a estimativa atual do vetor de coeficientes do filtro. Para dados complexos ambos
				$\epsilon(n)$ e $e(n)$ são de valores complexos. Mostre que o produto $\epsilon(n)e^{*}(n)$ é sempre de valor real.

				\textcolor{red}{Solução:}

				\paragraph{}Primeiramente é necessário reescrever $e(n)$ em termos de $\epsilon(n)$. Inicialmente podemos escrever os coeficientes de filtro do instante $n$,
				utilizando o erro a priori, da seguinte forma

				\begin{align}
					\boldsymbol{w}(n) = \boldsymbol{w}(n-1) + \epsilon(n) \boldsymbol{S}_{D}(n) \boldsymbol{x}(n),
				\end{align}

				e em seguida substituímos a expressão acima na definição do erro de estimação instantâneo e obtemos 

				\begin{align}
					e(n) &= d(n) - \boldsymbol{w}^{\text{H}}(n) \boldsymbol{x}(n), \\
					e(n) &= d(n) - \boldsymbol{x}^{\text{H}}(n) \boldsymbol{w}(n), \\
					e(n) &= d(n) - \boldsymbol{x}^{\text{H}}(n) \left[\boldsymbol{w}(n-1) + \epsilon(n) \boldsymbol{S}_{D}(n) \boldsymbol{x}(n)\right], \\
					e(n) &= d(n) - \boldsymbol{x}^{\text{H}}(n) \boldsymbol{w}(n-1) - \boldsymbol{x}^{\text{H}}(n) \epsilon(n) \boldsymbol{S}_{D}(n) \boldsymbol{x}(n), \\
					e(n) &= \underbrace{d(n) - \boldsymbol{x}^{\text{H}}(n) \boldsymbol{w}(n-1)}_{\epsilon(n)} - \epsilon(n) \boldsymbol{x}^{\text{H}}(n) \boldsymbol{S}_{D}(n) \boldsymbol{x}(n), \\
					e(n) &= \epsilon(n) - \epsilon(n) \boldsymbol{x}^{\text{H}}(n) \boldsymbol{S}_{D}(n) \boldsymbol{x}(n), \\
				\end{align}

				e em sequência desenvolvemos o conjugado de $e(n)$ do seguinte modo

				\begin{align}
					e^{*}(n) &= \epsilon^{*}(n) - \epsilon^{*}(n) \boldsymbol{x}^{\text{H}}(n) \boldsymbol{S}_{D}(n) \boldsymbol{x}(n),
				\end{align}

				onde o termo $\boldsymbol{x}^{\text{H}}(n) \boldsymbol{S}_{D}(n) \boldsymbol{x}(n)$ é a distância de Mahalanobis que será sempre real e positiva.
				Desse modo, ao fazermos $\epsilon(n) e^{*}(n)$ obtemos 

				\begin{align}
					\epsilon(n) e^{*}(n) &= \epsilon(n) \left[\epsilon^{*}(n) - \epsilon^{*}(n) \boldsymbol{x}^{\text{H}}(n) \boldsymbol{S}_{D}(n) \boldsymbol{x}(n)\right], \\
					\epsilon(n) e^{*}(n) &= \epsilon(n) \epsilon^{*}(n) - \epsilon(n) \epsilon^{*}(n) \boldsymbol{x}^{\text{H}}(n) \boldsymbol{S}_{D}(n) \boldsymbol{x}(n), \\
					\epsilon(n) e^{*}(n) &= \epsilon(n) \epsilon^{*}(n) \left(1 - \boldsymbol{x}^{\text{H}}(n) \boldsymbol{S}_{D}(n) \boldsymbol{x}(n)\right).
				\end{align}

				\paragraph{}Dessa forma, sabemos que o termos $\left(1 - \boldsymbol{x}^{\text{H}}(n) \boldsymbol{S}_{D}(n) \boldsymbol{x}(n)\right)$ será sempre real, embora nem sempre positivo,
				e $\epsilon(n) \epsilon^{*}(n)$ pode simplesmente ser visto como uma norma dada por $||\epsilon(n)||^{2} = \epsilon(n) \epsilon^{*}(n)$, demonstrando assim que sempre 
				teremos um valor real para $\epsilon(n) e^{*}(n)$.

			\item Seja um sinal $x(n)$ composto de uma senóide em meio à ruído. Simule um preditor adaptativo
			de ordem 2 com um algoritmo RLS considerando $\text{SNR} = 3 dB$ e $\text{SNR} \rightarrow \infty$. Variando o fator
			de esquecimento e/ou as condições iniciais verifique e comente sobre a ocorrência ou não de
			instabilidade numérica. Repita o procedimento como preditor de ordem 3.

				\textcolor{red}{Solução:}

				\paragraph{}Os resultados podem ser encontrados nas Figuras \ref{fig:L4Q3_a1} - \ref{fig:L4Q3_a8}. Nas figuras \ref{fig:L4Q3_a1} e \ref{fig:L4Q3_a2} temos a evolução
				dos coeficientes de filtro e do MSE para o RLS com fator de esquecimento $\lambda = 0.9$, SNR = $3$ dB e ordens $M = 2$ e $M = 3$, respectivamente. Nesses cenários podemos 
				ver que o RLS não atingiu a estabilidade em seus coeficientes de filtro embora tenha apresentado um comportamento MSE sem grandes variâncões de magnitude. Ademais, no segundo cenário
				é possível verificar maior estabilidade na evolução do MSE graças ao incremento na ordem do RLS. Em sequência, nas Figuras \ref{fig:L4Q3_a3} e \ref{fig:L4Q3_a4} temos dois cenárioss similares, 
				mas agora com um fator dde conhecimento igual a $\lambda = 0.99$. Diferentemente dos dois primeiros cenários agora é possível verificar que o RLS atingiu estabilidade em seus coeficientes de filtro, 
				além de um melhor desempenho na evolução do MSE do que nos casos anteriores. Isso se deve pois, diferente do que ocorre com o passo de aprendizado nos algoritmos estudados anteriormente, a medida que
				$\lambda$ cresce menos fléxivel torna-se o filtro. Desse modo, é mais fácil para esses novos cenários adaptem-se à evolução do canal com maior facilidade. Alem disso, novamente foi possível observar 
				o impacto da ordem do RLS na estabilidade da evolução do MSE, onde para o cenário $M = 3$ foi verificado um melhor desempenho do que para o cenário $M = 2$.

				\paragraph{}Por fim, nos resta analisar o impacto da SNR na estabilidade e desempenho do RLS. Nas Figuras \ref{fig:L4Q3_a5} e \ref{fig:L4Q3_a6} temos o equivalente ao primeiro par de cenários, mas agora 
				com a diferença de que ambos os cenários são de SNR infinita. Para o primeiro caso com $M = 2$ vemos uma estabilização perfeita dos coeficientes do filtro e uma evolução do erro MSE que tende ao erro de
				precisão da máquina. Dessa forma, podemos considerar esse um cenário ideal para o RLS. Sequencialmente, na figura seguinte consideramos um cenário $M = 3$ e podemos verificar que não existiu convergência para
				esse cenário. Inicialmente existiu uma certa estabilização na evolução do MSE, mas apos algumas iterações o filtro perdeu a estabilidade e seus coeficientes "explodiram". Esse comportamento pode ser explicado 
				tanto pelo aumento da ordem do filtro quanto pelo fator de esquecimento que torna o filtro muito pouco fléxivel e suscetível a instabilidades provocadas por mudanças repentinas. Por fim, nas Figuras \ref{fig:L4Q3_a7}
				e \ref{fig:L4Q3_a8} temos dois resultados estáveis na evolução do MSE e nos coeficientes do filtro RLS. Isso se deve principalmente a um fator de esquecimento que fornece maior capacidade de resistir a instabilidades
				geradas pelo canal.

			\item Considere um sinal branco gaussiano de variância unitária transmitido por um canal de comunicação
			de função de transferência $H(z) = 1 + 1.6z^{-1}$. Para compensar este canal utiliza-se um equalizador
			dado por $W(z) = w_{0} + w_{1} z^{-1}$. (Problema da lista de exercícios no. 3).
	
				\begin{enumerate}
					
					\item Calcule a adaptação do algoritmo usando o RLS e mostre a evolução temporal dos coeficientes.
				
						\textcolor{red}{Solução:}

						\paragraph{}A evolução temporal dos coeficientes do filtro pode ser encontrada na Figura \ref{fig:rls_coefficient}.  
						Aqui é possível verificar a convergência e estabilização dos coeficientes do filtro partindo do zero a medida que o número 
						de iterações cresce. Existem algumas oscilações ao final do processo, mas nada considerável. Caso o fator de esquecimento fosse
						aumentado possivelmente haveriam oscilações com maiores magnitudes nos coeficientes de filtro. 

					\item Obtenha as trajetórias sobre as curvas de nível, tendo condições iniciais nulas para os
					coeficientes do equalizador. Verifique qual a melhor inicialização do algoritmo RLS. Compare
					com os algoritmos LMS, LMS-Normalizado e Gauss-Newton.
				
						\textcolor{red}{Solução:}

						\paragraph{}As trajetórias dos algoritmos Newton, Gradiente, LMS e NLMS estão disponíveis nas Figuras \ref{fig:newton_contour}, \ref{fig:gradient_contour}, 
						\ref{fig:lms_contour} e \ref{fig:nlms_contour}. Na Figura \ref{fig:rls_contour} é apresentado o traçado da trajetória de convergência para o RLS.
						É possível verificar que o RLS apresenta um comportamento de convergência na superfície MSE mais próximo do algoritmo LMS. Existem alguns outliers durante 
						o processo de filtragem, mas de modo geral o filtro tende à solução ótima de uma forma mais organizada e estável do que o NLMS.

					\item Obtenha também a evolução do erro quadrático médio para cada um dos algoritmos anteriores.
				
						\textcolor{red}{Solução:} 

						\paragraph{}As evolução do erro quadrático médio para os algoritmos Newton, Gradiente, LMS e NLMS já foram abordadas na seção 
						anterior e podem ser revisitadas nas Figuras \ref{fig:newton_mse}, \ref{fig:gradient_mse}, \ref{fig:lms_mse} e \ref{fig:nlms_mse}.
						adicionalmente, o traçado da evolução do MSE para o método RLS está presente na Figura \ref{fig:rls_mse}. Nessa figura podemos verificar
						que o RLS apresenta uma latência de convergência menor que o LMS e NLMS, mas ao final do processo aparentemente o grau de estabilidade da
						solução dos coeficientes de filtro parece ser menor do que quando comparamos com os quatro algoritmos de filtragem mencionados anteriormente.

				\end{enumerate}

			\item Seja a questão 6 da lista de exercícios anterior (Algoritmos Recursivos - questão sobre Equalização
			Adaptativa). Implemente o RLS para a equalização do sistema considerado na letra (a) da mesma.
			Compare os resultados obtidos com o LMS. Verifique a velocidade de convergência para os casos de
			$\lambda = 0.9$, $\lambda = 0.99$ e $\lambda = 0.999$.

				\textcolor{red}{Solução:}

				\paragraph{}Assim como na questão da lista anterior foi considerado novamente que o filtro é de ordem $M = 15$.
				Os resultados podem ser encontrados nas Figuras \ref{fig:L4Q5_a1}, \ref{fig:L4Q5_a2}, \ref{fig:L4Q5_a3} e \ref{fig:L4Q5_a4}.
				Inicialmente é possível confirmar uma evidente vantagem na velocidade de convergência do RLS quando o comparamos com o algoritmo LMS.
				Independentemente do fator de esquecimento considerado todos apresentam uma vantagem considerável sobre o LMS quando se observa tanto a 
				latência quanto o desempenho da evolução do MSE. Desse modo, o RLS tem menor latência e melhor desempenho MSE quando comparamos com o LMS.
				
				\paragraph{}Já quando voltamos nossa análise para o impacto do fator de esquecimento é possível verificar uma interessante característica do 
				RLS. O fator de esquecimento atua de forma semelhante ao passo de aprendizado nos algoritmos LMS e NLMS, mas num sentido um  pouco diferente. Desse modo, quanto
				maior o seu valor mais flexível torna-se o filtro para adaptar-se a novos estímulos. Assim, nas Figuras \ref{fig:L4Q5_a2} e \ref{fig:L4Q5_a3}, onde o 
				fator de esquecimento é definido respectivamente por $\lambda = 0.9$ e $\lambda = 0.99$, é possível ver que o MSE apresenta oscilações de 
				elevadas magnitudades mesmo após a convergência pois estamos restringindo a flexibilidade do RLS. Já na Figura \ref{fig:L4Q5_a4} onde temos um fator de esquecimento 
				definido por $\lambda = 0.999$ temos oscilações de magnitudades consideravelmente menores quando comparamos com os dois casos anteriores. Já com relação ao valor de convergência
				MSE não é possível notar grandes diferenças entre os casos considerados, visto que todos oscilam em torno de um valor aproximadamente igual.
			
				\paragraph{}Por fim, nas Figuras \ref{fig:L4Q5_a5} e \ref{fig:L4Q5_a6} são traçados os gráficos da evolução temporal e da resposta em frequência para
				o cenário proposto, respectivamente. Na primeira figura vemos a adaptação do algoritmo RLS a medida que o número de iterações progride e, apesar do
				equalizador desconhecer o sinal verdadeiro, existe uma aparente melhora de desempenho a medida que o filtro progride. Já na figura seguinte são comparados
				diretamente as repostas em frequência dos filtros LMS e RLS em relação ao sistema desconhecido com $\mu = 0.001$ e $\lambda = 0.999$, respectivamente.
				A partir de tal resultado podemos verificar que o filtro RLS consegue adaptar-se mais facilmente ao sistema desconhecido enquanto o filtro LMS apresenta
				intensas variações que possivelmente prejudicariam o processo de filtragem.

				\newpage
				\begin{figure}[H]
					\centering
					\includegraphics[width=0.825\textwidth]{figs/L4Q3_rls_mse_2_3_9.png}
					\caption{SNR = 3 dB, M = 2 and $\lambda = 0.9$}
					\label{fig:L4Q3_a1}
				\end{figure}

				\begin{figure}[H]
					\centering
					\includegraphics[width=0.825\textwidth]{figs/L4Q3_rls_mse_3_3_9.png}
					\caption{SNR = 3 dB, M = 3 and $\lambda = 0.9$}
					\label{fig:L4Q3_a2}
				\end{figure}

				\begin{figure}[H]
					\centering
					\includegraphics[width=0.825\textwidth]{figs/L4Q3_rls_mse_2_3_99.png}
					\caption{SNR = 3 dB, M = 2 and $\lambda = 0.99$}
					\label{fig:L4Q3_a3}
				\end{figure}

				\begin{figure}[H]
					\centering
					\includegraphics[width=0.825\textwidth]{figs/L4Q3_rls_mse_3_3_99.png}
					\caption{SNR = 3 dB, M = 3 and $\lambda = 0.99$}
					\label{fig:L4Q3_a4}
				\end{figure}

				\begin{figure}[H]
					\centering
					\includegraphics[width=0.825\textwidth]{figs/L4Q3_rls_mse_2_inf_9.png}
					\caption{SNR = $\infty$ dB, M = 2 and $\lambda = 0.9$}
					\label{fig:L4Q3_a5}
				\end{figure}

				\begin{figure}[H]
					\centering
					\includegraphics[width=0.825\textwidth]{figs/L4Q3_rls_mse_3_inf_9.png}
					\caption{SNR = $\infty$ dB, M = 3 and $\lambda = 0.9$}
					\label{fig:L4Q3_a6}
				\end{figure}

				\begin{figure}[H]
					\centering
					\includegraphics[width=0.825\textwidth]{figs/L4Q3_rls_mse_2_inf_99.png}
					\caption{SNR = $\infty$ dB, M = 2 and $\lambda = 0.99$}
					\label{fig:L4Q3_a7}
				\end{figure}

				\begin{figure}[H]
					\centering
					\includegraphics[width=0.825\textwidth]{figs/L4Q3_rls_mse_3_inf_99.png}
					\caption{SNR = $\infty$ dB, M = 3 and $\lambda = 0.99$}
					\label{fig:L4Q3_a8}
				\end{figure}

				\begin{figure}[H]
					\centering
					\includegraphics[width=0.75\textwidth]{figs/rls_coefficients.png}
					\caption{Convergência dos coeficientes para o RLS. $\text{Amostras} = 5000$, $M = 2$, $\lambda = 0.99$}
					\label{fig:rls_coefficient}
				\end{figure}

				\begin{figure}[H]
					\centering
					\includegraphics[width=0.75\textwidth]{figs/rls_contour.png}
					\caption{Caminho de convergência na superficie MSE para o RLS. $\text{Amostras} = 5000$, $M = 2$, $\lambda = 0.99$}
					\label{fig:rls_contour}
				\end{figure}

				\begin{figure}[H]
					\centering
					\includegraphics[width=0.75\textwidth]{figs/rls_mse.png}
					\caption{Comportamento da evolução do MSE para o RLS. $\text{Amostras} = 5000$, $M = 2$, $\lambda = 0.99$}
					\label{fig:rls_mse}
				\end{figure}

				\begin{figure}[H]
					\centering
					\includegraphics[width=0.75\textwidth]{figs/L4Q5_lms.png}
					\caption{Comportamento da evolução do MSE para o LMS. $\text{Amostras} = 500$, $M = 15$, $\mu = 0.001$}
					\label{fig:L4Q5_a1}
				\end{figure}

				\begin{figure}[H]
					\centering
					\includegraphics[width=0.75\textwidth]{figs/L4Q5_rls_9.png}
					\caption{Comportamento da evolução do MSE para o RLS. $\text{Amostras} = 500$, $M = 15$, $\lambda = 0.9$}
					\label{fig:L4Q5_a2}
				\end{figure}

				\begin{figure}[H]
					\centering
					\includegraphics[width=0.75\textwidth]{figs/L4Q5_rls_99.png}
					\caption{Comportamento da evolução do MSE para o RLS. $\text{Amostras} = 500$, $M = 15$, $\lambda = 0.99$}
					\label{fig:L4Q5_a3}
				\end{figure}

				\begin{figure}[H]
					\centering
					\includegraphics[width=0.75\textwidth]{figs/L4Q5_rls_999.png}
					\caption{Comportamento da evolução do MSE para o RLS. $\text{Amostras} = 500$, $M = 15$, $\lambda = 0.999$}
					\label{fig:L4Q5_a4}
				\end{figure}

				\begin{figure}[H]
					\centering
					\includegraphics[width=0.75\textwidth]{figs/L4Q5_rls_t.png}
					\caption{Evolução temporal para o RLS. $\text{Amostras} = 500$, $M = 15$, $\lambda = 0.999$}
					\label{fig:L4Q5_a5}
				\end{figure}

				\begin{figure}[H]
					\centering
					\includegraphics[width=0.75\textwidth]{figs/L4Q5_filter_response.png}
					\caption{Comparativo da resposta em frequência para LMS e RLS. $\text{Amostras} = 500$, $M = 15$, $\lambda = 0.999$}
					\label{fig:L4Q5_a6}
				\end{figure}

		\end{enumerate}

\end{document}